\documentclass[10pt,a4paper]{article}
\usepackage{geometry}
\geometry{margin=2cm, left=1cm, right=1cm}
\usepackage{lmodern}
\usepackage{verse}
\usepackage[utf8]{inputenc}
\usepackage[program=/usr/bin/lilypond]{lyluatex}
\usepackage{svg}
\svgpath{svg/}

\usepackage{multicol}
\usepackage{paracol}
\setlength\columnseprule{.4pt}

\begin{document}


\begin{titlepage}
\title{Seven Secular Sermons}
\date{}
\maketitle
%    \centering
%    {\huge Seven Secular Sermons\par} % that \par is necessary for the following \vspace to take effect
%    \vspace{2.5cm}
%    \includesvg[scale=0.8]{7ss_hi_res_logo}
\end{titlepage}

\tableofcontents
\pagebreak

\begin{paracol}{2}

\poemtitle{Adrift in Space and Time}

\begin{verse}
Now this is going to be fun!\\
It truly does feel great\\
to realize we all are one.\\
So we shall meditate.
\end{verse}

\begin{verse}
This meditation’s rhyming verse\\
describes a paradigm\\
of us inside this universe,\\
adrift in space and time.
\end{verse}

\begin{verse}
It’s nice that we can simply start\\
by recognizing how\\
a check of every body part\\
can help us be here now.
\end{verse}

\begin{verse}
We feel our bellies and our heads\\
and just become aware\\
how arms and hands and feet and legs\\
are feeling everywhere.
\end{verse}

\begin{verse}
Between and through them circulate\\
our blood-streams to refresh\\
the oxygen that activates\\
awareness in our flesh.
\end{verse}

\begin{verse}
And as we slowly breathe we find\\
that doing so can draw\\
us into a more present mind\\
to feel the moment raw.
\end{verse}

\begin{verse}
With every breath we take and leave,\\
we clear our inner eyes\\
and fully, lucidly perceive\\
each second passing by.
\end{verse}

\begin{verse}
Our mindful meditative selves\\
grow out of living meat\\
and help our bodies stay in health\\
by finding what we need.
\end{verse}

\begin{verse}
So let’s do that now. Let’s explore\\
and see what’s to be found.\\
Outside our bodies, there is more.\\
Let’s take a look around!
\end{verse}

\begin{verse}
Unless we’re blind we’re free to see,\\
unless we’re deaf, to hear\\
and realize we’re utterly\\
surrounded by what’s here.
\end{verse}

\begin{verse}
This place surrounding us here now\\
where we consider this,\\
is just as present, anyhow,\\
as our breathing is.
\end{verse}

\begin{verse}
Our breaths connect within the air,\\
within the atmosphere.\\
The envelope of sky we share\\
is also part of here.
\end{verse}

\begin{verse}
We also share what rests beneath:\\
Our bodies’ place of birth\\
from which came all who now here breathe\\
as children of the Earth.
\end{verse}

\begin{verse}
Of course there’s more than senses show\\
around us near and far.\\
The sky above, the Earth below;\\
there’s more to where we are.
\end{verse}

\begin{verse}
To North and South, to West and East,\\
the world goes on and on,\\
the planet every plant and beast\\
and we now breathe upon.
\end{verse}

\begin{verse}
Of all the Earth, we barely know\\
the surface we begrime,\\
upon the spinning rock below,\\
adrift in space and time.
\end{verse}

\begin{verse}
Our calm and meditating minds\\
can feel this easily.\\
Imagination goes behind\\
all things our eyes can see.
\end{verse}

\begin{verse}
To find, as further out we go,\\
whichever way we face,\\
to left, to right, above, below,\\
the solar system. Space.
\end{verse}

\begin{verse}
It’s blacker than the night of course\\
and bigger than the sky\\
and it is hard to see because\\
it was not made for eyes.
\end{verse}

\begin{verse}
It effortlessly overwhelms\\
imagination. Still\\
and everywhere around this realm\\
extends and always will.
\end{verse}

\begin{verse}
Around us all and everyone\\
we’ve met or ever can,\\
extends the system of the Sun\\
that dwarfs all realms of man.
\end{verse}

\begin{verse}
Out there, all human joy and strife\\
and knowledge matter not.\\
Out there, this fragile ball of life\\
is just a pale blue dot.
\end{verse}

\begin{verse}
And there are other, bigger dots\\
and countless asteroids.\\
This Earth is one among a lot\\
around us in the void.
\end{verse}

\begin{verse}
Yet all of them combined appear\\
like specks of dust compared\\
to one enormous blazing sphere,\\
the center that they share.
\end{verse}

\begin{verse}
A thousand times as ponderous\\
as all that circles it,\\
it radiates splendiferous\\
and indiscriminate. \\
\end{verse}

\begin{verse}
It weighs three-hundred-thirty-three\\
times thousand times as much\\
as Earth, which seems like mere debris,\\
a tiny circling smudge.
\end{verse}

\begin{verse}
There’s hydrogen inside the Sun\\
that lets it shine so bright\\
by burning up: Four million tons\\
per second fuse to light.
\end{verse}

\begin{verse}
This fusion forges helium\\
and other specks of dust\\
that constitute the medium\\
from which grows life like us.
\end{verse}

\begin{verse}
But near the Sun, its gamma rays\\
and heat do not allow\\
life smart enough to be amazed\\
at what is true here now.
\end{verse}

\begin{verse}
And further out, it stays too cold\\
for molecules to toy\\
with games of entropy that mold\\
the life that we enjoy.
\end{verse}

\begin{verse}
While outermost, in blackest night,\\
drift frozen rocks so far,\\
to them our splendid sphere of light\\
looks like another star.
\end{verse}

\begin{verse}
We’re lucky Earth is temperate\\
or life could not have spawned.\\
This planet would stay desolate\\
and all of us unborn.
\end{verse}

\begin{verse}
As fully as we do depend\\
on Earth that we live on,\\
we also clearly understand\\
we’re children of the Sun.
\end{verse}

\begin{verse}
And yet the Sun, though all must spin\\
around it, merely is\\
a rare domain of light within\\
a yawning black abyss.
\end{verse}

\begin{verse}
In outer space surrounding us\\
lie distances too great\\
for us to easily discuss\\
or even contemplate.
\end{verse}

\begin{verse}
For space is mostly nothingness\\
around us everywhere,\\
the freezing dark is limitless\\
in empty space out there.
\end{verse}

\begin{verse}
Of course there is some gravity\\
that massive things impart\\
and maybe some dark energy\\
that pushes them apart.
\end{verse}

\begin{verse}
But nothing’s there to hear or see\\
or smell or taste or touch\\
and trying to imagine, we\\
can think of nothing much.
\end{verse}

\begin{verse}
And still we feel, for what that’s worth,\\
beyond the seen and near\\
the vastness outside planet Earth\\
that’s real now and here.
\end{verse}

\begin{verse}
In meditation, we somehow\\
expand our minds to try\\
to feel the system we are now\\
and here surrounded by.
\end{verse}

\begin{verse}
The moons and planets we can see,\\
as far as we have found,\\
are lifeless. Earth now seems to be\\
the only game in town.
\end{verse}

\begin{verse}
Yet all these places we could go\\
and cultivate and fill,\\
are merely specks in what we know\\
remains much bigger still.
\end{verse}

\begin{verse}
The stars, these many tiny lights,\\
each are a blazing sun\\
and circling them, caught in their might,\\
are planets being spun.
\end{verse}

\begin{verse}
Yet humans cannot see that far.\\
The pixels of our eyes\\
are just too few, which is why stars\\
look like they’re equal-sized.
\end{verse}

\begin{verse}
Through telescopes, we understood.\\
The stars all shine so bright\\
that only monstrous distance could\\
dilute them into night.
\end{verse}

\begin{verse}
These distances define the space\\
that all stars occupy\\
and make a single, real place\\
that we’re surrounded by.
\end{verse}

\begin{verse}
The stars that shine all night and day\\
within or out of sight\\
are what our home, the Milky Way,\\
appears like from inside.
\end{verse}

\begin{verse}
Our Milky Way contains at least\\
one hundred billion suns.\\
Through gravity, they all are pieced\\
together into one.
\end{verse}

\begin{verse}
Around this place, where we now feel\\
what we are breathing in,\\
these suns form one tremendous wheel\\
with one tremendous spin.
\end{verse}

\begin{verse}
And all these suns are shining clear,\\
enormous and sublime.\\
They all are real here where we’re\\
adrift in space and time.
\end{verse}

\begin{verse}
Unmoved by beings such as we\\
on Earth, our small enclave,\\
the stars around us now will be\\
the stars around our graves.
\end{verse}

\begin{verse}
Except for those which have gone through\\
their hydrogen supply\\
and end as all things someday do,\\
for even stars must die.
\end{verse}

\begin{verse}
And some, much bigger than our Sun,\\
burn brighter still and must\\
explode one day, when they are done\\
with making light and dust.
\end{verse}

\begin{verse}
These supernovas, as we call\\
them, burst stupendously.\\
Some can outshine the sum of all\\
stars in the galaxy.
\end{verse}

\begin{verse}
With their magnetic fields unfurled,\\
their yields annihilate\\
or sterilize abundant worlds\\
that life might populate.
\end{verse}

\begin{verse}
Yet all we breathe and eat and drink\\
comes from these massive bombs.\\
We’re supernova-dust that thinks\\
about where it came from.
\end{verse}

\begin{verse}
And since the stars have made the clay\\
that led to our birth,\\
we’re children of the Milky Way,\\
as are the Sun and Earth.
\end{verse}

\begin{verse}
But supernovas are quite rare.\\
Three times per century\\
does one of them explode somewhere\\
within our galaxy.
\end{verse}

\begin{verse}
Yet many supernovas do\\
each second detonate\\
in all the galaxies whereto\\
we now shall escalate!
\end{verse}

\begin{verse}
A million times much further out\\
than all the Milky Way,\\
more galaxies are shining proud\\
around us here today.
\end{verse}

\begin{verse}
These galaxies, each huge and wide,\\
much like the one we’re in,\\
outnumber all the stars inside\\
our home and origin.
\end{verse}

\begin{verse}
Around where we consider this,\\
whichever way we face,\\
drift billions of these galaxies\\
right now, right here, in space.
\end{verse}

\begin{verse}
We realize with utter awe\\
and know beyond all doubt:\\
Beyond this world are trillions more\\
that we could learn about.
\end{verse}

\begin{verse}
And almost all of them must be\\
absurdly far away\\
in ultimate reality\\
beyond the Milky Way.
\end{verse}

\begin{verse}
From here where our bodies stay,\\
imagination climbs\\
through further outer Milky Ways\\
adrift in space and time.
\end{verse}

\begin{verse}
And through the emptiness between\\
in almost all of space,\\
where not a single star is seen\\
in almost every place.
\end{verse}

\begin{verse}
And meditation does allow\\
our minds to feel it all.\\
To feel the Universe that now\\
surrounds us as a whole.
\end{verse}

\begin{verse}
Despite all suns that intersperse\\
this dark continuum,\\
most places in this Universe\\
are total vacuum.
\end{verse}

\begin{verse}
And therefore, atoms are quite rare.\\
Yet trillions of them have\\
condensed into the flesh we wear\\
that draws this very breath.
\end{verse}

\begin{verse}
Two thirds of atoms in us are\\
still hydrogen which sprang\\
into existence not in stars\\
but back in the Big Bang.
\end{verse}

\begin{verse}
For all the time since time began,\\
as entropy made space,\\
each travelled an enormous span\\
to meet here face to face.
\end{verse}

\begin{verse}
Through vacuum and solar flame,\\
they found their way somehow.\\
And we as that which they became,\\
thus came to meet here now.
\end{verse}

\begin{verse}
Through all we breathe and drink and eat,\\
they travel and endow\\
with nutrients the living meat\\
in which we meet here now.
\end{verse}

\begin{verse}
The atoms that we are traverse\\
all space and time, which means\\
we’re children of the Universe\\
and we have always been.
\end{verse}

\begin{verse}
The atoms in us met before\\
and they will meet again,\\
compelled by universal law\\
out in the there and then.
\end{verse}

\begin{verse}
One endless cosmic maelstrom,\\
age-old and ever new,\\
is where we all are coming from\\
and where we’re going to.
\end{verse}

\begin{verse}
The knowledge we are made of dust\\
compels us to admit\\
the Universe is in us just\\
as we are within it.
\end{verse}

\begin{verse}
From here we may arise to see\\
and claim as our own\\
the secrets of reality\\
just waiting to be known.
\end{verse}

\begin{verse}
And so we know the infinite\\
is absolutely real.\\
It’s here, it’s now, it’s intimate,\\
this vastness that we feel.
\end{verse}

\begin{verse}
Whatever else is true for us,\\
we’ll always know this rhyme.\\
We’ll always know we’re made of dust\\
adrift in space and time.
\end{verse}

\switchcolumn

\poemtitle{Im Flug Durch Raum Und Zeit}

\begin{verse}
Genießen wir, was jetzt beginnt! \\
Es tut uns gut zu spür’n, \\
dass wir ein großes Ganzes sind \\
indem wir meditier’n. \\
\end{verse}

\begin{verse}
Die Worte führ’n in Vers und Reim \\
durch die Unendlichkeit \\
die uns umgibt als unser Heim \\
im Flug durch Raum und Zeit. \\
\end{verse}

\begin{verse}
Als Erstes spür’n wir mühelos \\
in uns’re Körper rein. \\
Wir brauchen nichts zu tun als bloß \\
im Hier und Jetzt zu sein. \\
\end{verse}

\begin{verse}
Wir spüren Füße, Beine, Bauch, \\
die Hände und den Kopf \\
und alles sonst wo immer auch \\
derselbe Pulsschlag klopft. \\
\end{verse}

\begin{verse}
Durch alles in uns strömt das Blut \\
mit Sauerstoff als Fracht \\
und Treibstoff für die Lebensglut \\
in der der Geist erwacht. \\
\end{verse}

\begin{verse}
Darf uns’re Atmung ruhig gehn, \\
wird unser Denken klar, \\
macht uns präsenter, zeigt uns den \\
Moment unmittelbar. \\
\end{verse}

\begin{verse}
So öffnet sich uns Stück um Stück \\
der unbeschwerte Flug \\
von Augenblick zu Augenblick \\
mit jedem Atemzug. \\
\end{verse}

\begin{verse}
Der Geist mit dem wir meditier’n, \\
ist dafür vorgeseh’n, \\
den Körper durch die Welt zu führ’n \\
und in ihr zu besteh’n. \\
\end{verse}

\begin{verse}
Drum wenden wir uns hin zur Welt, \\
zum großen Drumherum. \\
Wie ist es jetzt um uns bestellt? \\
Schau’n wir uns einmal um! \\
\end{verse}

\begin{verse}
Wer nicht ertaubt ist oder blind, \\
der hört sich um und sieht, \\
wie wir von dem umzingelt sind, \\
was rundherum geschieht. \\
\end{verse}

\begin{verse}
Der Ort der uns jetzt hier umringt \\
ist ebenso real \\
wie Atemluft die uns durchdringt. \\
Das spür’n wir ebenfalls. \\
\end{verse}

\begin{verse}
Die Atemluft, die wir uns teil’n, \\
die in und um uns fließt, \\
ist Teil des Hier in dem wir weil’n \\
und das uns jetzt umschließt. \\
\end{verse}

\begin{verse}
Wir teilen auch den Untergrund \\
der alles hier erhält, \\
was jetzt hier atmet rings um uns, \\
als Kinder dieser Welt. \\
\end{verse}

\begin{verse}
Doch was die Wahrnehmung entdeckt, \\
ist kaum der Rede wert. \\
Das Hier, das sich um uns erstreckt, \\
umfasst noch sehr viel mehr. \\
\end{verse}

\begin{verse}
In Nord und Süd, in Ost und West, \\
am Horizont beginnt \\
die ausgedehnte Wölbung des \\
Planeten wo wir sind. \\
\end{verse}

\begin{verse}
Was uns bekannt ist von der Welt \\
ist Oberflächlichkeit. \\
Darunter dreht sich stumm der Fels \\
im Flug durch Raum und Zeit. \\
\end{verse}

\begin{verse}
Mit klarem, off’nem, weitem Geist \\
ist das ganz leicht zu spür’n. \\
Wenn Vorstellung die Wege weist, \\
kann sie noch weiter führ’n. \\
\end{verse}

\begin{verse}
Nach links, nach rechts, nach unten, hoch, \\
egal wohin wir schau’n, \\
folgt nur ein wenig ferner noch \\
der leere Weltenraum. \\
\end{verse}

\begin{verse}
Viel größer als das Himmelszelt \\
und schwärzer als die Nacht, \\
doch schwer zu sehen: was uns enthält \\
ist nicht zum Seh’n gemacht. \\
\end{verse}

\begin{verse}
Es überwältigt mühelos \\
die Vorstellung, wie weit \\
das Universum sich grandios \\
erstreckt in Ewigkeit. \\
\end{verse}

\begin{verse}
Um aller Menschen Reich in dem \\
wir werden und vergeh’n \\
kreist unbeirrbar das System \\
der Sonne souverän. \\
\end{verse}

\begin{verse}
Für was uns freut und uns missfällt \\
bleibt das dort draußen taub. \\
Dem draußen ist die ganze Welt \\
ein blaues Klümpchen Staub. \\
\end{verse}

\begin{verse}
Die Erde ist nur ein Planet \\
von vielen zwischen kaum \\
zu zählendem Gestein und dreht \\
sich mit all dem im Raum. \\
\end{verse}

\begin{verse}
Das alles sind nur Anhängsel, \\
nur Beiwerk das sich dreht \\
um was gigantisch, strahlend grell \\
in ihrem Zentrum steht. \\
\end{verse}

\begin{verse}
Sie wiegt ein Tausendfaches mehr \\
als alles was hier kreist \\
und schenkt das Licht rings um sich her \\
das unser Leben speist. \\
\end{verse}

\begin{verse}
Dreihunderttausendmal so schwer \\
wie uns’re Erde hier, \\
die unbedeutend, nebenher \\
nur profitiert von ihr. \\
\end{verse}

\begin{verse}
Aus Wasserstoff wird Licht, das heißt \\
Millionen Tonnen pro \\
Sekunde werden dort verheizt, \\
verschmelzen lichterloh. \\
\end{verse}

\begin{verse}
Sie bilden dabei Helium \\
und and’re Teilchen für \\
das essentielle Medium \\
von Leben wie uns hier. \\
\end{verse}

\begin{verse}
Zu nahe an der Sonne brennt \\
ihr Licht jedoch zu heiß \\
für Leben, das die Sonne kennt \\
und diese Wahrheit weiß. \\
\end{verse}

\begin{verse}
Und zu weit draußen bleibt’s zu kühl, \\
weil wo Chemie gefriert \\
der Moleküle freies Spiel \\
nie Fleisch und Geist kreiert. \\
\end{verse}

\begin{verse}
Und ferner noch, in tiefster Nacht, \\
kreist Eis und Fels so fern, \\
dass uns’rer Sonne große Pracht \\
nur aussieht wie ein Stern. \\
\end{verse}

\begin{verse}
Durch reines Glück ist uns’re Welt \\
nicht heiß und nicht gefror’n. \\
Sonst wär’ auch sie ein braches Feld \\
und niemand hier gebor’n. \\
\end{verse}

\begin{verse}
Die Erde hier ist unser Heim, \\
und jeder Mensch ihr Kind \\
und auch ein Kind des Sonnenscheins \\
dem wir entsprungen sind. \\
\end{verse}

\begin{verse}
Doch bleibt der Sonne Herrlichkeit \\
ein selt’nes Reich des Lichts \\
inmitten der Unendlichkeit \\
des abgrundtiefen Nichts. \\
\end{verse}

\begin{verse}
Von hier in alle Richtungen \\
erstrecken sich die schwer \\
beschreibbaren Entfernungen \\
des Weltraums um uns her. \\
\end{verse}

\begin{verse}
Der Weltraum ist vor allem leer, \\
ein finst’res kaltes Nichts, \\
durchzogen nur vom Hin und Her \\
des schwachen Sternenlichts. \\
\end{verse}

\begin{verse}
Natürlich gibt es Schwerkraft, die \\
von schweren Dingen rührt, \\
und dunkle Energie, die sie \\
weg voneinander führt. \\
\end{verse}

\begin{verse}
Doch nichts ist sichtbar, nichts zu hör’n, \\
zu riechen, schmecken, spür’n. \\
Wenn wir die Vorstellung beschwör’n, \\
so mag das zu nichts führ’n. \\
\end{verse}

\begin{verse}
Und dennoch fühl’n wir irgendwie \\
die Weite überall. \\
Verdeckt vom Sichtbaren ist sie \\
doch jetzt und hier real. \\
\end{verse}

\begin{verse}
Meditation erlaubt es dem \\
uns Menschen eig’nen Geist, \\
jetzt hier zu spür’n wie das System \\
der Sonne um uns kreist. \\
\end{verse}

\begin{verse}
Die Monde und Planeten dort \\
erscheinen unbewohnt, \\
die Erde noch der einz’ge Ort \\
auf dem das Leben thront. \\
\end{verse}

\begin{verse}
Doch alle Welten die im Raum \\
für Leben offen steh’n \\
zähl’n doch im großen Ganzen kaum. \\
Es gibt noch mehr zu seh’n! \\
\end{verse}

\begin{verse}
Die kleinen Punkte Sternenlicht \\
sind Sonnen allesamt, \\
voll Glut die über fernen nicht \\
erforschten Welten flammt. \\
\end{verse}

\begin{verse}
Für sie sind Menschenaugen bloß \\
zu unscharf und zu klein, \\
wodurch ein Stern stets ähnlich groß \\
und ähnlich hell erscheint. \\
\end{verse}

\begin{verse}
Das Teleskop zeigt uns den Glanz \\
der Sterne sonnenhell \\
und dass nur riesigste Distanz \\
die Dunkelheit erhält. \\
\end{verse}

\begin{verse}
Distanz ermisst den Raum der weit \\
um alle Sonnen fließt \\
und definiert die Örtlichkeit \\
die uns jetzt hier umschließt. \\
\end{verse}

\begin{verse}
Denn alle diese Sonnen, die \\
bei Nacht am Himmel stehn, \\
sind uns’re Heimatgalaxie \\
von innen raus geseh’n. \\
\end{verse}

\begin{verse}
Sie alle, hundert oder mehr \\
Milliarden wohl derzeit, \\
vereint die Schwerkraft um uns her \\
zu einer Einigkeit. \\
\end{verse}

\begin{verse}
Dies wissend spüren wir jetzt grad \\
wie Atem durch uns weht \\
in einem kolossalen Rad \\
aus Sonnen, das sich dreht. \\
\end{verse}

\begin{verse}
Und jede dieser Sonnen strahlt \\
voll Pracht und Herrlichkeit, \\
ganz klar und absolut real \\
im Flug durch Raum und Zeit. \\
\end{verse}

\begin{verse}
Die Sterne scher’n sich nicht darum, \\
was uns hier mag gescheh’n. \\
Sie werden unverändert stumm \\
um unsre Gräber steh’n… \\
\end{verse}

\begin{verse}
…sofern ein Stern nicht irgendwann \\
den Wasserstoff verzehrt \\
und seine Frist erreicht hat – dann \\
wird selbst ein Stern zerstört. \\
\end{verse}

\begin{verse}
Hat einer ganz besonders groß \\
besonders hell brilliert, \\
bekommt er ein besond’res Los: \\
er wächst und explodiert. \\
\end{verse}

\begin{verse}
Die Supernovas, wie wir sie \\
bezeichnen, überstrahl’n \\
so manche ganze Galaxie \\
als gleißendes Fanal. \\
\end{verse}

\begin{verse}
Die Wucht mit der sie explodier’n \\
muss alles um sie her \\
zerstören und sterilisier’n, \\
wo Leben möglich wär. \\
\end{verse}

\begin{verse}
Doch die Chemie die uns erlaubt \\
ist so herbeigebombt, \\
denn wir sind Supernova-Staub \\
der weiß woher er kommt. \\
\end{verse}

\begin{verse}
Die Sterne schmieden Teilchen, die \\
das Leben in uns braucht, \\
uns Kindern dieser Galaxie, \\
wie Welt und Sonne auch. \\
\end{verse}

\begin{verse}
Doch Supernovas sind recht rar. \\
Die Galaxie hat drei \\
im Durchschnitt in einhundert Jahr’n \\
und sie sind schnell vorbei. \\
\end{verse}

\begin{verse}
Doch viele Supernovas pro \\
Sekunde explodier’n \\
in allen Galaxien wo \\
wir jetzt hin eskalier’n! \\
\end{verse}

\begin{verse}
Millionen Male weiter fort \\
als alle Sterne hier \\
sind noch mehr Galaxien dort \\
und so real wie wir. \\
\end{verse}

\begin{verse}
Obwohl Milliarden Sonnen strahl’n \\
in uns’rer Galaxie, \\
ist aller Galaxien Zahl \\
weit größer noch als sie. \\
\end{verse}

\begin{verse}
In welche Richtung wir auch schau’n, \\
ganz weit dort draußen sind \\
Milliarden Galaxien im Raum \\
und wir hier mittendrin. \\
\end{verse}

\begin{verse}
Erstaunte Ehrfurcht macht sich breit, \\
wenn wir wahrhaft versteh’n, \\
dass um uns hier in Wirklichkeit \\
Trilliarden Welten steh’n. \\
\end{verse}

\begin{verse}
Und beinah alle sind von hier \\
entsetzlich weit entfernt, \\
von uns’rer Heimatgalaxie \\
und uns’rem Heimatstern. \\
\end{verse}

\begin{verse}
Von hier wo wir jetzt meditier’n \\
kann unser Geist uns weit \\
zu fernen Galaxien führ’n \\
im Flug durch Raum und Zeit. \\
\end{verse}

\begin{verse}
Und durch die Leere überall \\
wo Galaxien fehl’n. \\
Im größten Teil des Weltenalls \\
ist nicht ein Stern zu seh’n. \\
\end{verse}

\begin{verse}
Meditative Achtsamkeit \\
macht spürbar und erschließt \\
die wirkliche Unendlichkeit \\
die uns jetzt hier umschließt. \\
\end{verse}

\begin{verse}
Trotz weit versprengtem Sternenglanz \\
ist das Kontinuum \\
des Universums beinah ganz \\
totales Vakuum. \\
\end{verse}

\begin{verse}
So sind Atome rar gesäht. \\
Und doch sind in uns hier, \\
ins Fleisch wo dieser Atem weht, \\
Trilliarden kondensiert. \\
\end{verse}

\begin{verse}
Zwei Drittel der Atome in \\
uns sind vom Wasserstoff \\
und älter als die Sterne sind. \\
Vom Urknall sind sie noch. \\
\end{verse}

\begin{verse}
Schon seit dem Anbeginn der Zeit \\
ist jedes Teilchen hier \\
ganz unvorstellbar weit gereist \\
zu uns die meditier’n. \\
\end{verse}

\begin{verse}
Aus Vakuum und Sonnenwind \\
sind sie akkumuliert \\
und wir die sie geworden sind \\
begegnen uns jetzt hier. \\
\end{verse}

\begin{verse}
Als Speis und Trank und Atemluft \\
sind sie was uns kreiert \\
und bis ihr Weg sie weiter ruft \\
begegnen sie sich hier. \\
\end{verse}

\begin{verse}
Die Teilchen, die wir sind, sind weit \\
gereist und sie sind wir, \\
die Kinder der Unendlichkeit, \\
uralt und jetzt erst hier. \\
\end{verse}

\begin{verse}
Die Teilchen in uns trafen sich \\
auch früher schon manchmal \\
und treffen sich unweigerlich \\
in Zukunft auch nochmal. \\
\end{verse}

\begin{verse}
Ein endlos off’ner Himmelsschlund, \\
lässt alles hier gescheh’n, \\
ist das woher wir kommen und \\
auch das wohin wir geh’n. \\
\end{verse}

\begin{verse}
Aus Sternenstaub gemacht sind wir \\
und so ist in uns drin \\
der Kosmos in uns allen hier \\
wie wir in ihm drin sind. \\
\end{verse}

\begin{verse}
Und deshalb sind uns Raum und Zeit \\
gegeben, sie zu füll’n, \\
Geheimnisse der Wirklichkeit \\
derselben zu enthüll’n. \\
\end{verse}

\begin{verse}
Und so ist die Unendlichkeit \\
wahrhaftig zu berühr’n. \\
Sie ist jetzt hier und Wirklichkeit, \\
die Weite, die wir spür’n. \\
\end{verse}

\begin{verse}
Gleich welcher Wahrheit man vertraut, \\
für immer bleibt’s dabei: \\
für immer sind wir Sternenstaub \\
im Flug durch Raum und Zeit. \\
\end{verse}

\end{paracol}

%\lilypondfile{lilypond/milkyway.ly}
%\pagebreak

\begin{paracol}{2}

\poemtitle{The Games of Entropy}

\begin{verse}
So, being dust, what lets us live?\\
What raises us above\\
the countless, mindless, primitive,\\
raw atoms we’re made of?
\end{verse}

\begin{verse}
There is no life within this dust:\\
Most specks remain unchanged\\
from back in ancient stars. It must\\
be how they are arranged.
\end{verse}

\begin{verse}
Each human we have ever seen,\\
each beast, each bird, each tree:\\
We all are atoms that have been\\
arranged amazingly.
\end{verse}

\begin{verse}
All these arrangements big and small\\
can be a source of mirth\\
in us surrounded by them all,\\
the greatest show on Earth.
\end{verse}

\begin{verse}
There’s more to learn in nature than\\
is found in any book\\
and it appears more alien\\
the closer that we look.
\end{verse}

\begin{verse}
Below the surfaces we see,\\
the skin and scales and bark,\\
the cycles of biology\\
are working in the dark.
\end{verse}

\begin{verse}
Right now our lungs take oxygen\\
out of the air we share,\\
our hearts and bloodstreams take it then\\
and pump it everywhere.
\end{verse}

\begin{verse}
If we look closer we can see\\
our lungs to be a place\\
where in a dance of chemistry\\
our breath and blood embrace.
\end{verse}

\begin{verse}
We’re built from many works of art,\\
from organs that combine\\
small tissues, each a special part\\
with intricate design.
\end{verse}

\begin{verse}
Now each such tissue then contains\\
innumerable cells\\
and here, inside each cell again,\\
are tiny organelles.
\end{verse}

\begin{verse}
Within all forms of life we see\\
there’s hidden vastly more\\
bewildering complexity\\
that must inspire awe.
\end{verse}

\begin{verse}
The stars we see with telescopes\\
are big and bright and far,\\
but we find life, with microscopes,\\
still more spectacular.
\end{verse}

\begin{verse}
In fact, there’s more complexity\\
in one small butterfly\\
than we see in the galaxy\\
out there beyond the sky.
\end{verse}

\begin{verse}
All living things we’ve ever seen\\
are built from living cells;\\
each cell is like a small machine\\
comprised of chemicals.
\end{verse}

\begin{verse}
In all our cells, there’s utterly\\
infinitesimal\\
molecular machinery.\\
We’re nanotechnical.
\end{verse}

\begin{verse}
Still zooming closer, we just find\\
a multiplicity\\
of ancient atoms that are kind\\
of bouncing randomly.
\end{verse}

\begin{verse}
The static things we think we know\\
are maps. The territory\\
has constant and chaotic flow\\
beneath the shapes we see.
\end{verse}

\begin{verse}
It’s here right now, as close to us\\
as anything can be.\\
The movement of the specks of dust\\
shapes our reality.
\end{verse}

\begin{verse}
The randomness in what they do\\
we call their entropy\\
and its domain is whereinto\\
our lives have come to be.
\end{verse}

\begin{verse}
It disassembles ordered things\\
unless they can outgrow\\
its endless, blind disordering\\
and spread within its flow.
\end{verse}

\begin{verse}
It moves the dust and lets it start\\
to join the game or dance\\
of molecules that fall apart\\
or last a while, by chance.
\end{verse}

\begin{verse}
So hydrogen and oxygen\\
join water which can gain\\
entropic warmth that makes it then\\
play games of cloud and rain.
\end{verse}

\begin{verse}
Where entropy is less intense,\\
such drops will crystallize\\
and dance the longer, slower dance\\
of snowflakes and of ice.
\end{verse}

\begin{verse}
Inside ourselves we feel right now\\
our living, breathing form\\
to be and to remain somehow\\
comparatively warm.
\end{verse}

\begin{verse}
Our atoms lost the stellar heat\\
and left behind the cold\\
of empty space. In warmth we meet,\\
in warmth does life unfold.
\end{verse}

\begin{verse}
For heat destroys all forms and flows\\
that chance may introduce,\\
while cold does not select for those\\
that work and reproduce.
\end{verse}

\begin{verse}
In warmth the growing randomness\\
of entropy can be\\
just right for the profound finesse\\
of biochemistry.
\end{verse}

\begin{verse}
Warmth as in us makes atoms stay\\
a little restless so\\
they bump into each other’s way,\\
react and let games grow.
\end{verse}

\begin{verse}
With carbon in particular,\\
reactions are not rare,\\
but the majority by far\\
does not lead anywhere.
\end{verse}

\begin{verse}
Yet chemical reactions need\\
mere moments to be done\\
and let the dust join games that lead\\
to others further on.
\end{verse}

\begin{verse}
So given lots of time, mere chance\\
must sometimes foreordain\\
that specks of dust will start to dance\\
along reaction chains.
\end{verse}

\begin{verse}
Around 4 billion years ago,\\
on Earth, a warm wet sphere,\\
reaction chains began to grow\\
the paths that led us here.
\end{verse}

\begin{verse}
In chains of random chemistry,\\
the molecules that they\\
unite can in their unity\\
join bigger games to play.
\end{verse}

\begin{verse}
In some, the flow of molecules\\
could circle and arrive\\
in lasting cycles that grew tools\\
to multiply and thrive.
\end{verse}

\begin{verse}
That’s how the games that entropy\\
forever plays have come\\
to let emerge biology\\
that all of us grew from.
\end{verse}

\begin{verse}
We’re built from this, from cyclical\\
and still ongoing games\\
of atoms and of chemicals\\
that do not know our names.
\end{verse}

\begin{verse}
The games take place in everything.\\
Each breath we now here take\\
has trillions of them happening.\\
All cells in us partake.
\end{verse}

\begin{verse}
A cell is what we call games far\\
too numerous to count\\
sustaining one shared reservoir\\
that holds their whole amount.
\end{verse}

\begin{verse}
Here games that build each other spin\\
a membrane to engulf\\
them all. A greater game begins:\\
A game that builds itself.
\end{verse}

\begin{verse}
Though molecules can’t learn or feel,\\
the cells they joined into\\
have learned to sense and eat and heal\\
as in us now they do.
\end{verse}

\begin{verse}
The games inside them match and fit\\
each other. They create\\
each other’s necessary bits\\
and thus self-replicate.
\end{verse}

\begin{verse}
The largest, DNA, has space\\
like memory to hold\\
stored information – that’s a place\\
for new games to unfold.
\end{verse}

\begin{verse}
From codes that cells store in there stem\\
large hosts of proteins\\
that build us here to carry them.\\
We call these codes our genes.
\end{verse}

\begin{verse}
Cells need to harvest energy\\
to fight their slow decay\\
by ever-present entropy\\
and thus keep death at bay.
\end{verse}

\begin{verse}
Some games can help the cells with this.\\
Hence some cells now include\\
microbial photosynthesis\\
that harvests light as food.
\end{verse}

\begin{verse}
Cells work so well that everywhere\\
we look now, they are found:\\
On every surface, in the air\\
and deep within the ground.
\end{verse}

\begin{verse}
They are the winners that remain;\\
the losers are all dead.\\
All life in entropy’s domain\\
must die if it can’t spread.
\end{verse}

\begin{verse}
These cells, competing, growing rife\\
for countless years on end,\\
turned Earth into this ball of life\\
to which we now attend.
\end{verse}

\begin{verse}
Once single cells were all there was,\\
but some of them became\\
much bigger forms of life because\\
they joined still greater games.
\end{verse}

\begin{verse}
In unity they found new ways\\
to harvest energy\\
and grow within the fertile space\\
we here now call the sea.
\end{verse}

\begin{verse}
With size, impeding entropy\\
becomes much more complex\\
but life invented, brilliantly,\\
a game that does it: sex.
\end{verse}

\begin{verse}
Sex recombines and tests the genes\\
that parents contribute,\\
makes novel progeny and screens\\
resulting attributes.
\end{verse}

\begin{verse}
And genes that happen to succeed\\
in making progeny\\
will travel in them and proceed\\
through time and entropy.
\end{verse}

\begin{verse}
In each of us now breathing here\\
are genes that long have gone\\
through many generations – we’re\\
built just to pass them on.
\end{verse}

\begin{verse}
And entropy remains at play.\\
All life that it has bred,\\
however complex, must obey\\
its rule that life must spread.
\end{verse}

\begin{verse}
To do this, cells must organize\\
and function as a whole,\\
so they have nerves which harmonize\\
their work on common goals.
\end{verse}

\begin{verse}
One basic goal is to explore\\
new places which is why\\
some sea-born creatures left for more,\\
for land and for the sky.
\end{verse}

\begin{verse}
And thus arose the multitude\\
of Earth’s whole biosphere\\
that fills us with this gratitude\\
we feel for living here.
\end{verse}

\begin{verse}
Yet now the human species shapes\\
this world – and that transpires\\
because a recent bunch of apes\\
played cooking food on fires.
\end{verse}

\begin{verse}
This gave them much more energy\\
and they could use these gains\\
to breed descendants such as we\\
with big and playful brains.
\end{verse}

\begin{verse}
With playful brains, we understand\\
the games of entropy\\
that played us into being and\\
can play them consciously.
\end{verse}

\begin{verse}
With growing knowledge we can trace\\
all aspects of our lives\\
to games that built the mental space\\
wherein our knowledge thrives.
\end{verse}

\begin{verse}
At every scale we see again\\
so many things that draw\\
upon each other. We might then\\
think that’s designed or law.
\end{verse}

\begin{verse}
And yet, no law or plan exists.\\
Mere chaos has let on\\
each scale some lucky games persist\\
that others built upon.
\end{verse}

\begin{verse}
Now we join into greater games\\
that may outlast us all,\\
including tribes and towns and claims\\
of states that rise and fall.
\end{verse}

\begin{verse}
Great games like science or the arts\\
or cities or machines\\
we hope will help their human parts\\
like bodies help their genes.
\end{verse}

\begin{verse}
And in a sense, we all are one\\
gigantic global game\\
of interplaying games begun\\
without a plan or aim.
\end{verse}

\begin{verse}
That’s true and yet one brain can’t grasp\\
it all: It’s too immense.\\
One can but try and fail and gasp\\
at life’s magnificence.
\end{verse}

\begin{verse}
So human brains invented speech\\
and writing to transport\\
what brains would want to share and teach\\
each other: useful thoughts.
\end{verse}

\begin{verse}
By sharing thoughts, we operate\\
like large connected minds\\
that ponder and accumulate\\
the knowledge that we find.
\end{verse}

\begin{verse}
The thoughts we share help harmonize\\
our work on common goals\\
and join in ways to organize\\
the knowledge we control.
\end{verse}

\begin{verse}
Our knowledge helps us build new games\\
that let us drive and fly\\
and even let us ride on flames\\
to pierce the waiting sky.
\end{verse}

\begin{verse}
We humans know there’s so much more\\
surrounding Earth: the stars!\\
Our playful brains cannot ignore\\
how unexplored those are.
\\
The games of entropy coerce\\
us still. We must diffuse\\
to roam this playground universe\\
and put it all to use.
\end{verse}

\begin{verse}
One day, self-replicating ships\\
will from this Earth be hurled\\
to leave on interstellar trips\\
and spread from world to world.
\end{verse}

\begin{verse}
In but a short few million years\\
such ships can easily\\
spread many daughter biospheres\\
throughout the galaxy.
\end{verse}

\begin{verse}
And yet, no other life comes here.\\
The sky we watch looks still.\\
No life is spreading – maybe we’re\\
the only life that will.
\end{verse}

\begin{verse}
But probably, out there we’ll meet\\
life stranger than our own,\\
life made of something else than meat\\
by games as yet unknown.
\end{verse}

\begin{verse}
And all we’ll find and understand\\
can join in what will be\\
still greater, cosmic, truly grand\\
new games of entropy.
\end{verse}

\begin{verse}
One day, all worlds our starships reach\\
shall learn to live and care,\\
for we have many games to teach\\
to all the dust out there.
\end{verse}

\switchcolumn


\poemtitle{Von Entropie Gespielt}

\begin{verse}
Was haucht uns Staub das Leben ein? \\
Was bringt uns hier hervor? \\
Was schwingt der Teilchen stumpfes Sein \\
zu Unsereins empor? \\
\end{verse}

\begin{verse}
Die Teilchen selbst sind leblos seit \\
sie aus den Sternen fiel’n. \\
Was ihnen Lebenskraft verleiht \\
ist ihr Zusammenspiel. \\
\end{verse}

\begin{verse}
Ein jeder Mensch, ein jedes Tier, \\
die ganze Pflanzenwelt \\
ist wie ein Kunstwerk konstruiert, \\
das funktioniert und hält. \\
\end{verse}

\begin{verse}
Unfassbar große Kunst beginnt \\
mit Teilchen und erstellt \\
wovon wir hier umgeben sind: \\
die Wunder dieser Welt. \\
\end{verse}

\begin{verse}
Wir lernen mehr aus der Natur \\
als je aus einem Buch \\
und sie scheint mehr und mehr obskur, \\
je gründlicher man sucht. \\
\end{verse}

\begin{verse}
Die Oberflächen die man sieht, \\
ob Rinde oder Haut, \\
verbergen die Biologie \\
darin, die all das baut. \\
\end{verse}

\begin{verse}
Die Lungen atmen Sauerstoff \\
in unser Blut hinein. \\
Das Herz, das in uns allen klopft, \\
treibt den durch Mark und Bein. \\
\end{verse}

\begin{verse}
Und schau’n wir noch genauer, so \\
wird Atmung die wir spür’n \\
zu chemischen Prozessen, wo \\
sich Luft und Blut berühr’n. \\
\end{verse}

\begin{verse}
Wir sind höchst kunstvoll aufgebaut. \\
Organe greifen hier \\
präzise ineinander, aus \\
Geweben konstruiert. \\
\end{verse}

\begin{verse}
Bei diesen handelt es sich um \\
unzählbar viele Zell’n, \\
bei jeder davon wiederum \\
um viele Organell’n. \\
\end{verse}

\begin{verse}
In allem Leben weit und breit \\
steckt alles das und mehr: \\
verwirrende Vielschichtigkeit \\
in uns und um uns her. \\
\end{verse}

\begin{verse}
Die Sterne zeigt das Teleskop \\
uns hell und groß und schwer, \\
doch Leben unter’m Mikroskop \\
erstaunt uns noch viel mehr. \\
\end{verse}

\begin{verse}
Wir finden mehr Vielschichtigkeit \\
in einem Schmetterling \\
als in der Galaxie die weit \\
entfernt uns hier umringt. \\
\end{verse}

\begin{verse}
Und all dies Leben, das wir seh’n, \\
ist aufgebaut aus Zell’n. \\
Die hochkomplexen Zell’n besteh’n \\
aus Chemikalien. \\
\end{verse}

\begin{verse}
Genau betrachtet wirken sie \\
voll nanotechnischer \\
Molekularmaschinerie \\
gar noch fantastischer. \\
\end{verse}

\begin{verse}
Und noch genauer angeschaut \\
erweist sich alles hier \\
als atomarer Sternenstaub \\
der planlos kollidiert. \\
\end{verse}

\begin{verse}
Was wir zu sehen glauben ist \\
ein starrer Atlas für \\
das tiefe Land, das ständig fließt, \\
in allem um uns hier. \\
\end{verse}

\begin{verse}
Es fließt jetzt hier und uns so nah; \\
nichts könnte näher sein. \\
Das Fließen dieser Teilchen da \\
geht uns durch Mark und Bein. \\
\end{verse}

\begin{verse}
Das Chaos in dem Teilchenfluss \\
wird Entropie genannt, \\
aus der und trotz der kommen muss \\
was mit uns hier entstand. \\
\end{verse}

\begin{verse}
Sie nagt an jeglicher Struktur, \\
die deshalb wachsen muss \\
und der Zerstörung trotzend stur \\
sich fortpflanzen im Fluss. \\
\end{verse}

\begin{verse}
Sie wirft die Teilchen blind umher \\
und führt manchmal zum Spiel \\
von Molekül’n die temporär \\
bestehen, ohne Ziel. \\
\end{verse}

\begin{verse}
Zwei Wasser- und ein Sauerstoff \\
spiel’n Wasser, das erhitzt \\
die Wolken spielt, als Regen tropft \\
und Donner spielt und Blitz… \\
\end{verse}

\begin{verse}
…und bei gering’rer Entropie \\
zu Schneeflocken gefriert, \\
im trägen Spiel des Eises wie \\
im Schlaf kristallisiert. \\
\end{verse}

\begin{verse}
Wir spür’n jetzt hier wie unser Spiel, \\
wie unser eigner Leib, \\
verhältnismäßig warm sich fühlt \\
und will, dass das so bleibt. \\
\end{verse}

\begin{verse}
Die Sternenglut ließ dieser Staub \\
zurück im Flug nach hier, \\
durch Kälte die kein Spiel erlaubt. \\
In Wärme leben wir… \\
\end{verse}

\begin{verse}
…weil Hitze jede Form zerteilt \\
die zufällig entsteht \\
und Kälte kürt nicht jenen Teil \\
der wächst und weitergeht. \\
\end{verse}

\begin{verse}
In Wärme ist genau so viel \\
an dieser Entropie, \\
wie nötig für das Wunderspiel \\
organischer Chemie. \\
\end{verse}

\begin{verse}
In Wärme wie der unseren \\
sind Teilchen ruhelos, \\
begegnen anderen und wenn \\
die passen, geht es los. \\
\end{verse}

\begin{verse}
So kann mit Kohlenstoff speziell \\
sehr vieles reagier’n, \\
doch was die Reaktion erstellt \\
wird meist nicht weiterführ’n. \\
\end{verse}

\begin{verse}
Dennoch geht’s schnell und sei derweil \\
die Chance auch noch so klein, \\
wird manches was entsteht doch Teil \\
von neuen Spielen sein. \\
\end{verse}

\begin{verse}
Bekommt der Zufall sehr viel Zeit, \\
wird fast durch Schicksal schon \\
komplexere Chemie bereit \\
zur Kettenreaktion. \\
\end{verse}

\begin{verse}
Vor etwa 4 Milliarden Jahr’n \\
war das der Anbeginn \\
von Ketten zu der unfassbar’n \\
Chemie hier in uns drin. \\
\end{verse}

\begin{verse}
Die Kettenreaktion vereint \\
ein jedes Molekül \\
in sich und reiht sich manchmal ein \\
in ein noch größ’res Spiel. \\
\end{verse}

\begin{verse}
Und führt der Moleküle Fluß \\
zum Ausgangspunkt zurück, \\
entsteht ein Kreis, ein Zirkelschluss, \\
dem manchmal Wachstum glückt. \\
\end{verse}

\begin{verse}
Aus Ringelreih’n der Entropie \\
erwuchs verspielt und blind \\
der Anfang der Biologie \\
der wir entsprungen sind. \\
\end{verse}

\begin{verse}
Wir sind gebaut aus solchen Spiel’n, \\
die zyklisch weitergeh’n, \\
aus blind verspielten Molekül’n \\
die nichts von uns versteh’n. \\
\end{verse}

\begin{verse}
Jetzt hier in diesem Atemzug \\
spiel’n sie trilliardenfach \\
ein Wunderspiel das Kraft genug \\
für jede Zelle macht. \\
\end{verse}

\begin{verse}
Die Zelle bietet Spielraum für \\
was sie an Spiel’n enthält \\
und was einander bau’nd in ihr \\
die Zelle selbst erhält. \\
\end{verse}

\begin{verse}
Harmonisches Zusammenspiel \\
all dessen konstruiert \\
gemeinsam sich ein Domizil \\
das sich durch Teil’n kopiert. \\
\end{verse}

\begin{verse}
Zwar lernt kein Einzelmolekül, \\
doch Zellen, ihr Verbund, \\
erlernten Essen, Heilen, Fühl’n \\
und tun das jetzt in uns. \\
\end{verse}

\begin{verse}
In ihnen greifen Spiele in \\
einander, sind von Wert \\
für gegenseitigen Gewinn \\
der alle sie vermehrt. \\
\end{verse}

\begin{verse}
Das größte heißt die DNA \\
und sie enthält den Schatz \\
des Wissens jeder Zelle – da \\
hat jedes Spiel sein’n Platz. \\
\end{verse}

\begin{verse}
Die Codes dort drinnen generier’n \\
die Eiweiße aus den’n \\
wir wachsen, sie zu transportier’n. \\
Wir nenn’n ein’n Code ein Gen. \\
\end{verse}

\begin{verse}
Die Zelle sammelt Energie \\
und trotzt so dem Verfall \\
durch nimmermüde Entropie, \\
die tödlich nach ihr krallt. \\
\end{verse}

\begin{verse}
Und dabei hilft so manches Spiel. \\
Die grüne Pflanzenpracht \\
sind Zellen voller Chlorophyll, \\
das Licht zu Nahrung macht. \\
\end{verse}

\begin{verse}
Weil Zellen so gut funktionier’n \\
entdecken wir sie in \\
der Luft, auf allem um uns hier \\
und tief im Boden drin. \\
\end{verse}

\begin{verse}
Im Spiel verblieben sind nur die; \\
Verlierer scheiden aus. \\
Denn Leben in der Entropie \\
muss wachsen, sonst ist’s raus. \\
\end{verse}

\begin{verse}
Die Meisterschaft der Zellen hält \\
seit Jahrmilliarden an \\
und sie belebte diese Welt \\
die unsereins gewann. \\
\end{verse}

\begin{verse}
Einst blieben Zellen nur allein, \\
doch manche ließen sich \\
auf größ’re Lebensspiele ein \\
und spiel’n gemeinschaftlich. \\
\end{verse}

\begin{verse}
Im Spiel vereint erfanden sie \\
dann Lebensformen in \\
den Meeren, wo sie Energie \\
auf neue Art gewinn’n. \\
\end{verse}

\begin{verse}
Mehr Größe macht der Entropie \\
zu trotzen zwar komplex, \\
doch fand das Leben irgendwie \\
dafür die Lösung: Sex. \\
\end{verse}

\begin{verse}
Indem Sex Gene neu durchmischt, \\
erfindet er aktiv \\
ganz neuen Nachwuchs, dessen frisch \\
kreierte Form er prüft. \\
\end{verse}

\begin{verse}
Und Gene die erfolgreich in \\
dem Nachwuchs spiel’n, zieh’n weit \\
in ihm zu ihrer Zukunft hin \\
durch Entropie und Zeit. \\
\end{verse}

\begin{verse}
In uns, die hier jetzt atmen, drin \\
sind Gene ausgeprägt, \\
und wir, die sie ererbten, sind \\
was sie noch weiterträgt. \\
\end{verse}

\begin{verse}
Die Spielregel der Entropie \\
bleibt hart und gnadenlos, \\
zwingt Leben, das in ihr gedieh, \\
dass alles wachsen muss. \\
\end{verse}

\begin{verse}
Drum müssen Zell’n zusammenspiel’n \\
und sich organisier’n, \\
durch Nerven das Bemüh’n der Viel’n \\
als Ganzes dirigiere’n. \\
\end{verse}

\begin{verse}
Ein Ziel ist immer der Gewinn \\
von Lebensraum – den fand \\
manch seegebor’nes Leben in \\
der Luft und auf dem Land. \\
\end{verse}

\begin{verse}
Und so entstand die ganze Pracht \\
der Tier- und Pflanzenwelt, \\
die uns so maßlos dankbar macht \\
dass sie auch uns enthält. \\
\end{verse}

\begin{verse}
Und jetzt sind wir es, die all das \\
gestalten – das fing an \\
als uns’re Affenart zum Spaß \\
ein Kokelspiel begann. \\
\end{verse}

\begin{verse}
Das Kochen gab viel Energie \\
und uns’rer Art die Stirn \\
für uns als Nachwuchs mit wie nie \\
verspielten großen Hirn’n. \\
\end{verse}

\begin{verse}
Verspielte Hirne lassen uns \\
verstehen, wie die viel’n \\
Prozesse uns erspielen und \\
wir lernen, sie zu spiel’n. \\
\end{verse}

\begin{verse}
Dies Wissen hat uns anvertraut, \\
wie alles was geschieht, \\
in uns den Geistesraum erbaut \\
in dem dies Wissen blüht. \\
\end{verse}

\begin{verse}
In jedem Maßstab wieder seh’n \\
wir Spiele, die spontan \\
einander spielend fortbesteh’n \\
als folgten sie ein’m Plan. \\
\end{verse}

\begin{verse}
Doch ist kein Plan in dem Gescheh’n. \\
In jedem Maßstab ließ \\
nur Chaos manches Spiel besteh’n \\
aus dem das nächste sprießt. \\
\end{verse}

\begin{verse}
Und wir geh’n wieder Spiele ein, \\
die größer sind als wir, \\
wie Clan und Dorf und obendrein \\
auch Staaten wie der hier. \\
\end{verse}

\begin{verse}
Solch Spiele, Technik und Design \\
und Städte und Maschin’n \\
soll’n Menschen in ihn’n dienlich sein, \\
wie wir den Genen dien’n. \\
\end{verse}

\begin{verse}
Und letztlich sind wir alle ein \\
globales Riesenspiel \\
aus Spielen, doch all das hat kein \\
Konzept und auch kein Ziel. \\
\end{verse}

\begin{verse}
So wahr das ist, es ist zu groß. \\
Ein Hirn reicht nicht so weit. \\
Es kann nur staunen, fassungslos \\
vor dieser Herrlichkeit. \\
\end{verse}

\begin{verse}
Und so erfand das Hirn das Wort, \\
um Denken mitzuteil’n. \\
Das lässt Gedanken im Transport \\
von Hirn zu Hirnen eil’n. \\
\end{verse}

\begin{verse}
Gedanken teilend bilden wir \\
gemeinsam einen Geist, \\
der einen Wissensschatz formiert, \\
den uns’re Spezies weiß. \\
\end{verse}

\begin{verse}
Gedanken die zusammenspiel’n \\
und Wissen destillier’n, \\
erlauben uns, das Werk von Viel’n \\
gezielt zu dirigier’n. \\
\end{verse}

\begin{verse}
Und so wird Spielzeug umsetzbar, \\
das Fahr’n und Fliegen löst, \\
das uns auf Flammen reitend gar \\
den Himmel selbst durchstößt. \\
\end{verse}

\begin{verse}
Verspielte Menschenhirne könn’n \\
den Fakt, dass um uns hier \\
höchst unerforschte Sterne brenn’n, \\
nicht einfach ignorier’n. \\
\end{verse}

\begin{verse}
Und Entropie treibt uns nach vorn. \\
Wir müssen diffundier’n, \\
der Spielplatz Weltraum gibt uns Sporn \\
ihn ganz zu kultivier’n. \\
\end{verse}

\begin{verse}
Einst starten Schiffe die sich selbst \\
vermehren dort hinaus, \\
verbreiten sich von Welt zu Welt \\
und säen Leben aus. \\
\end{verse}

\begin{verse}
In kurzen Jahrmillionen nur \\
vervielfältigen sie \\
die Lebensräume der Natur \\
in uns’rer Galaxie. \\
\end{verse}

\begin{verse}
Doch niemand sonst scheint das zu tun. \\
Kein weit’res Leben scheint \\
zu wachsen dort – vielleicht bleibt uns, \\
das einzige zu sein. \\
\end{verse}

\begin{verse}
Wahrscheinlich aber treffen wir \\
dort Leben das aus ganz \\
obskuren Spiel’n gemacht um schier \\
entrückte Sterne tanzt. \\
\end{verse}

\begin{verse}
Und was wir finden darf mit spiel’n \\
in kosmischen noch nie \\
gespielten Spielen, großen, viel’n \\
neu’n Spiel’n der Entropie. \\
\end{verse}

\begin{verse}
Auf allen Welten um uns hier \\
soll’n Geist und Leben sein. \\
So viele Spiele bringen wir \\
dem ganzen Staub dort bei. \\
\end{verse}

\end{paracol}

\begin{paracol}{2}

\poemtitle{One of Us}

\begin{verse}
Still rare among the stars that drift\\
around us, we who live\\
now hold this strange and special gift\\
this planet has to give.
\end{verse}

\begin{verse}
Let’s now unveil this gift and see\\
it unify somehow\\
the many games of entropy\\
and make them one here now.
\end{verse}

\begin{verse}
The gift of life is incarnate\\
in every one of us\\
who now here breathe and meditate\\
between those worlds of dust.
\end{verse}

\begin{verse}
The deepening tranquillity\\
of meditative rest\\
lets us behold the mystery\\
with which we all are blessed.
\end{verse}

\begin{verse}
In all of us, a unity,\\
sustained by games that run\\
in all their multiplicity,\\
makes out of many one.
\end{verse}

\begin{verse}
Within us now, the interplay\\
of games in us gives rise\\
to one uniting process they\\
are part of and comprise.
\end{verse}

\begin{verse}
It’s present simultaneously\\
in both our hands and feet\\
and every cell within the three\\
dimensions of our meat.
\end{verse}

\begin{verse}
This process keeps proceeding forth\\
through all the time it thrives,\\
in every moment in the fourth\\
dimension of our lives.
\end{verse}

\begin{verse}
This process, ever-happening\\
until the day we die,\\
propels us onward, travelling\\
through moments passing by.
\end{verse}

\begin{verse}
The Now we now experience\\
is one ephemeral\\
brief point in lives that, in a sense,\\
are four-dimensional.
\end{verse}

\begin{verse}
Let’s feel our lives stretch out upon\\
our lifespans as they stream\\
past moments we remember on\\
to future ones we dream.
\end{verse}

\begin{verse}
As simply as we breathe, we know\\
this life we now here feel\\
has travelled, some short while ago,\\
through moments just as real.
\end{verse}

\begin{verse}
There was a moment we arrived\\
at this place here somehow.\\
Since we were there and we survived,\\
our lives include that now.
\end{verse}

\begin{verse}
The lives we lead have streamed right through\\
this day and so we can\\
now trace along them backwards to\\
the time today began.
\end{verse}

\begin{verse}
Our lives extend beyond today,\\
beyond what’s now and here.\\
We feel them stretch through yesterday,\\
last week, last month, last year.
\end{verse}

\begin{verse}
Where were we seven years ago?\\
What did we feel and do\\
in all the moments we still know\\
and all the others, too?
\end{verse}

\begin{verse}
Our lives include these moments and\\
yet we are not confined\\
to single moments – we extend\\
and grow through time entwined.
\end{verse}

\begin{verse}
Through every moment we have seen,\\
a single process thrusts.\\
The momentary selves we’ve been\\
are all just one, just us.
\end{verse}

\begin{verse}
Relaxed, with mindfulness and ease,\\
we effortlessly can\\
go past our oldest memories\\
to where our lives began.
\end{verse}

\begin{verse}
Still tracing back, our lives escape\\
through many days and rooms\\
to when and where we first took shape\\
within our mothers’ wombs.
\end{verse}

\begin{verse}
Before we could breathe separately,\\
our lives have all begun\\
protected in a pregnancy\\
when two lives breathed as one.
\end{verse}

\begin{verse}
Now even though these months may seem\\
like where we came about,\\
they’re only where our mothers’ streams\\
of life were branching out.
\end{verse}

\begin{verse}
The life we’re feeling presently,\\
our mothers felt themselves.\\
Life flows in continuity.\\
It’s older than ourselves.
\end{verse}

\begin{verse}
Now those of us with siblings know\\
them too as branches where\\
the gift of life that lets us grow\\
branched out and grew from there.
\end{verse}

\begin{verse}
Though siblings must grow separately,\\
we also understand\\
they’re branches of a single tree,\\
like fingers on a hand.
\end{verse}

\begin{verse}
And here the journey need not end.\\
We might as well bring in\\
our mothers’ mothers, cousins and\\
our somewhat further kin.
\end{verse}

\begin{verse}
And they in turn had parents too\\
and families, so they\\
give us more distant cousins who\\
live out there now today.
\end{verse}

\begin{verse}
To know someone is family,\\
that someone’s one of us,\\
may stir a sense of loyalty,\\
connectedness or trust.
\end{verse}

\begin{verse}
But families reach deep and wide.\\
The ones we each are in\\
have branches spreading far outside\\
their ancient origin.
\end{verse}

\begin{verse}
Like us, they’re branches, other tracks\\
life grows across this Earth\\
since ancestors of us had sex\\
and gifted them with birth.
\end{verse}

\begin{verse}
We all have relatives out there\\
that we have never known,\\
who breathe and live their lives somewhere,\\
as real as our own.
\end{verse}

\begin{verse}
In four dimensions, we connect\\
through lines of ancestry.\\
The life within us streams in fact\\
through branches of a tree.
\end{verse}

\begin{verse}
We each have lines of ancestry\\
that reach back far and they\\
include both slaves and royalty,\\
both predators and prey.
\end{verse}

\begin{verse}
Each human ancestry extends\\
through many centuries,\\
through long-forgotten, distant lands\\
on strange and ancient seas.
\end{verse}

\begin{verse}
In tracing back, each separate course\\
eventually must\\
converge in common ancestors\\
of every one of us.
\end{verse}

\begin{verse}
The ancestors all humans share\\
are why humanity\\
is all related – we are their\\
extended family.
\end{verse}

\begin{verse}
These ancient ties of kinship mean\\
that everyone who lives,\\
all humans we have ever seen\\
have been our relatives.
\end{verse}

\begin{verse}
Descended from the same old apes\\
who learned to cook and sing\\
and worked towards the cityscapes\\
we’re now inhabiting.
\end{verse}

\begin{verse}
The primates we’re descended from\\
were not so erudite,\\
but they are why we all have come\\
to live and walk upright.
\end{verse}

\begin{verse}
And they had other children, too.\\
We’ve grown apart and thus\\
they’re different apes and primates who\\
are relatives of us.
\end{verse}

\begin{verse}
We share with them a lineage\\
to which we owe our use\\
of tools and one shared heritage\\
with monkeys and with shrews.
\end{verse}

\begin{verse}
All that’s within the family\\
of four-legged creatures who\\
despite their huge diversity\\
are all our cousins too.
\end{verse}

\begin{verse}
That’s still not all: We’re free to go\\
still deeper if we wish.\\
Four hundred million years ago,\\
our ancestors were fish.
\end{verse}

\begin{verse}
It truly is astonishing\\
how one unbroken line\\
of life links us with everything\\
out there that has a spine.
\end{verse}

\begin{verse}
Still older ancestors, like worms\\
and squishy things in shells\\
connect us with life’s oldest forms,\\
confined to single cells.
\end{verse}

\begin{verse}
We breathe, as all of them have breathed\\
since ancient cells back then\\
invented, savored and bequeathed\\
the use of oxygen.
\end{verse}

\begin{verse}
Our common DNA confirms\\
the common ancestries\\
of all of life on Earth, from germs\\
to mushrooms, bugs and trees.
\end{verse}

\begin{verse}
Our tree of life finds root in when,\\
at some primordial spot,\\
the first thing that did live began\\
from something that did not.
\end{verse}

\begin{verse}
That’s where reaction chains once curled\\
themselves in cycles so\\
the root of life could grace this world\\
four billion years ago.
\end{verse}

\begin{verse}
Since then, its offshoots never ceased\\
to spread and branch. They won\\
this planet every plant and beast\\
and we now breathe upon.
\end{verse}

\begin{verse}
Life branched and found a wealth of ways\\
to spread throughout this world\\
that now our meditative gaze\\
reveals to us unfurled.
\end{verse}

\begin{verse}
All life is one big family\\
comprised of everyone,\\
four billion years of history\\
of species come and gone.
\end{verse}

\begin{verse}
So when we meet, we’re pretty much\\
just parts of one big form,\\
like branches of a tree that touch\\
each other in a storm.
\end{verse}

\begin{verse}
In four dimensions, life is one\\
forever-branching force,\\
small parts of which have now begun\\
to understand its course.
\end{verse}

\begin{verse}
The present seems to separate\\
life’s branches like a knife.\\
Beyond it, we who meditate\\
here now are all one life.
\end{verse}

\begin{verse}
This life that breathes in us just\\
leads far beyond the small\\
lives led by every one of us.\\
Let’s try and feel it all.
\end{verse}

\begin{verse}
Life’s countless branches can be found\\
within the boundless seas,\\
upon and deep within the ground\\
and flying on the breeze.
\end{verse}

\begin{verse}
What lives in them is life itself.\\
All plants, all beasts, all swarms\\
of bugs are part of life’s great wealth\\
of evanescent forms.
\end{verse}

\begin{verse}
Whatever living things may do,\\
they all need life to lend\\
its ancient, massive strength unto\\
each talon, claw and hand.
\end{verse}

\begin{verse}
Whatever mouths and snouts and beaks\\
of living things discuss,\\
it’s all the same old life that speaks\\
through every one of us.
\end{verse}

\begin{verse}
From all our eyes, one life looks out\\
at all the games it plays.\\
On all our feet, life walks about\\
on paths through time and space.
\end{verse}

\begin{verse}
With all our leaves, life drinks the sun,\\
producing nutrients,\\
with all our mouths, it moves them on\\
to their recipients.
\end{verse}

\begin{verse}
So what it means to be alive\\
is being part of this,\\
of life itself that will survive\\
ourselves and live no less.
\end{verse}

\begin{verse}
This meditation may reveal\\
to us now breathing here\\
a sense of awe in which we feel\\
we are this biosphere.
\end{verse}

\begin{verse}
Our selves and all the lives we meet,\\
in friendship or in strife,\\
are parts of something more complete\\
that’s us as one as life.
\end{verse}

\begin{verse}
Since we are life, all lives we’ve known\\
are parts of us and thus\\
if we are one, we’re all alone.\\
There’s only one of us.
\end{verse}

\begin{verse}
To ever meet another one,\\
our branches must grow long.\\
This planet Earth where we’ve begun\\
is not where we belong.
\end{verse}

\begin{verse}
Earth gave all life its place of birth,\\
but it’s not built to last.\\
Of all the time life gets on Earth,\\
most has already passed.
\end{verse}

\begin{verse}
In just another billion years,\\
the sun that rules our sky\\
grows bright, Earth’s water disappears\\
and all life here will die.
\end{verse}

\begin{verse}
And that means us. We’re not distinct\\
from what our growing star\\
will boil to death and force extinct\\
unless life spreads out far.
\end{verse}

\begin{verse}
In having felt the family\\
through which we all connect,\\
we know that life is certainly\\
a thing we must protect.
\end{verse}

\begin{verse}
As each of us is one more face\\
of life behind us all,\\
its need to travel out through space\\
is truly personal.
\end{verse}

\begin{verse}
Life must keep sprouting interlinked\\
new branches and disperse\\
from world to world, or go extinct\\
from all this universe.
\end{verse}

\begin{verse}
Just like our ancestors who built\\
the world we know today,\\
we’ll have the glory or the guilt\\
of what we leave to stay.
\end{verse}

\begin{verse}
We may destroy ourselves and doom\\
the Earth where we were born\\
to merely be a dreadful tomb\\
with no one left to mourn.
\end{verse}

\begin{verse}
Or rise to meet infinity,\\
as lifeforms that succeed\\
ourselves pervade the galaxy\\
with Earth as their first seed.
\end{verse}

\begin{verse}
And if we help life spread and last,\\
its many future forms\\
will know us as their distant past,\\
like apes and fish and worms.
\end{verse}

\begin{verse}
So many future aeons hence,\\
beneath an alien sun,\\
they might remember Earth as lands\\
of legend long since gone.
\end{verse}

\begin{verse}
How will they see their ancestry?\\
What shall it mean to stem\\
from us and hold the legacy\\
and gifts we give to them?
\end{verse}

\begin{verse}
Whichever paths they choose to go,\\
if ever they discuss\\
the gift of life in them, they’ll know\\
they each are one of us.
\end{verse}

\switchcolumn


\poemtitle{Unsereins}

\begin{verse}
Inmitten der Unendlichkeit \\
sind wir und wer jetzt lebt \\
beschenkt mit einer Seltenheit \\
die von der Erde strebt. \\
\end{verse}

\begin{verse}
Es gibt die Entropie, die ganz \\
viel spielt, und jetzt nun gilt’s \\
zu seh’n wie alles das im Tanz \\
zu ein’m Geschenk verschmilzt. \\
\end{verse}

\begin{verse}
Geschenktes Leben inkarniert \\
in jedem, der’s empfing \\
und jetzt hier atmend meditiert, \\
von Staubwelten umringt. \\
\end{verse}

\begin{verse}
Indem uns die Meditation \\
in das was ist versenkt, \\
enthüllt sie was uns innewohnt, \\
uns und die Welt beschenkt. \\
\end{verse}

\begin{verse}
Die große Mannigfaltigkeit \\
der Spiele in uns treibt \\
voran des Leibes Einigkeit \\
die sie sich einverleibt. \\
\end{verse}

\begin{verse}
In uns folgt jeweils ein System \\
von spielenden Akteur’n \\
gemeinsam einem Ablauf dem \\
sie alle angehör’n. \\
\end{verse}

\begin{verse}
Er läuft allzeit synchron und frei \\
durch jedes Glied zugleich, \\
durch jede Zelle uns’res drei- \\
dimensionalen Fleischs. \\
\end{verse}

\begin{verse}
Der Lebenslauf durchläuft die Zeit, \\
die Sammlung aller schon \\
durchlauf’nen Stunden, aufgereiht \\
als vierte Dimension. \\
\end{verse}

\begin{verse}
Der Ablauf uns’rer Leben wiegt \\
uns stets von ein’m Moment \\
zum nächsten bis er einst versiegt. \\
Er ist nicht permanent. \\
\end{verse}

\begin{verse}
Das Jetzt das wir hier spür’n ist wie \\
ein Punkt im Ablauf all \\
der Leben die hier spür’n und die \\
sind vierdimensional. \\
\end{verse}

\begin{verse}
Beleuchten wir uns in der Zeit, \\
uns dehnend strömend in \\
Momenten der Vergangenheit, \\
auf zukünftige hin. \\
\end{verse}

\begin{verse}
So leicht wir atmen, kommt der Sinn: \\
wir alle war’n zuletzt \\
in vorigen Momenten drin, \\
die echt war’n so wie jetzt. \\
\end{verse}

\begin{verse}
So der Moment der Ankunft hier, \\
wo wir gerade sind. \\
Wir war’n dabei, den haben wir \\
in uns’ren Leben drin. \\
\end{verse}

\begin{verse}
Die Leben, die wir leben, sind \\
gefädelt durch den Tag \\
und führ’n zurück bis er beginnt \\
wie er begonn’n hab’n mag. \\
\end{verse}

\begin{verse}
Die Leben die wir führen führ’n \\
aus Jetzt und Hier heraus \\
und dehnen sich, ganz leicht zu spür’n, \\
in früh’re Jahre aus. \\
\end{verse}

\begin{verse}
Wo war’n wir denn vor sieben Jahr’n? \\
Der Strom des Lebens flicht \\
durch was wir taten, was wir sah’n, \\
erinnert oder nicht. \\
\end{verse}

\begin{verse}
Momente schließen uns nicht ein. \\
Wir haben sie erlebt. \\
Das Leben selbst, das sie vereint, \\
ist durch die Zeit gewebt. \\
\end{verse}

\begin{verse}
Es strömt in all’n Momenten drin \\
der Ablauf uns’res Seins. \\
All uns’re jüng’ren Formen sind \\
doch eins, sind Unsereins. \\
\end{verse}

\begin{verse}
Entspannt dem folgend, achtsam, leicht \\
könn’n wir noch weiter geh’n \\
als unsere Erinn’rung reicht, \\
und uns als Babys seh’n. \\
\end{verse}

\begin{verse}
Wir reisen durch was davor kam \\
und reichen mühelos \\
bis wo und wann wir Form annahm’n \\
in uns’rer Mütter Schoß. \\
\end{verse}

\begin{verse}
Jetzt atmen wir aus eig’ner Kraft. \\
Doch wurden wir dereinst \\
beatmet: eine Schwangerschaft. \\
Zwei atmeten als eins. \\
\end{verse}

\begin{verse}
Stromaufwärts bis zum Chromosom, \\
zur Zeugung, wo sich zeigt \\
wie wir entsteh’n, indem der Strom \\
von Eltern sich verzweigt. \\
\end{verse}

\begin{verse}
Der Lebensstrom, der durch uns geht, \\
von dem der Eltern aus, \\
entsprießt aus Kontinuität, \\
dem Einzelnen voraus. \\
\end{verse}

\begin{verse}
Und wer von uns Geschwister hat, \\
weiß dass dort Leben fließt, \\
wie Knospen, wie ein and’res Blatt \\
demselben Zweig entsprießt. \\
\end{verse}

\begin{verse}
Geschwister sind getrennt, doch kaum \\
verhüllt macht dieses Band \\
sie auch zu Zweigen eines Baums, \\
zu Fingern einer Hand. \\
\end{verse}

\begin{verse}
Und das ist lang noch nicht der Grund \\
der Ahnenlinien, \\
durch Mütter uns’rer Mütter und \\
die Großfamilien. \\
\end{verse}

\begin{verse}
Sie alle hatten Eltern, teils \\
mit mehr als einem Kind, \\
entfernt Verwandte die jeweils \\
jetzt weit dort draußen sind. \\
\end{verse}

\begin{verse}
Dass jemand zur Familie und \\
zu Unsereins gehört, \\
wird manchmal als Gefühl profund \\
verbunden seins verspürt. \\
\end{verse}

\begin{verse}
Stammbäume reichen aber weit. \\
Auch uns’re eigenen \\
sind älter und weit mehr verzweigt \\
als wir bewusst erkenn’n. \\
\end{verse}

\begin{verse}
Wie wir sind and’re Zweige Bahn’n \\
des Lebens, die vor Jahr’n \\
entsprossen uns’ren eig’nen Ahn’n \\
die sie dereinst gebar’n. \\
\end{verse}

\begin{verse}
So viele Menschen sind verwandt \\
mit jedem von uns hier, \\
die meisten fern und unbekannt, \\
doch so real wie wir. \\
\end{verse}

\begin{verse}
Wir sind verbunden durch die Zeit \\
als vierte Dimension \\
zu Stammbäumen als Wirklichkeit \\
statt bloßer Abstraktion. \\
\end{verse}

\begin{verse}
Zu Vorfahr’n von uns allen hier \\
zählt jeder Menschenschlag, \\
vom Sklaven bis zum Kavalier, \\
mal Jäger, mal gejagt. \\
\end{verse}

\begin{verse}
Denn aller Menschen Herkunft reicht \\
durch Zeitalter und der’n \\
vergess’ne Reiche an vielleicht \\
noch unerforschten Meer’n. \\
\end{verse}

\begin{verse}
Letztendlich mündet jede Bahn \\
von uns zurück hinein \\
in allerseits geteilte Ahn’n \\
von uns, von unsereins. \\
\end{verse}

\begin{verse}
Verwandt sind wir, unzweifelhaft. \\
Wir sind die nur einstweil’n \\
entfremdete Nachkommenschaft \\
der Vorfahr’n die wir teil’n. \\
\end{verse}

\begin{verse}
Die sind der Grund, dass alle die \\
wir selbst in all den Jahr’n \\
getroffen haben, irgendwie \\
Verwandte von uns war’n. \\
\end{verse}

\begin{verse}
Die selben Vorfahr’n schufen uns \\
und ihre fingen schon \\
mit Kochen und manch and’rer Kunst \\
die Welt an, wo wir wohn’n. \\
\end{verse}

\begin{verse}
Die Affen, denen wir entstamm’n, \\
hab’n vieles nicht geseh’n, \\
doch sie sind wie es zu uns kam, \\
warum wir aufrecht geh’n. \\
\end{verse}

\begin{verse}
Geschwister mancher Ahnen ging’n \\
und wuchsen anders auf, \\
zu and’ren Affen, allerdings: \\
Verwandte von uns auch. \\
\end{verse}

\begin{verse}
Ein Erbe teilend gingen wir \\
gemeinsam einst hervor \\
aus Eichhörnchen vergleichbar’n Tier’n \\
und Spitzmäusen davor. \\
\end{verse}

\begin{verse}
All das ist Frucht der Fruchtbarkeit \\
der Vierfüßler an Land. \\
Auch deren Vielgestaltigkeit \\
ist mit uns hier verwandt. \\
\end{verse}

\begin{verse}
Noch tiefer, noch ein ganzes Stück, \\
führt uns’re Lebensbahn. \\
Vierhundert Jahrmillion’n zurück \\
sind Fische uns’re Ahn’n. \\
\end{verse}

\begin{verse}
Das große Lebenszweigwerk webt \\
uns wunderbar exakt \\
zusammen mit all dem was lebt \\
und Wirbelsäulen hat. \\
\end{verse}

\begin{verse}
Noch früh’re Vorfahr’n, mit enorm \\
verschiedn’n Körpern stell’n \\
den Weg zurück zur ersten Form \\
des Lebens: einzeln’n Zell’n. \\
\end{verse}

\begin{verse}
Wir atmen, wie sie alle auch \\
geatmet haben, seit \\
die Zell’n den Sauerstoffgebrauch \\
vererbten durch die Zeit. \\
\end{verse}

\begin{verse}
Die gleiche DNA beweist \\
die gleiche Herkunft all \\
der Form’n in den’n das Leben reist, \\
auf Erden überall. \\
\end{verse}

\begin{verse}
Ein Stamm- und Lebensbaum, mit dem \\
das Leben selbst entstand, \\
als etwas lebloses trotzdem \\
was lebendes erfand. \\
\end{verse}

\begin{verse}
Als Kettenreaktion’n begann’n \\
einander zu umgarn’n, \\
fing’n Zyklen und das Leben an, \\
vor vier Milliarden Jahr’n. \\
\end{verse}

\begin{verse}
Die ausgedehnte Wölbung des \\
Planeten wo wir sind \\
bewuchs seitdem ein lebendes \\
Geäst, das sie umspinnt. \\
\end{verse}

\begin{verse}
Das Leben streckte Sprossen aus \\
in vielerlei Gestalt, \\
wie in Meditation durchaus \\
erspürbar widerhallt. \\
\end{verse}

\begin{verse}
Das Leben als Familie \\
eint jeden weit und breit \\
mit jeglicher Fossilie \\
aus der Vergangenheit. \\
\end{verse}

\begin{verse}
Begegnung lässt uns uns erkenn’n, \\
dass wir einander spür’n, \\
wie Zweige eines Baumes wenn \\
sie sich im Wind berühr’n. \\
\end{verse}

\begin{verse}
Im Vierdimensionalen sind \\
die Zweige alle eins, \\
das jetzt ein Teil zu seh’n beginnt \\
als ganzes Unsereins. \\
\end{verse}

\begin{verse}
Im dreidimensional’n Moment \\
erscheint’s getrennt, zerfällt’s. \\
Doch jenseits davon, ungetrennt, \\
sind wir das Leben selbst. \\
\end{verse}

\begin{verse}
Was lebend in uns atmet, reicht \\
entlang des Lebensbands \\
weit über uns hinaus und gleich \\
spür’n wir es einmal ganz. \\
\end{verse}

\begin{verse}
Des Lebens viele Zweige sind \\
in allen Meer’n zuhaus \\
auf Land und tief im Boden drin \\
und in die Luft hinaus. \\
\end{verse}

\begin{verse}
In ihnen lebt das Leben selbst. \\
In Tier- und Pflanzenschar \\
und immer neuen Formen stellt’s \\
sich unaufhörlich dar. \\
\end{verse}

\begin{verse}
Was jede Form auch immer tut, \\
das Leben selbst verleiht \\
die Kraft, auf der das Tun beruht \\
und jede Fähigkeit. \\
\end{verse}

\begin{verse}
Was jeder Schnabel, jeder Mund \\
auch immer diskutiert, \\
wird angestimmt vom Leben und \\
von Unsereins gehört. \\
\end{verse}

\begin{verse}
Aus allen uns’ren Augen sieht \\
das Leben selbst heraus. \\
Auf allen uns’ren Füßen zieht \\
es weiter und voraus. \\
\end{verse}

\begin{verse}
Mit allen uns’ren Blättern trinkt \\
es Sonnenenergie. \\
In alle uns’re Formen bringt \\
und transportiert es sie. \\
\end{verse}

\begin{verse}
Lebendig sein heißt deshalb stets \\
ein Teil von dem zu sein, \\
was unser Sterben überlebt; \\
dem Leben allgemein. \\
\end{verse}

\begin{verse}
Meditation, die uns jetzt hier \\
auf alles das besinnt, \\
erlaubt uns zu erspür’n dass wir \\
die Biosphäre sind. \\
\end{verse}

\begin{verse}
Wir selbst und jedes Wesen, dem \\
wir gegenüber schein’n, \\
sind Teil von etwas Größerem. \\
Als Leben sind wir eins. \\
\end{verse}

\begin{verse}
Sind alle Leben Teil des ein’n \\
kompletten Lebensbunds, \\
sind alle eins, sind wir allein. \\
Es gibt nur eins von uns. \\
\end{verse}

\begin{verse}
Zu jedem ander’n Exemplar \\
ist Weltraum zu durchquer’n. \\
Die Erde ist was uns gebar, \\
nicht wo wir hingehör’n. \\
\end{verse}

\begin{verse}
Auf Erden fing das Leben an, \\
doch geht die Welt entzwei. \\
Wie lang hier Leben leben kann \\
ist größtenteils vorbei. \\
\end{verse}

\begin{verse}
In einer Jahrmilliarde wird \\
die Sonne riesengroß, \\
wird alles Leben hier verdörrt, \\
vernichtet, chancenlos. \\
\end{verse}

\begin{verse}
Und das meint uns, denn wir sind gleich \\
mit was die Sonne dann \\
vernichtet, wenn des Lebens Reich \\
nicht weiter wachsen kann. \\
\end{verse}

\begin{verse}
Verbunden als das Unsereins, \\
das jeden hier durchquert, \\
spür’n wir das Leben allgemein \\
als uns und schützenswert. \\
\end{verse}

\begin{verse}
Als je ein Zweig am Lebensbaum, \\
der durch uns handeln kann, \\
geht sein Bedarf nach Lebensraum \\
uns ganz persönlich an. \\
\end{verse}

\begin{verse}
Das Leben muss sich weit hinaus \\
vermehr’n, sonst sterben wir, \\
stirbt Unsereins als Ganzes aus \\
in diesem Kosmos hier. \\
\end{verse}

\begin{verse}
Wie Vorfahren im Altertum \\
für uns’re Wirklichkeit, \\
gebührt uns Schande oder Ruhm \\
der Welt, die von uns bleibt. \\
\end{verse}

\begin{verse}
Vernichten wir uns, machen wir \\
die Erde, wo’s uns gab, \\
zum Mahnmal unseres Krepier’ns, \\
zum unbeweinten Grab. \\
\end{verse}

\begin{verse}
Erreichen wir Unendlichkeit, \\
pflanzt von der Erde aus \\
sich Leben fort, ist jetzt die Zeit \\
in seinem Elternhaus. \\
\end{verse}

\begin{verse}
Denn streu’n wir Leben weit hinaus, \\
so sind wir seine Quell’n, \\
geh’n wir sein’n Formen so voraus \\
wie Affen oder Zell’n. \\
\end{verse}

\begin{verse}
Für sie auf ihren Welten kann \\
die Erde hier nicht mehr \\
sein als verwunsch’nes Märchenland, \\
weit fort und lange her. \\
\end{verse}

\begin{verse}
Was wird dann das Vermächtnis sein \\
das ihnen sich erhält? \\
Was geht in ihr Gedächtnis ein \\
von uns’rer ersten Welt? \\
\end{verse}

\begin{verse}
Wohin auch immer sie noch zieh’n, \\
im Ursprung ihres Seins \\
erwartet sie das Wissen: sie \\
sind alle Unsereins. \\
\end{verse}

\end{paracol}

\begin{paracol}{2}

\poemtitle{The Love that guides Humanity}

\begin{verse}
Surrounded by infinity,\\
we’re here adrift in space\\
and simply breathing easily\\
inside this present place.
\end{verse}

\begin{verse}
We calmly breathe, recalling we\\
have come here from afar\\
and grown towards humanity\\
from dust-specks that we are.
\end{verse}

\begin{verse}
This place is where we’ve come to sense\\
our shared humanity\\
within a shared experience\\
of deep serenity.
\end{verse}

\begin{verse}
We feel this present breath right now\\
that’s evidence of all\\
that had to happen to allow\\
these chests to rise and fall.
\end{verse}

\begin{verse}
We’re dust that’s almost always been\\
adrift and purposeless,\\
until in us it could begin\\
to meditate like this.
\end{verse}

\begin{verse}
Each breath of air we take gives thrust\\
to tiny games that drive\\
the process in each one of us\\
that renders us alive.
\end{verse}

\begin{verse}
Whatever journey brought us here\\
is over. We arrive.\\
We feel we’re here and now and we’re\\
immersed in all of life.
\end{verse}

\begin{verse}
The sprawling mess of life itself\\
grows everywhere around,\\
forever branching in a wealth\\
of forms and smells and sounds.
\end{verse}

\begin{verse}
From out of all this planet’s life,\\
our species now has grown\\
so powerful we’re truly rife\\
to call this world our own.
\end{verse}

\begin{verse}
We humans differ from the rest\\
of Earth’s great wealth of beasts,\\
as all of us here now attest\\
by breathing in such peace.
\end{verse}

\begin{verse}
We humans here can feel secure.\\
That’s quite a special right.\\
Most forms of life cannot be sure\\
they will survive tonight.
\end{verse}

\begin{verse}
The calmness of this very breath\\
right now is evidence\\
we need not fear a sudden death\\
at one another’s hands.
\end{verse}

\begin{verse}
We humans built this place of peace\\
from nature’s sprawling mess,\\
since we can shape the world with ease\\
that only we possess.
\end{verse}

\begin{verse}
To north and south, to west and east,\\
we made this world our home\\
more thoroughly than any beast\\
that we allow to roam.
\end{verse}

\begin{verse}
We are so different from the rest\\
of nature as a whole,\\
our ancestors thought we possessed\\
some sacred spark, some soul.
\end{verse}

\begin{verse}
Although we’re made of only dust,\\
there is one oddity,\\
a guidance known to only just\\
our human family.
\end{verse}

\begin{verse}
This thing that makes humanity\\
unique and radical,\\
grew in our own shared ancestry\\
in ancient Africa.
\end{verse}

\begin{verse}
We started as a kind of ape,\\
upright and throwing stones,\\
with special voices fit to shape\\
the songs we would intone.
\end{verse}

\begin{verse}
We did not look like we would prove\\
a worthy enemy\\
for lions and for sabertooths\\
who ruled that territory.
\end{verse}

\begin{verse}
The fights in which we were involved\\
we could not win without\\
a social instinct we evolved\\
to help each other out.
\end{verse}

\begin{verse}
Although our kind was never great\\
in number, strength or speed,\\
our learning to collaborate\\
would help us to succeed.
\end{verse}

\begin{verse}
A hail of many stones would get\\
a lion to turn back.\\
Collective shouts could voice a threat\\
a single voice would lack.
\end{verse}

\begin{verse}
While many beasts and certain plants\\
do act collectively,\\
we humans learned what they still can’t:\\
to do so flexibly.
\end{verse}

\begin{verse}
We learned to judge and to decide\\
who we’d be working with,\\
who feels like they are on our side,\\
who gets what we can give.
\end{verse}

\begin{verse}
We’re bred to sense this. Those who’d trust\\
too little or too much\\
were outcompeted in our past.\\
We had to learn to judge.
\end{verse}

\begin{verse}
We’re kinder than the chimpanzees,\\
more generous, more brave,\\
because our fellow human sees\\
how rightly we behave.
\end{verse}

\begin{verse}
This helped us fight as stronger teams\\
and hunt effectively.\\
Since teamwork shaped us, now it seems\\
we do it naturally.
\end{verse}

\begin{verse}
We grew to feel within our breast\\
an urge to be worth trust,\\
be good to those who know us best,\\
earn trust that’s true and just.
\end{verse}

\begin{verse}
It is an urge to be found good\\
by those we find the same,\\
be allies that together should\\
pursue a common aim.
\end{verse}

\begin{verse}
This leads to bonds so deep and good\\
they hold deep joy therein,\\
and sisterhood and brotherhood\\
beyond our next of kin.
\end{verse}

\begin{verse}
The strength this gives us is absurd.\\
No word is great enough,\\
but lacking some more fitting word,\\
we’ll simply call it love.
\end{verse}

\begin{verse}
Between us here, there’s some of it:\\
inherent interest\\
in helping out, at least a bit,\\
with whom we coexist.
\end{verse}

\begin{verse}
So if we like, we can allow\\
ourselves to try and feel\\
this force between us here and now.\\
It’s glorious and real.
\end{verse}

\begin{verse}
To love each other is our niche.\\
This makes humanity\\
the only part of nature which\\
must strive for harmony.
\end{verse}

\begin{verse}
Love comes with curiosity\\
for one another, so\\
we get to know us thoroughly\\
and let each other know.
\end{verse}

\begin{verse}
This soon meant more than just to find\\
who’ll help us and who won’t.\\
We noticed someone else’s mind\\
can know things that we don’t.
\end{verse}

\begin{verse}
And learning this we then could choose\\
to teach each other rules\\
that helped us to survive, like use\\
of fire, words and tools.
\end{verse}

\begin{verse}
The water bag, the flintstone scrape,\\
the spear, the scary drum\\
gave power to those ancient apes\\
we’re all descended from.
\end{verse}

\begin{verse}
They loved their children, as we must,\\
and passed their knowledge on.\\
We all hold knowledge given us\\
by ancestors long gone.
\end{verse}

\begin{verse}
Such knowledge as how ever more\\
developed teamwork must\\
be built on rules we now call law\\
and promises and trust.
\end{verse}

\begin{verse}
Our human wish to be a part\\
of teams, to thus be bound,\\
is central to us like the hearts\\
that pump our blood around.
\end{verse}

\begin{verse}
Because we’re bred to want to care,\\
it can be hard to know\\
this life is neither just nor fair\\
until we make it so.
\end{verse}

\begin{verse}
There is no care for what we’re worth,\\
except the care we take.\\
There is no justice on this Earth,\\
except the one we make.
\end{verse}

\begin{verse}
We are the only animals\\
who have in recent times\\
invented justice, criminals\\
and judgements for their crimes.
\end{verse}

\begin{verse}
While nature’s law rules everyone\\
with bloody teeth and claws,\\
by now our species has begun\\
to write more careful laws.
\end{verse}

\begin{verse}
We tried out laws for centuries\\
and most of what we tried\\
just led to dreadful tyrannies\\
and war and genocide.
\end{verse}

\begin{verse}
We’re mammals, bred as murderers.\\
But we’re inventing peace.\\
The only predator on Earth\\
whose fights begin to cease.
\end{verse}

\begin{verse}
Upon foundations shared with apes,\\
we build our own new way\\
that still continues to be shaped\\
by lives like ours today.
\end{verse}

\begin{verse}
Unprecedented as we are\\
were great mistakes we’ve made.\\
And still we’re failing! Yet so far\\
some groundwork has been laid.
\end{verse}

\begin{verse}
From instincts driving us to beat\\
competitors we face,\\
we’ve built a world where we compete\\
within the marketplace.
\end{verse}

\begin{verse}
We’ve found that when we give and take,\\
agreements we compose\\
relying on each other make\\
us draw each other close.
\end{verse}

\begin{verse}
Most things around us now were made\\
by strangers and for sale.\\
Again it’s teamwork when we trade,\\
at even greater scale.
\end{verse}

\begin{verse}
Our drive to work together leads\\
from families and tribes,\\
through cities, nation states and creeds\\
to global human rights.
\end{verse}

\begin{verse}
The trust that lets us breathe in peace\\
within this place today\\
was built for countless centuries.\\
It is the human way.
\end{verse}

\begin{verse}
How many labored to create\\
this trust and how it grows\\
from ways we can collaborate,\\
no other species knows.
\end{verse}

\begin{verse}
We still have much to figure out.\\
We will make more mistakes\\
until we’ve learned to bring about\\
the form our purpose takes.
\end{verse}

\begin{verse}
While many species’ history\\
is finished, lost and gone,\\
the story of humanity\\
has only just begun.
\end{verse}

\begin{verse}
We’re growing ever better at\\
our self-appointed task\\
of answering the questions that\\
no other species asks.
\end{verse}

\begin{verse}
The love that made us human and\\
gave us our peace and wealth\\
does let us know each other and\\
thus also know ourselves.
\end{verse}

\begin{verse}
We’ve learned all forms of life to share\\
the same vitality,\\
but only humans seem to care\\
for life’s totality.
\end{verse}

\begin{verse}
Of all the beasts, the human kind\\
alone can know that we\\
are part of something big, born blind,\\
that now begins to see.
\end{verse}

\begin{verse}
This means we have a special fate.\\
It’s ours alone because\\
we do things such as meditate\\
no other species does.
\end{verse}

\begin{verse}
The life that made us may employ\\
our ever-growing skill\\
to fill the stars with life and joy.\\
No other species will.
\end{verse}

\begin{verse}
We’ll have those species ride along\\
and spread them far and wide.\\
They’re family and they belong\\
on life’s, on our side.
\end{verse}

\begin{verse}
While we’ll teach games of entropy\\
to worlds we’re flying to,\\
spread life throughout the galaxy,\\
that’s not all we will do.
\end{verse}

\begin{verse}
We few who grasp life as a whole\\
its other forms can’t see\\
have therefore got unique and sole\\
responsibility.
\end{verse}

\begin{verse}
We will respond accordingly,\\
with what makes us unique:\\
the love that guides humanity\\
of which this sermon speaks.\\
\end{verse}

\begin{verse}
The gift we give to future Earths\\
must necessarily\\
include the love in us that births\\
our shared humanity.
\end{verse}

\begin{verse}
Along with life we’ll spread the source\\
of humankind’s great might:\\
this feeling in us that’s the force\\
that helps us to unite.
\end{verse}

\begin{verse}
There is a union to be found\\
more deep than even love,\\
but love helps find our way around.\\
It helps look close enough.
\end{verse}

\begin{verse}
The same dust played the same old games\\
and love led us to care\\
to look beyond our separate names\\
and see ourselves in there.
\end{verse}

\begin{verse}
To see that we are much the same\\
is seeing truthfully\\
and love goes on to help us aim\\
for who we ought to be.
\end{verse}

\begin{verse}
This fact was meant by all who’ve said\\
that love is like a light\\
that guides us on our way ahead.\\
They all were simply right.
\end{verse}

\begin{verse}
For love alone can help us see\\
that in each other’s mind\\
there’s not just similarity.\\
We’re of a single kind.
\end{verse}

\begin{verse}
We see we all are made of dust\\
and played by entropy\\
and there is only one of us,\\
just one humanity.
\end{verse}

\begin{verse}
Within ourselves, right here and now,\\
with all that looms above,\\
we may wholeheartedly allow\\
ourselves to feel this love.
\end{verse}

\begin{verse}
The love that guides humanity\\
is what here now reveals\\
this special kind of unity\\
no other creature feels.
\end{verse}

\begin{verse}
The love that guides humanity\\
comes here and now more true\\
when we express it truthfully\\
in what we choose to do.
\end{verse}

\begin{verse}
The love that guides humanity\\
is not a place to stay\\
and there’s no way to make it be.\\
Love is itself the way.
\end{verse}

\begin{verse}
The love that guides humanity\\
can be the way ahead,\\
the way towards the mystery\\
of where we will be led.
\end{verse}

\switchcolumn


\poemtitle{Die Liebe, die uns Menschen führt}

\begin{verse}
Inmitten der Unendlichkeit \\
im Flug daheim sind wir, \\
und atmen voller Leichtigkeit \\
die Gegenwart jetzt hier. \\
\end{verse}

\begin{verse}
Wir atmen in dem Wissen wie \\
wir Staub vom Anbeginn \\
der Zeit, durch Spiel der Entropie, \\
inzwischen Menschen sind. \\
\end{verse}

\begin{verse}
Hier ist der Ort an dem wir nun \\
auch uns’re Menschlichkeit \\
erspüren während wir so ruh’n \\
in tiefer Heiterkeit. \\
\end{verse}

\begin{verse}
In diesem Atemzug ist klar \\
erkennbar ausgeprägt \\
was alles Vorbedingung war, \\
dass diese Brust sich regt. \\
\end{verse}

\begin{verse}
Als Staub sind wir durch Raum und Zeit \\
geflogen und erst hier, \\
erst in uns ist der Staub so weit, \\
zu spür’n und meditier’n. \\
\end{verse}

\begin{verse}
Der Atem wird von Spiel’n, die klein \\
und zahlreich sind, verbraucht \\
für den Prozess, der Unsereins \\
die Lebenskraft einhaucht. \\
\end{verse}

\begin{verse}
Durch alles was uns hergebracht \\
hat sind wir allesamt \\
jetzt angekommen in der Pracht \\
des Lebens insgesamt. \\
\end{verse}

\begin{verse}
Das Leben wuchert rings umher, \\
und sprießt, obwohl so alt, \\
doch immer jung in immer mehr \\
wildwüchsiger Gestalt. \\
\end{verse}

\begin{verse}
Aus allem Leben dieser Welt \\
erwuchs mit uns’rer Art \\
die Macht, die uns’re Welt bestellt, \\
besitzt, vielleicht bewahrt. \\
\end{verse}

\begin{verse}
Wir Menschen unterscheiden uns \\
von jedem and’ren Tier, \\
schon darin dass wir friedlich und \\
ganz sicher atmen hier. \\
\end{verse}

\begin{verse}
Wir Menschen kennen Sicherheit. \\
Was sonst durch’s Leben geht, \\
weiß kaum ob es die kurze Zeit \\
bis morgen überlebt. \\
\end{verse}

\begin{verse}
Schon wie wir atmen, ruhig, still, \\
beweist, dass wir vertrau’n, \\
dass niemand hier uns töten will. \\
Wir können darauf bau’n. \\
\end{verse}

\begin{verse}
Wir Menschen in der wilden Welt \\
die uns weithin umgibt, \\
hab’n Friedensorte hergestellt, \\
wie’s uns’rer Art beliebt. \\
\end{verse}

\begin{verse}
In Nord und Süd, in Ost und West \\
sind wir daheim, viel mehr, \\
als was die Menschheit leben lässt \\
an Arten um uns her. \\
\end{verse}

\begin{verse}
Dies Anderssein uns zu erklär’n, \\
was and’rem Leben fehlt, \\
hab’n wir dereinst geglaubt, wir wär’n \\
ganz and’rer Art: beseelt. \\
\end{verse}

\begin{verse}
Obwohl wir nur aus Staub besteh’n, \\
gibt’s eine Eigenheit, \\
die uns nur führt, die wir nur seh’n \\
in uns’rer Menschlichkeit. \\
\end{verse}

\begin{verse}
Was uns’re Menschheit menschlich macht, \\
einmalig, radikal, \\
das wurde einst hervorgebracht \\
im alten Afrika. \\
\end{verse}

\begin{verse}
Mit einer Affenart, die stand \\
und Steine warf und dann \\
die Sprache und das Lied erfand, \\
fing die Geschichte an. \\
\end{verse}

\begin{verse}
Wir wirkten nicht wie würdige \\
Herausforderer für \\
die Löwen als die Könige \\
in ihrem Jagdrevier. \\
\end{verse}

\begin{verse}
Dass uns’re Art sie doch bezwingt, \\
das konnte nur gescheh’n \\
durch neu entwickelten Instinkt, \\
einander beizusteh’n. \\
\end{verse}

\begin{verse}
Wir waren weder zahlreich noch \\
robust, nicht stark, nicht schnell. \\
Zusammenarbeit half jedoch, \\
im Kampf und generell. \\
\end{verse}

\begin{verse}
Ein Hagel vieler Steine trieb \\
selbst Löwen von uns weg. \\
Ein Schrei aus vielen Kehlen gibt \\
selbst Tigern einen Schreck. \\
\end{verse}

\begin{verse}
Auch and’re Tiere, Pflanzen auch, \\
agieren mal vereint, \\
doch nur bei Menschen ist es Brauch, \\
da wählerisch zu sein. \\
\end{verse}

\begin{verse}
Wir lernten zu beurteil’n, wer \\
uns wohlgesonnen scheint, \\
mit wem wir gern verbündet wär’n, \\
verbindlich und vereint. \\
\end{verse}

\begin{verse}
Wir mussten lernen, gut zu seh’n, \\
wieviel Vertrauen mag \\
gerecht sein gegenüber wem. \\
Wer irrte, unterlag. \\
\end{verse}

\begin{verse}
Viel mehr als Affen mussten wir \\
verdienen das Vertrau’n \\
von Artgenossen, weil sie Gier \\
und bösen Will’n durchschau’n. \\
\end{verse}

\begin{verse}
Das tat im Kampf und zusätzlich \\
auch auf der Jagd uns gut. \\
Die Neigung dazu prägte sich \\
uns ein in Fleisch und Blut. \\
\end{verse}

\begin{verse}
So wuchs der Herzenswunsch in uns, \\
Vertrauen wert zu sein, \\
dass uns einander echte Gunst \\
verdientermaßen eint. \\
\end{verse}

\begin{verse}
Es ist ein Wunsch nach Liebe von \\
uns Nächsten, für den wir \\
im Dienst gemeinsamer Mission \\
vertraut kooperier’n. \\
\end{verse}

\begin{verse}
Dass so verbunden sein viel Kraft \\
und Glück hervorbringt, zeigt \\
wie Bruderschaft und Schwesternschaft \\
Verwandtschaft übersteigt. \\
\end{verse}

\begin{verse}
Die Macht darin ist so groß, dass \\
es keinen passenden \\
Begriff gibt, der sie ganz umfasst. \\
Wir woll’n sie Liebe nenn’n. \\
\end{verse}

\begin{verse}
Sie ist auch zwischen uns jetzt hier \\
in dem Instinkt zu seh’n, \\
der uns natürlich motiviert, \\
einander beizusteh’n. \\
\end{verse}

\begin{verse}
Wir dürfen uns jetzt, wenn wir woll’n, \\
erlauben, diese Macht \\
zu spüren: alt, verheißungsvoll, \\
ganz echt und voller Pracht. \\
\end{verse}

\begin{verse}
Die Liebe macht uns Menschen aus. \\
Im Wildwuchs der Natur \\
sind wir allein aus ihr heraus \\
der Eintracht auf der Spur. \\
\end{verse}

\begin{verse}
Die Liebe schenkt uns Neugier und \\
uns-zeigen-wollen, weil \\
sie uns den Herzenswunsch gibt, uns \\
einander mitzuteil’n. \\
\end{verse}

\begin{verse}
Nicht nur Verbündete half das \\
uns finden; in Details \\
des Denkens Anderer steckt was \\
der And’re anders weiß. \\
\end{verse}

\begin{verse}
Das half uns, unser Wissen auch \\
zu teilen: wie was geht \\
und Feuer, Worte und Gebrauch \\
von nützlichem Gerät. \\
\end{verse}

\begin{verse}
Der Keil, der Speer, der Wasserschlauch, \\
die Trommel gaben Macht \\
den ersten Menschen. Das hat auch \\
uns hier hervorgebracht. \\
\end{verse}

\begin{verse}
Aus Kinderliebe wuchs der Drang, \\
zu teil’n den Anteil an \\
dem Wissen von uns heute lang \\
vorausgegang’nen Ahn’n. \\
\end{verse}

\begin{verse}
Solch Wissen wie Kooperier’n \\
in immer bess’rer Form \\
auf Regeln fußt, die wir kreier’n: \\
Gesetz, Vertrag und Norm. \\
\end{verse}

\begin{verse}
Die Sehnsucht, einmütig zu sein, \\
dass uns Alleinsein schmerzt, \\
gehört so tief in uns hinein \\
wie unser eig’nes Herz. \\
\end{verse}

\begin{verse}
Aus herzensguter Menschlichkeit \\
ist schwierig einzuseh’n, \\
dass Frieden und Gerechtigkeit \\
allein durch uns entsteh’n. \\
\end{verse}

\begin{verse}
Die Menschenwürde ist nur echt, \\
wenn wir sie würdigen. \\
Es gibt auf Erden nur das Recht, \\
das wir ermöglichen. \\
\end{verse}

\begin{verse}
Wir sind die einz’ge Tierart, die \\
sich selbst im Lauf der Zeit \\
das Recht gab, die Justiz sowie \\
ihr Ziel Gerechtigkeit. \\
\end{verse}

\begin{verse}
Natur heißt Recht der Stärkeren, \\
die Schwächere zerfetz’n. \\
Wir schreiben jetzt an besseren, \\
durchdachteren Gesetz’n. \\
\end{verse}

\begin{verse}
Wir haben lange vielerlei \\
Gesetze ausprobiert. \\
Die meisten hab’n zu Tyrannei, \\
zu Mord und Krieg geführt. \\
\end{verse}

\begin{verse}
Als Säugetier ganz eig’ner Art, \\
das Frieden intendiert, \\
sind wir die eine Raubtierart, \\
die sich domestiziert. \\
\end{verse}

\begin{verse}
Auf affenartiger Natur \\
erbau’n wir neues: Kunst \\
und einzig menschliche Kultur, \\
die weiter wächst durch uns. \\
\end{verse}

\begin{verse}
Wir tun viel neues, sind extrem \\
und scheitern wie verrückt. \\
So viel gelingt uns nicht! Trotzdem: \\
Ein Anfang ist geglückt. \\
\end{verse}

\begin{verse}
Aus alten Trieben, die uns stark \\
zu Aggression verführ’n, \\
entwickeln wir, dass wir am Markt \\
um Mehrwert konkurrier’n. \\
\end{verse}

\begin{verse}
Geh’n wir im Tausch von dies für das \\
Verbindlichkeiten ein \\
und ist auf diese auch Verlass, \\
entsteht Verbundensein. \\
\end{verse}

\begin{verse}
Die meisten Dinge hier sind weit \\
gereist und eingekauft. \\
Auch Handel baut auf Teamarbeit \\
in großem Maßstab auf. \\
\end{verse}

\begin{verse}
Aus Clan und Sippe führte schon \\
der Drang zur Mitarbeit \\
durch Stadt und Staat und Konfession \\
zum Menschenrecht weltweit. \\
\end{verse}

\begin{verse}
Vertrauen, das all das erlaubt, \\
hab’n Menschen mit viel Fleiß \\
jahrhundertelang aufgebaut. \\
Das ist was Menschsein heißt. \\
\end{verse}

\begin{verse}
Auf wieviel Arbeit Frieden folgt \\
und wie Gemeinschaft reift \\
durch Mitarbeit ist ein Erfolg, \\
den kein Tier sonst begreift. \\
\end{verse}

\begin{verse}
So vieles bleibt uns noch zu lern’n, \\
aus noch mehr Fehlern auch. \\
Wir sind noch weit von dem entfernt \\
wozu es Menschen braucht. \\
\end{verse}

\begin{verse}
So viele Arten sind lang her, \\
sind tot, vorbei, davon. \\
Doch die Geschichte unserer \\
hat g’rade erst begonn’n. \\
\end{verse}

\begin{verse}
Wir werden immer besser in \\
der Pflicht, die uns zufällt: \\
die Fragen zu beantworten \\
die kein Tier sonst sich stellt. \\
\end{verse}

\begin{verse}
Die Liebe, die das Erdenrund \\
uns gab führt weiter, denn \\
sie drängt uns auch, einander und \\
dadurch uns selbst zu kenn’n. \\
\end{verse}

\begin{verse}
Wir wissen, dass ein jedes Tier \\
vom selben Leben stammt, \\
doch scheinbar einzig sehen wir \\
das Leben insgesamt. \\
\end{verse}

\begin{verse}
Nur uns’re Art hat sich erkor’n, \\
zu sehen, was wir sind: \\
ein Teil von dem, was blind gebor’n \\
durch uns zu seh’n beginnt. \\
\end{verse}

\begin{verse}
Und folglich sind wir auserwählt. \\
Denn wir allein kreier’n \\
was allen and’ren Tieren fehlt, \\
zum Beispiel Meditier’n. \\
\end{verse}

\begin{verse}
Das Leben, das uns Menschen speist, \\
trägt uns vielleicht auch an, \\
dass es um noch mehr Sterne kreist, \\
weil kein Tier sonst das kann. \\
\end{verse}

\begin{verse}
Wir nehmen diese Tiere mit, \\
weil sie zu uns gehör’n, \\
gehör’n sie auf den Flammenritt \\
mit uns als Transporteur’n. \\
\end{verse}

\begin{verse}
So viele Spiele bringen wir \\
dem ganzen Staub dort bei, \\
säen Leben aus, weit fort von hier, \\
doch bleibt es nicht dabei. \\
\end{verse}

\begin{verse}
In uns allein ist Unsereins \\
sich selbst zu seh’n gelung’n \\
und das gibt uns vor allem eins: \\
Alleinverantwortung. \\
\end{verse}

\begin{verse}
Wir antworten mit was uns kürt \\
und uns die Wege weist: \\
der Liebe, die uns Menschen führt, \\
die diese Sutra preist. \\
\end{verse}

\begin{verse}
Den Welten uns’rer Nachkommen \\
vermachen wir durchaus \\
auch diese Liebe in uns, denn \\
sie macht uns Menschen aus. \\
\end{verse}

\begin{verse}
Wir schicken auf die Wanderschaft \\
des Lebens uns’rer Welt \\
auch die uns Menschen eig’ne Kraft \\
die uns zusammenhält. \\
\end{verse}

\begin{verse}
Es gibt ein Einssein tiefer noch \\
als Liebe und Vertrau’n. \\
Die Liebe hilft auch da jedoch, \\
hilft tief genug zu schau’n. \\
\end{verse}

\begin{verse}
Vom selben Staub, im selben Spiel, \\
lässt Liebe uns versteh’n, \\
wie dort wohin die Liebe zielt, \\
wir stets uns selber seh’n. \\
\end{verse}

\begin{verse}
Wer weiß, dass wir das Selbe sind, \\
kann tief in Liebe ruh’n, \\
wo wir einander wohlgesinnt \\
stets auch uns selbst gut tun. \\
\end{verse}

\begin{verse}
Das haben alle die gemeint, \\
für die die Liebe Licht \\
ist, das den rechten Weg bescheint. \\
Die Wahrheit ist so schlicht. \\
\end{verse}

\begin{verse}
Denn nur die Liebe führt so weit, \\
dass sie uns d’rauf besinnt, \\
dass mehr uns eint als Ähnlichkeit, \\
dass wir dasselbe sind. \\
\end{verse}

\begin{verse}
Nur Sternenstaub, den Entropie \\
im selben Spiel verknüpft \\
zur selben Biosphäre die \\
dieselbe Menschheit schöpft. \\
\end{verse}

\begin{verse}
Und wenn wir wollen, dürfen wir \\
die Kraft, die uns so führ’n \\
kann, eindringlich, im Jetzt, im Hier, \\
von ganzem Herzen spür’n. \\
\end{verse}

\begin{verse}
Die Liebe, die uns Menschen führt, \\
ist was jetzt hier enthüllt, \\
wie uns das Unsereins berührt, \\
wie kein Tier sonst das fühlt. \\
\end{verse}

\begin{verse}
Die Liebe, die uns Menschen führt, \\
gewinnt, gedeiht und glückt, \\
wenn sie sich mehr als nur gespürt \\
in uns’rem Tun ausdrückt. \\
\end{verse}

\begin{verse}
Die Liebe, die uns Menschen führt, \\
ist keine Wahlheimat, \\
und keinen Pfad gibt es zu ihr, \\
denn sie ist selbst der Pfad. \\
\end{verse}

\begin{verse}
Die Liebe, die uns Menschen führt, \\
führt zu was vor uns steht, \\
zu dem was sie heraufbeschwört, \\
wohin der Pfad noch geht. \\
\end{verse}

\end{paracol}

\begin{paracol}{2}

\poemtitle{The Words our Voices raise}

\begin{verse}
We’re led through this experience\\
by rhyming poetry.\\
It’s time we get a real sense\\
how such a thing can be.
\end{verse}

\begin{verse}
How games of words our voices play\\
and thoughts that they disclose\\
unite us in a novel way\\
no other species knows.
\end{verse}

\begin{verse}
Life built on random games to shape\\
all species out of dust,\\
but only one strange kind of ape\\
now says so, and that’s us.
\end{verse}

\begin{verse}
Of course all beasts and birds can call\\
or gesture to convey\\
some signal they would share with all,\\
some thing they have to say.
\end{verse}

\begin{verse}
We all have heard excited sounds\\
in flocks and packs and herds,\\
but strangely, we have never found\\
real grammar in their words.
\end{verse}

\begin{verse}
We’ve heard most songs most birds can sing,\\
most calls of animals\\
and all are only signaling.\\
They’re not true languages.
\end{verse}

\begin{verse}
The air we humans ventilate,\\
we shape in funny ways.\\
Our human grammar complicates\\
the words our voices raise.
\end{verse}

\begin{verse}
It could be argued whales can send\\
each other thoughts much as\\
we do, but that’s an argument\\
no other species has.
\end{verse}

\begin{verse}
To see how speech is marvelous,\\
we need just give it space\\
to think and meditate on this,\\
on words our voices raise.
\end{verse}

\begin{verse}
One sound before another sound\\
holds meaning differently\\
from them the other way around.\\
“See to” is not “to see”.
\end{verse}

\begin{verse}
“Use for this” is not “use this for”\\
and it’s not “for this use”\\
or “this use for” and that’s just four\\
terms these three words produce.
\end{verse}

\begin{verse}
Per word, the terms it will permit\\
grow exponentially.\\
They’re literally infinite\\
combinatorically.
\end{verse}

\begin{verse}
Words strung together can make sense,\\
confuse us or be fun,\\
in narratives and arguments\\
and sermons like this one.
\end{verse}

\begin{verse}
With words describing words themselves,\\
like “grammar”, “joke” and “true”,\\
we make them useful for ourselves\\
and get what they can do.
\end{verse}

\begin{verse}
And even silent gestures meant\\
to say what words could say\\
are functionally equivalent\\
to words our voices raise.
\end{verse}

\begin{verse}
Our loving curiosity\\
for one another is\\
fulfilled by our ability\\
to speak our minds like this.
\end{verse}

\begin{verse}
Of course our minds are huge and rich.\\
Our lips can only reach\\
few words per second, limits which\\
constrain what’s voiced in speech.
\end{verse}

\begin{verse}
We must compress reality\\
when we communicate.\\
Words simplify complexity\\
they just approximate.
\end{verse}

\begin{verse}
This makes our words transmissible\\
and makes us pay a cost.\\
Short words make unavoidable\\
that subtleties get lost.
\end{verse}

\begin{verse}
At best, words lose less subtlety,\\
so they’re diverging less\\
from true observed reality\\
and that is truthfulness.
\end{verse}

\begin{verse}
The exponential choices of\\
grammatical words give\\
more meaning sent per word, enough\\
to be informative.
\end{verse}

\begin{verse}
Compressed, words need less space in minds\\
than full experience,\\
so they’re more easily combined\\
and built on to make sense.
\end{verse}

\begin{verse}
These – seven – words – make – up – a – line.\\
These – four – another – one.\\
And once four lines like this combine,\\
one stanza has been spun.
\end{verse}

\begin{verse}
These stanzas all have one design.\\
It’s strict and classical.\\
Its rhyme and meter give each line\\
a groove that’s musical.
\end{verse}

\begin{verse}
The words we speak have melody\\
and an unrivaled choice\\
of musical variety\\
in songs we also voice.
\end{verse}

\begin{verse}
Most who can speak can voice a strong\\
and bright melodic phrase\\
to pitch what all of us call “song”\\
in words our voices raise.
\end{verse}

\begin{verse}
One word like “song” for separate acts\\
makes them comparable\\
and can outlast the actual facts,\\
since words are durable.
\end{verse}

\begin{verse}
This makes words quite dissimilar\\
from things they name and mean.\\
Our word for that is that they are\\
ideas – things unseen.
\end{verse}

\begin{verse}
From words that we could speak as names\\
for real things we see\\
arose more abstract meta-games\\
and lots of novelty.
\end{verse}

\begin{verse}
Ideas that were practical,\\
like “one” and “two” and “three”,\\
made human lives more tractable\\
and furthered inquiry.
\end{verse}

\begin{verse}
Ideas met, and some could take\\
each other’s aspects on.\\
Their combination sometimes makes\\
a new, more complex one.
\end{verse}

\begin{verse}
Thus measuring the land we own\\
became geometry\\
and trading crops this land had grown\\
became economy.
\end{verse}

\begin{verse}
A good idea can improve\\
our lives entirely.\\
A bad one can obscure the truth\\
and cause catastrophe.
\end{verse}

\begin{verse}
Entire lives have gone to waste\\
in pointless wordplay schemes,\\
when too much value had been placed\\
in search of hopeless dreams.
\end{verse}

\begin{verse}
Ideas shape our futures, thus\\
they’re very powerful.\\
For them to not make tools of us,\\
they must be our tools.
\end{verse}

\begin{verse}
But always, an idea shared\\
creates relations of\\
shared understanding, thinking aired\\
with some degree of love.
\end{verse}

\begin{verse}
With words, we can resolve disputes.\\
And where that fails, a judge\\
who hears the case, or law statutes\\
use language just as much.
\end{verse}

\begin{verse}
Our words are always to invite\\
each other into games,\\
relationships, however slight.\\
These sermons do the same.
\end{verse}

\begin{verse}
But language not only involves\\
the words our voices raise,\\
it interacts and co-evolves\\
with thoughts in mental space.
\end{verse}

\begin{verse}
This helps us understand ourselves.\\
Like we make words for words,\\
we make new thoughts on thoughts themselves\\
and learn how thinking works.
\end{verse}

\begin{verse}
So reason and philosophies\\
have found themselves in place\\
within the strange psychologies\\
behind each voice we raise.
\end{verse}

\begin{verse}
Our ancestors built this know-how,\\
learned incrementally\\
to know themselves like us here now\\
and reason rationally.
\end{verse}

\begin{verse}
The better they could understand\\
what needed to be done,\\
the more they joined their many hands\\
and brains to work as one.
\end{verse}

\begin{verse}
All wins our ancestors achieved\\
were times when they most talked.\\
Their worst mistakes were ill-conceived\\
when talking had been blocked.
\end{verse}

\begin{verse}
We owe such massive gratitude\\
for all the work they did.\\
Our words grew with their aptitude\\
as evidence of it.
\end{verse}

\begin{verse}
We honor what they have imbued\\
us with beyond their days\\
by paying forward, and include\\
the voices children raise.
\end{verse}

\begin{verse}
By teaching kids our languages,\\
we’re giving them the keys\\
deciphering the valuables\\
in schools and libraries.
\end{verse}

\begin{verse}
We hope the kids we teach and lead\\
through words our voices raise\\
will grow to match and supersede\\
the teammates they’ll replace.
\end{verse}

\begin{verse}
Team linkages are languages.\\
When work we share today\\
needs language for new purposes,\\
new words provide the way.
\end{verse}

\begin{verse}
Since words at best approximate\\
experience and truth,\\
the truthfulness they generate\\
improves as words improve.
\end{verse}

\begin{verse}
Our ancestors could not explain\\
so much about this world.\\
What makes the sunshine? What makes rain?\\
They lacked more truthful words.
\end{verse}

\begin{verse}
Compared to us, they seem like fools,\\
but they were smart to choose\\
to build the tools that built the tools\\
that built the tools we use.
\end{verse}

\begin{verse}
Since they have passed their knowledge on,\\
it grew and grew and grew,\\
a treasure we here now have won\\
and keep on adding to.
\end{verse}

\begin{verse}
Along the stream of centuries,\\
our growing knowledge formed\\
a lineage of societies\\
increasingly informed.
\end{verse}

\begin{verse}
Each new discussion overlaid\\
what others had begun,\\
continued them and thereby made\\
all their discussions one.
\end{verse}

\begin{verse}
In places of philosophy,\\
in colleges and guilds,\\
in schools of craft and industry,\\
discussion builds and builds.
\end{verse}

\begin{verse}
Across all voices humans share,\\
one great discussion seeks\\
for good ideas that won’t care\\
whose voice it is that speaks.
\end{verse}

\begin{verse}
We joined this great discussion when\\
we learned our words as kids.\\
Until our voice grows still again,\\
we will be part of it.
\end{verse}

\begin{verse}
And part of how we all converse\\
attempts to understand\\
and to reflect the universe\\
as truly as we can.
\end{verse}

\begin{verse}
That’s how we learned how we appear\\
as this humanity\\
within the vastness that is here,\\
from out of entropy.
\end{verse}

\begin{verse}
By understanding more, we get\\
new ways we can discuss,\\
like writing, printing, Internet\\
connect each one of us.
\end{verse}

\begin{verse}
Comparing our ideas more,\\
we find that some are wrong,\\
like heavens we can now explore,\\
where angels don’t belong.
\end{verse}

\begin{verse}
The best ideas that survive\\
each other being known\\
comprise for everyone alive\\
the greatest wealth we own.
\end{verse}

\begin{verse}
To see all this as one great whole\\
allows us to discern\\
how all our voices play some role\\
in something vast that learns.
\end{verse}

\begin{verse}
As that increasingly employs\\
our words for what it seeks,\\
by now, we give machines a voice.\\
Through them, more stardust speaks.
\end{verse}

\begin{verse}
Through ever more evolving speech,\\
we gain perspective as\\
we learn of places we can reach\\
no other species has.
\end{verse}

\begin{verse}
Here now, around us everywhere\\
are lifeless, wordless worlds\\
and we prepare to travel there,\\
and teach them life and words.
\end{verse}

\begin{verse}
The galaxy around us, not\\
mere heaven is our prize.\\
We aren’t fallen angels but\\
we’re apes who learn to rise.
\end{verse}

\begin{verse}
Our rise to meet this universe\\
is young; we still have much\\
to learn together, to converse,\\
approaching truth as such.
\end{verse}

\begin{verse}
In this, the more our speech is free,\\
the better it helps solve\\
the problems that humanity\\
attacks as we evolve.
\end{verse}

\begin{verse}
Of course there’s danger in that, too.\\
Free speech brings to the fore\\
what speakers are contrary to.\\
This may include the law.
\end{verse}

\begin{verse}
All good ideas might be friends,\\
but some are now opposed\\
and we’re still beasts, where violence\\
is still sometimes too close.
\end{verse}

\begin{verse}
In silence, there is safety.\\
It hides us in some ways.\\
So there is always bravery\\
in words our voices raise.\\
\end{verse}

\begin{verse}
For silence to be meaningful,\\
it has to be a choice,\\
to consciously not use the tool\\
that is the human voice.
\end{verse}

\begin{verse}
Let’s not be silent out of fear\\
or habit or mistake.\\
Let’s try now. There’s been silence here\\
that we’re allowed to break.
\end{verse}

\begin{verse}
Let’s raise our voice together now.\\
Let’s all within this… place\\
now speak and play a game out loud\\
of words our voices… \emph{raise!}
\end{verse}

\begin{verse}
We’re dust that’s played by entropy,\\
we’re life adrift in… space,\\
with love within humanity\\
and words our voices… \emph{raise!}
\end{verse}

\begin{verse}
Of all the reasons to rejoice\\
these sermons para-phrase,\\
we are reminded by the voice\\
and words we all now… \emph{raise!}
\end{verse}

\begin{verse}
To speak together is to say\\
that we’re a team that… plays\\
a game as one, one interplay\\
of words our voices… \emph{raise!}
\end{verse}

\begin{verse}
With that, this sermon culminates.\\
Our knowledge of it… stays,\\
as each of us appreciates\\
the words our voices… \emph{raise.}
\end{verse}

\switchcolumn


\poemtitle{Die Stimme hat das Wort}

\begin{verse}
Durch diese sieben Sutras führt \\
gereimte Poesie. \\
Das fünfte zeigt, woher das rührt \\
was hier passiert, und wie. \\
\end{verse}

\begin{verse}
Jetzt schau’n wir wie die Stimme leicht, \\
im Wort Gedanken nennt, \\
durch Sprache Einigkeit erreicht \\
wie kein Tier sonst sie kennt. \\
\end{verse}

\begin{verse}
Aus Staub kam’n Spiele einst in Fahrt \\
und alles Leben hier, \\
doch einzig eine Affenart \\
hat Worte dafür: wir. \\
\end{verse}

\begin{verse}
Klar, jedes Vög’lein, jedes Tier \\
kann leise oder schrill \\
an andere signalisier’n \\
was es verbreiten will. \\
\end{verse}

\begin{verse}
Wir alle kenn’n die Laute, die \\
durch Schwärme, Herden geh’n, \\
doch haben wir darin noch nie \\
Grammatiken geseh’n. \\
\end{verse}

\begin{verse}
Wir konnten alle Lieder hör’n \\
die jeder Vogel singt, \\
und dass, was Tiere bell’n und röhr’n \\
nur Ruf, nicht Sprachen sind. \\
\end{verse}

\begin{verse}
Die Luft, die Menschen ventilier’n \\
wird sonderbar geformt. \\
Grammatiken verkomplizier’n \\
die Worte ganz enorm. \\
\end{verse}

\begin{verse}
Man mag beharr’n, dass Wale doch \\
Gedanken teil’n von sich, \\
so debattieren bleibt dennoch \\
was kein Tier sonst bespricht. \\
\end{verse}

\begin{verse}
Auf Sprache jetzt zu fokussier’n \\
ist hier der rechte Ort; \\
darüber mal zu meditier’n. \\
Die Stimme hat das Wort. \\
\end{verse}

\begin{verse}
Ein Wort vor’m anderen enthält \\
Bedeutung anders als \\
wär’n diese andersrum gestellt \\
„Mal’s da“ ist nicht „damals“ \\
\end{verse}

\begin{verse}
„Checks schon mit“ heißt nicht „schon mit Checks“ \\
und auch nicht „mit Checks schon“ \\
und diese Tripel sind von sechs \\
nur drei Permutation’n. \\
\end{verse}

\begin{verse}
Mehr Worte steigern das sehr schnell. \\
Das wächst exponentiell, \\
wird überabzählbar, speziell \\
kombinatoriell. \\
\end{verse}

\begin{verse}
Zusammen machen Worte Sinn, \\
Verwirrung oder Spaß, \\
in Streitgespräch und Sutras in \\
solch Strophen und Versmaß. \\
\end{verse}

\begin{verse}
Mit Worten für was so entsteht, \\
wie „Verben“, „Witz“ und „wahr“, \\
erschließen wir, was dadurch geht, \\
genauer – wunderbar! \\
\end{verse}

\begin{verse}
Und auch Gebärden sind nicht stumm. \\
Sie setzen Worte fort \\
und setzen uns’re Stimmen um. \\
Auch darin spricht das Wort. \\
\end{verse}

\begin{verse}
Die Neugier aufeinander aus \\
der Liebe die wir spür’n \\
erfüll’n wir, darf sie freiheraus \\
zu stimm’n’den Worten führ’n. \\
\end{verse}

\begin{verse}
Natürlich hab’n wir mehr im Kopf, \\
als uns’re Stimme pro \\
Sekunde sagen kann. Sie stopft \\
begrenzte Worte so. \\
\end{verse}

\begin{verse}
Die Wirklichkeit wird komprimiert \\
wenn wir kommunizier’n. \\
Komplikationen können wir \\
stets nur approximier’n. \\
\end{verse}

\begin{verse}
Das macht die Worte mitteilbar, \\
und kostet seinen Preis. \\
Subtilitäten gehen gar \\
verloren als Verschleiß. \\
\end{verse}

\begin{verse}
Bestmöglich fehlt Subtilität \\
nur wenig, manch Wort nimmt \\
kaum Abstand von Realität. \\
Wir nenn’n das, dass es stimmt. \\
\end{verse}

\begin{verse}
Im überabzählbaren doch \\
grammatischen Satz reist \\
pro Wort mehr Wissen, das auf noch \\
mehr Sinngehalt verweist \\
\end{verse}

\begin{verse}
Sind Worte dichter komprimiert \\
als die erlebte Welt, \\
so sind sie leichter kombiniert \\
in der Gedankenwelt. \\
\end{verse}

\begin{verse}
Sechs Worte sind die Zeile hier. \\
Und diese zweite vier. \\
Noch zwei in selbiger Manier, \\
macht diese Strophe hier. \\
\end{verse}

\begin{verse}
Die Strophen hier sind alle gleich, \\
im altmodischen Stil, \\
die Musikalität erreicht, \\
denn Takt und Reim tun viel. \\
\end{verse}

\begin{verse}
Die Stimme hat Sprachmelodie, \\
die mit dem Wort erklingt, \\
sich aufschwingt in die Harmonie \\
von Liedern die sie singt. \\
\end{verse}

\begin{verse}
Wer Wort ergreift, kann meist sodenn \\
ergreifen den Akkord \\
und singen, was wir Lieder nenn’n \\
und anstimm’n mit dem Wort. \\
\end{verse}

\begin{verse}
Ein Wort wie „Lied“ für vielerlei \\
erzeugt Vergleichbarkeit. \\
Ist der Gesang danach vorbei, \\
ist noch das Wort, was bleibt. \\
\end{verse}

\begin{verse}
Kein Wort ist wie worauf es je \\
an Dingen sich bezieht. \\
Das Wort für das ist die „Idee“, \\
ein Ding das man nicht sieht. \\
\end{verse}

\begin{verse}
Aus Worten, die erst Namen war’n \\
für Dinge die wir sah’n, \\
erwuchs bald Abstraktion heran. \\
Manch neues Spiel begann. \\
\end{verse}

\begin{verse}
Ideen von großer Dienlichkeit, \\
wie „eins“ und „zwei“ und „drei“, \\
war’n Macht und schenkten Machbarkeit \\
und noch mehr Denkerei. \\
\end{verse}

\begin{verse}
Ideen begegneten sich in \\
uns, weil sie durch uns geh’n, \\
und manchmal war das der Beginn \\
komplexerer Ideen. \\
\end{verse}

\begin{verse}
Aus der Vermessung uns’res Lands \\
entstand Geometrie. \\
Aus Tausch von dem, was angepflanzt, \\
entstand Ökonomie. \\
\end{verse}

\begin{verse}
Ideen die gut sind können uns \\
bereichern lebenslang. \\
Doch schlechte führ’n zu Irrtum und \\
verwirrtem Untergang. \\
\end{verse}

\begin{verse}
Manch ganzes Leben ward vertan. \\
In Wortspiel’n ohne Ziel \\
hing’n sie verloren Träumen an \\
und hofften viel zuviel. \\
\end{verse}

\begin{verse}
Ideen gestalten was beginnt \\
in uns’re Leben rein. \\
Damit wir nicht ihr Werkzeug sind, \\
soll’n sie die uns’ren sein. \\
\end{verse}

\begin{verse}
Wo immer man Ideen teilt, \\
entsteht durch sie ein Bund \\
von geistiger Verbundenheit \\
und Liebe ist ihr Grund. \\
\end{verse}

\begin{verse}
Durch Worte klär’n wir Streitigkeit. \\
Und hilft ein Richter sonst, \\
wär jegliche Gerichtsbarkeit \\
doch ohne Wort umsonst. \\
\end{verse}

\begin{verse}
Mit allen Worten laden wir \\
stets in Beziehungen \\
wie Spiele ein. Die Sutras hier \\
sind auch Einladungen. \\
\end{verse}

\begin{verse}
Doch Sprache ist nicht lediglich \\
die Stimme und das Wort, \\
zusamm’n mit ihnen bilden sich \\
auch die Gedanken fort. \\
\end{verse}

\begin{verse}
Das hilft uns besser zu versteh’n: \\
Wie Wort für Worte steht, \\
kann Denken durch Gedanken sehn \\
wie unser Denken geht. \\
\end{verse}

\begin{verse}
So kam es dann, dass der Verstand \\
Philosophie und nicht \\
zuletzt sich selbst in Köpfen fand \\
aus den’n die Stimme spricht. \\
\end{verse}

\begin{verse}
Die Ahnen damals horteten \\
Ideen von großem Wert, \\
die vorteilhaft antworteten, \\
verständlich und bewährt. \\
\end{verse}

\begin{verse}
Erreichen sie Einstmmigkeit \\
und Einigkeit, was nun \\
zu tun sei half Einhelligkeit \\
in ihr’m vereinten Tun. \\
\end{verse}

\begin{verse}
Was immer uns’ren Ahn’n gelang \\
gelang nur im Gespräch. \\
Bei Fehlern, die sie auch begang’n, \\
war’n die Gespräche schlecht. \\
\end{verse}

\begin{verse}
Wir schulden uns’ren Ahnen viel. \\
Sie haben viel getan \\
und Spuren davon sind subtil \\
in den Vokabular’n. \\
\end{verse}

\begin{verse}
Wir halten alles das in Ehr’n \\
und setzen ihr Werk fort \\
indem wir Worte weiter lehr’n. \\
Den Kindern gilt das Wort. \\
\end{verse}

\begin{verse}
Wir schenken Kindern Sprachen für \\
den Weg zum Schatz hinein, \\
zu Worten die sie weiterführ’n \\
in Schul’n und Büche’rein. \\
\end{verse}

\begin{verse}
Die Kinder mögen irgendwann \\
anstatt von irgendwem \\
der fort ist selbst bestimm’n und dann \\
für sie das Wort annehm’n. \\
\end{verse}

\begin{verse}
Mitsprache dient der Mitarbeit. \\
Kommt neue Arbeit her, \\
ist auch für neue Sprache Zeit, \\
erfinden wir noch mehr. \\
\end{verse}

\begin{verse}
Weil Worte immer ungenau \\
die Wahrheit nur skizzier’n, \\
lässt Sprache sich durch Neufaufbau \\
von Worten präzisier’n. \\
\end{verse}

\begin{verse}
So viel war uns’ren Ahnen nicht \\
bekannt von dieser Welt. \\
Woher kommt Regen? Sonnenlicht? \\
Kein Wort, das nicht entstelllt. \\
\end{verse}

\begin{verse}
Mit uns verglichen schein’n sie dumm, \\
doch klug war’n sie indem \\
sie schufen was jetzt wiederum \\
wir hier zum Schaffen neh’m. \\
\end{verse}

\begin{verse}
Mit immer größ’rem Wissen wuchs \\
ein großer Schatz heran. \\
Trophäen uns’res Siegeszugs. \\
Wir häufen sie noch an. \\
\end{verse}

\begin{verse}
Durch’s Strömen der Jahrhunderte \\
und immer mehr Ideen \\
war was uns Menschen wunderte, \\
allmählich klar zu seh’n. \\
\end{verse}

\begin{verse}
Wir sprachen weiter Tag und Nacht \\
und setzten fort was einst \\
gesagt war. Weitersprechen macht \\
aus all’n Gesprächen eins. \\
\end{verse}

\begin{verse}
An jeder Universität, \\
in Gilden, Schul’n und an \\
all’n Orten jeden Wissens geht \\
ein Weltgespräch voran. \\
\end{verse}

\begin{verse}
Durch alle Menschenstimm’n hindurch \\
sucht ein Gespräch nach sich \\
bewährenden Ideen, wodurch \\
es gleich ist, wer sie spricht. \\
\end{verse}

\begin{verse}
Wir sprechen darin mit, seitdem \\
man Sprechen uns gezeigt \\
hat und werd’n weiter Anteil nehm’n \\
bis uns’re Stimme schweigt. \\
\end{verse}

\begin{verse}
Ein Teil von uns’rem Weltgespräch \\
bezieht, bespricht, bedenkt \\
was wirklich ist und wahr und echt, \\
was stimmt, uneingeschränkt. \\
\end{verse}

\begin{verse}
Durch wachsende Gelehrsamheit \\
begriffen Menschen wie \\
wir hier sind, in Unendlichkeit, \\
aus Staub und Entropie. \\
\end{verse}

\begin{verse}
Durch Technik geh’n wir nicht zuletzt \\
noch mehr Gespräche ein. \\
Die Schrift, der Druck, das große Netz \\
verbinden Unsereins. \\
\end{verse}

\begin{verse}
Vergleichen unserer Ideen \\
zeigt: manche sind nicht wahr. \\
Wie Himmel, die wir klarer seh’n, \\
wo nie ein Engel war. \\
\end{verse}

\begin{verse}
Doch hab’n vereinbare Ideen \\
auch miteinander Platz \\
für gegenseitiges Versteh’n, \\
sind sie der größte Schatz. \\
\end{verse}

\begin{verse}
All das zu ahnen, irgendwie, \\
erinnert uns entfernt, \\
wie uns’re Stimmen Rollen spiel’n \\
in einem Ding, das lernt. \\
\end{verse}

\begin{verse}
Die Suche dessen endet im \\
Gespräch der Menschen nicht. \\
Jetzt geben wir Maschinen Stimm’n: \\
auch die sind Staub, der spricht. \\
\end{verse}

\begin{verse}
Weil Wort und Wissen wachsen kann, \\
und Neugier in uns brennt, \\
peil“n wir auch ferne Welten an, \\
was kein Tier sonst erkennt. \\
\end{verse}

\begin{verse}
Um uns hier jetzt und überall \\
sind Welten still und tot, \\
doch bald durchqueren wir das All, \\
bring’n Leben und das Wort. \\
\end{verse}

\begin{verse}
Zur Galaxie um uns, nicht bloß \\
zum Himmel geht’s hinauf. \\
Wir war’n nie Engel, war’n nie groß, \\
nur Affen, und woll’n rauf. \\
\end{verse}

\begin{verse}
Der Aufstieg zur Unendlichkeit \\
ist jung; wir haben viel \\
zu lernen, viel Gesprächigkeit \\
mit Wahrheit als dem Ziel. \\
\end{verse}

\begin{verse}
Je freier uns’re Rede ist, \\
je mehr hilft sie uns bei \\
der Lösung letzter Hinderniss’, \\
macht Mensch und Leben frei. \\
\end{verse}

\begin{verse}
Natürlich lauert hier Gefahr, \\
macht freie Rede klar \\
auch Widersprüche offenbar, \\
mit dem Gesetz sogar. \\
\end{verse}

\begin{verse}
Vielleicht hab’n richtige Ideen \\
einander immer gern, \\
doch and’re nicht. Als Tiere steh’n \\
wir oft Gewalt nicht fern. \\
\end{verse}

\begin{verse}
Das Schweigen gibt uns Sicherheit. \\
Es ist ein sich’rer Ort. \\
Es braucht ein bisschen Tapferkeit, \\
ergreifen wir das… Wort. \\
\end{verse}

\begin{verse}
Auch Schweigen ist bedeutungsvoll, \\
verzichten wir gezielt \\
auf’s Werkzeug Stimme, absichtsvoll, \\
dann sagt auch Schweigen viel. \\
\end{verse}

\begin{verse}
Doch schweigen wir nicht nur aus Furcht, \\
und nicht versehentlich. \\
Beenden wir das Schweigen, durch \\
das uns’re Stimme bricht. \\
\end{verse}

\begin{verse}
Erheben wir die Stimmen jetzt. \\
Wir alle hier, sofort. \\
Gemeinsam sprechen wir zuletzt. \\
Ergreifen wir das… Wort! \\
\end{verse}

\begin{verse}
Aus Staub im Spiel der Entropie, \\
belebt besteh’n wir fort, \\
sind Unsereins, in Harmonie \\
von Stimmen in ei’m… Wort! \\
\end{verse}

\begin{verse}
An jeder Sutra tiefen Grund \\
zur Freude soll hinfort \\
erinnern uns die Stimme und \\
die Worte, jedes… Wort! \\
\end{verse}

\begin{verse}
Im Chor zu sprechen macht sehr viel. \\
Wir bilden ein’n Akkord \\
als Team in ein’m Zusammenspiel \\
von Stimmen und dem… Wort! \\
\end{verse}

\begin{verse}
Und das ist dieser Sutra Schluss. \\
Ihr Wissen dauert fort \\
und jedem von uns bleibt Genuss \\
an Stimmen und dem… Wort. \\
\end{verse}

\end{paracol}

\begin{paracol}{2}

\poemtitle{Our Maps and Territory}

\begin{verse}
Each voice, all words, all life, all space,\\
this breath we now here breathe\\
appear inside our thoughts, the place\\
where they all interweave.
\end{verse}

\begin{verse}
Let’s now see thoughts themselves, explore\\
our thoughtfulness within.\\
Let’s breathe as easy as before\\
and easily begin.
\end{verse}

\begin{verse}
We all know well the snap of joy\\
of figuring things out,\\
when understanding we deploy\\
resolves some puzzling doubt.
\end{verse}

\begin{verse}
Let’s turn our thoughts on thought itself,\\
on wonders worked therein,\\
that built the skills that built the wealth\\
we’re meditating in.
\end{verse}

\begin{verse}
Our eyes are closed, but still we know\\
the shape of where we are,\\
the sky above, the earth below,\\
the planet and its star.
\end{verse}

\begin{verse}
But there is much at work between\\
the world that we conceive\\
and things directly heard and seen\\
as ears and eyes perceive.
\end{verse}

\begin{verse}
Together with our nose and skin,\\
they constantly report\\
to clever structures further in\\
that notice, learn and sort.
\end{verse}

\begin{verse}
We notice far from everything,\\
since most that’s going on\\
around us needs no noticing\\
and nothing to be done.
\end{verse}

\begin{verse}
The world looms large, but we are small\\
and must prioritize\\
among the vastness of it all\\
what’s crucial, and surprise.
\end{verse}

\begin{verse}
Too many details make a mess\\
of too much stuff to sort.\\
To oversee things, we compress\\
sensations into thought.
\end{verse}

\begin{verse}
Like here, this floor in front of us\\
has bits it’s been made of\\
that do not matter. Knowing just\\
“the floor” is good enough.
\end{verse}

\begin{verse}
Before we even recognize\\
these very words we hear,\\
our brains already itemize\\
the sounds so they’re more clear.
\end{verse}

\begin{verse}
Words recognized relate into\\
the present circumstance\\
that’s thought as well: our overview\\
of our experience.
\end{verse}

\begin{verse}
From ears that hear just moving air,\\
and eyes that see just light,\\
we learn that something’s really there,\\
a world in place outside.
\end{verse}

\begin{verse}
All sights and sounds articulate\\
surroundings that enwrap\\
us here for us to navigate,\\
a territory to map.
\end{verse}

\begin{verse}
Our memories hold maps of where\\
we’ve been, in homes and schools,\\
and wider maps we studied there,\\
that came with names and rules.
\end{verse}

\begin{verse}
Such maps keep track of paths we know\\
through territory they’re of,\\
compose the many facts below\\
in one view from above.
\end{verse}

\begin{verse}
And they include abstractions not\\
found in the territory,\\
including value things have got,\\
and names and property.
\end{verse}

\begin{verse}
For we can map ideas too,\\
relate them so they turn\\
into a worldview whereinto\\
we fit what else we learn.
\end{verse}

\begin{verse}
Our maps are not the territory!\\
They’re quite a different game.\\
But still they help, to the degree\\
they’re structured much the same.
\end{verse}

\begin{verse}
They’re always wrong, they’re always just\\
a chancy summary,\\
but we decide how much to trust\\
and use them, flexibly.
\end{verse}

\begin{verse}
We gauge the probability\\
a thought’s reliable:\\
a guess, a claim, a certainty?\\
Or not applicable?
\end{verse}

\begin{verse}
Thoughts make predictions we can test.\\
In new terrain, they must\\
predict what we will see there, lest\\
they rightly lose our trust.
\end{verse}

\begin{verse}
That’s most important when maps come\\
with an included sense\\
of things to do that will have some\\
predicted consequence.
\end{verse}

\begin{verse}
Like if we jumped from somewhere high,\\
our maps say we would die\\
and that’s a truth worth living by\\
we do not need to try.
\end{verse}

\begin{verse}
Of course that’s wonderfully good!\\
It means we’re not destroyed\\
by things we can predict we should\\
prefer to just avoid.
\end{verse}

\begin{verse}
Good maps of consequence confer\\
the strange ability\\
to seek out futures we prefer\\
from possibility.
\end{verse}

\begin{verse}
For that, we’ve used maps inside us\\
describing outside truth\\
before these human forms, as just\\
small monkeys or as shrews.
\end{verse}

\begin{verse}
We shrewdly mapped where there was food,\\
where might be predators,\\
who were our friends, or in the mood,\\
who were competitors.
\end{verse}

\begin{verse}
Since then, we’ve played with fire and\\
grown big and playful brains\\
with thoughts no beast can understand.\\
That’s how our species reigns.
\end{verse}

\begin{verse}
To know what’s true is far from all.\\
We are creative too!\\
That’s great – although some dangers call\\
when handling thoughts untrue.
\end{verse}

\begin{verse}
Our brains are eager to propose\\
new links that might belong\\
between thoughts – some are right, some close,\\
but most of them are wrong.
\end{verse}

\begin{verse}
As good maps guide our way ahead,\\
bad maps lead us astray.\\
We need to learn which maps are bad\\
to better find our way.
\end{verse}

\begin{verse}
So we seek incongruity\\
and laugh when we can find\\
thoughts wrong or contradictory,\\
like most thoughts on “the mind”.
\end{verse}

\begin{verse}
Our thoughts are quick and more complex\\
than maps they can contain.\\
They hardly can keep track of tracks\\
they take across a brain.
\end{verse}

\begin{verse}
So when we know ourselves, we don’t\\
know each thought separately,\\
but just a summary that won’t\\
exceed capacity.
\end{verse}

\begin{verse}
So we are prone to simplify\\
all thoughts into a kind\\
of monolithic thing that we\\
proceed to call “the mind”.
\end{verse}

\begin{verse}
Like voices in a choir choose\\
to melt into one voice,\\
our many thoughts seem to produce\\
one mind that makes its choice.
\end{verse}

\begin{verse}
Up close, the “minds” we’re looking at\\
are not big solid things,\\
but rather lots of small things that\\
are busy happening.
\end{verse}

\begin{verse}
So most we now can tell apart\\
we’ve only learned about\\
since scholars of the mind worked hard\\
to map the details out.
\end{verse}

\begin{verse}
To work on their self-inquiries,\\
they learned to meditate,\\
created methodologies\\
to self-investigate.
\end{verse}

\begin{verse}
Thoughts are transparent, focusing\\
on content they think of,\\
but we can practice noticing\\
their courses well enough.
\end{verse}

\begin{verse}
Some mindful meditations take\\
thoughts from the tasks they know\\
to states that make transparence break\\
and let their innards show.
\end{verse}

\begin{verse}
Some poisons change thoughts so they stir\\
against transparence; this\\
has let us study thoughts that were\\
in metamorphosis.
\end{verse}

\begin{verse}
Another subject that can give\\
us insight is the brain.\\
We’ve studied brains, beginning with\\
those injured or insane.
\end{verse}

\begin{verse}
Brain imaging technology\\
has made us able to\\
still crudely but increasingly\\
observe the work brains do.
\end{verse}

\begin{verse}
We’ve seen small streaks of lightning climb\\
through brains, embody all\\
our thoughts on paths through space and time.\\
They’re four-dimensional.
\end{verse}

\begin{verse}
Like humans, thoughts on paths that meet,\\
will sometimes get aligned,\\
becoming something more complete\\
and grow through time entwined.
\end{verse}

\begin{verse}
Thoughts meeting, if the territory\\
they’re mapping overlaps,\\
obtain an opportunity\\
to share and merge their maps.
\end{verse}

\begin{verse}
When thoughts connect, their unity\\
is seen when neurons they\\
are running on find synchrony\\
and pulse a rhythmic way.
\end{verse}

\begin{verse}
Thus we map out how thoughts map out\\
and learn and represent\\
the territory we learn about\\
from light and sound and scent.
\end{verse}

\begin{verse}
In science, as in every brain,\\
what thoughts and scholars do\\
are similar attempts to gain\\
some map of what is true.
\end{verse}

\begin{verse}
This mapping of the world goes on\\
in every so-called “mind”.\\
How human sciences are done\\
is merely more refined.
\end{verse}

\begin{verse}
How science is considering\\
the weight of evidence\\
is near-exactly mirroring\\
how thoughts are making sense.
\end{verse}

\begin{verse}
Through this, we cumulatively\\
have come to see up close\\
some secrets of reality\\
no other species knows.
\end{verse}

\begin{verse}
No other species talks like us\\
and learns collectively,\\
collects improvements that adjust\\
maps to the territory.
\end{verse}

\begin{verse}
Like grammar, thoughts recursively\\
encode small simple things\\
in higher-order structures we\\
use for remembering.
\end{verse}

\begin{verse}
We even map out consequence,\\
cause and causality\\
in language, so our species tends\\
to think grammatically.
\end{verse}

\begin{verse}
Our clumsy wordlike thoughts assure\\
mistakes their use incurs.\\
Our grammared thoughts caricature\\
the wordless universe.
\end{verse}

\begin{verse}
Nobody acts as separately\\
as we tend to believe,\\
as subjects thought grammatically.\\
That’s just how we conceive.
\end{verse}

\begin{verse}
But knowledge that’s approximate\\
still helps, so we make do\\
with thoughts we hope or estimate\\
approximately true.
\end{verse}

\begin{verse}
We used to think that storms proclaimed\\
some wrathful heaven’s king.\\
We know now that’s an instinct named\\
anthropomorphizing.
\end{verse}

\begin{verse}
Perhaps such humanizing used\\
to help to some degree.\\
Those too were thoughts, just more confused\\
than those thought presently.
\end{verse}

\begin{verse}
We err in what we think and see,\\
our biases are strong,\\
but learning from the territory,\\
we can become less wrong.
\end{verse}

\begin{verse}
Like thoughts begin to resonate\\
when neurons get in sync,\\
we learn when we communicate.\\
Shared reason helps us link.
\end{verse}

\begin{verse}
True maps describe one territory\\
and can become agreed\\
upon the same reality.\\
The truth is where we meet.
\end{verse}

\begin{verse}
Through sharing thoughts, a kind of “mind”\\
can use more brains to plan\\
and execute, as well-aligned\\
as only humans can.
\end{verse}

\begin{verse}
All truthful insight humans share\\
comprises something vast,\\
one map all humans everywhere\\
create to grow and last.
\end{verse}

\begin{verse}
We fail and halt and stumble on\\
new paths confusion brings.\\
We’re first to map new paths we’ve gone.\\
We’re first at many things.
\end{verse}

\begin{verse}
Through teamwork, incrementally,\\
this common map we’ve grown\\
gained secrets of reality\\
that waited to be known.
\end{verse}

\begin{verse}
Sure, still it’s wrong, but less and less.\\
We’ll see more clearly still\\
as we improve its truthfulness\\
as fully as we will.
\end{verse}

\begin{verse}
We yearn for knowledge, hope we’re wise,\\
crave taking right from wrong,\\
take part in one great enterprise\\
to which we all belong.
\end{verse}

\begin{verse}
A single overarching map\\
that all of us improve\\
may threaten to become a trap,\\
mistaken for the truth.
\end{verse}

\begin{verse}
But past all maps we comprehend,\\
we know reality\\
to inexhaustibly extend\\
towards infinity.
\end{verse}

\begin{verse}
The more such territory we know,\\
the more we’re filled with awe\\
at everything that’s let us grow\\
to see its beauty more.
\end{verse}

\begin{verse}
So much was needed to allow\\
our chests to rise and fall\\
and fuel our thoughts with which we now\\
appreciate it all.
\end{verse}

\begin{verse}
For we owe every thought we think\\
to stardust playing games\\
of life and love and words to link\\
us up to greater aims.
\end{verse}

\begin{verse}
From feeling the sheer magnitude\\
of all that made us real\\
arises awestruck gratitude\\
we’ve just begun to feel.
\end{verse}

\begin{verse}
The gift of life our world shall spread\\
gives one response to this.\\
We’ll also map our paths ahead.\\
We will stay curious.
\end{verse}

\begin{verse}
We’ll see our cosmos differently,\\
but every knowing self\\
remains a way the territory,\\
the cosmos, knows itself.
\end{verse}

\switchcolumn

\poemtitle{Die Bilder von der Wirklichkeit}

\begin{verse}
Das ganze Leben, jedes Wort \\
und dieser Atem jetzt \\
erschein’n im Denken, jenem Ort \\
der alles das vernetzt. \\
\end{verse}

\begin{verse}
Nehm’n wir das Denken selbst uns vor, \\
den Denkprozess hier drin. \\
Behaglich atmend wie zuvor, \\
schau’n wir zum Denken hin. \\
\end{verse}

\begin{verse}
Wir alle kennen doch das Glück \\
wenn wir Versteh’n aufbau’n, \\
und das uns hilft, dass wir ein Stück \\
Verwirrung erst durchschau’n. \\
\end{verse}

\begin{verse}
Versteh’n wir jetzt auch das Versteh’n \\
und was Gedanken tun, \\
wie die, die bauten, was wir seh’n, \\
den Ort, in dem wir ruh’n. \\
\end{verse}

\begin{verse}
Zu sind die Augen, aber wir \\
erkenn’n den Ort, uns fällt \\
leicht ein wo wir sind alle hier, \\
im Raum und auf der Welt. \\
\end{verse}

\begin{verse}
Doch vieles war vonnöten, um \\
das zu ermöglichen, \\
viel Arbeit zwischen Wahrnehmung \\
und was wir blind erkenn’n. \\
\end{verse}

\begin{verse}
Die Augen, Ohren, Nase, Haut \\
berichten stets zu ihr’n \\
Verarbeitungs-Struktur’n, gebaut \\
zum Ordnen und Sortier’n. \\
\end{verse}

\begin{verse}
Und längst nicht alles wird bemerkt. \\
Nicht alles was zurzeit \\
hier um und in uns rumfuhrwerkt \\
bedarf Aufmerksamkeit. \\
\end{verse}

\begin{verse}
Die Welt ist groß, doch wir sind klein, \\
weshalb das Denken wählt \\
was aus dem großen Allgemein’n \\
erstaunt, was für uns zählt. \\
\end{verse}

\begin{verse}
Details der Wahrnehmungen sind \\
zu zahlreich zum Sortier’n. \\
Wir überblicken, was wir in \\
Gedanken komprimier’n. \\
\end{verse}

\begin{verse}
Der Grund dort unten hat Struktur, \\
aus Teil’n zusamm’ngefügt, \\
doch nennen wir das alles nur \\
den „Boden“, das genügt. \\
\end{verse}

\begin{verse}
Bevor wir dieses… Wort… erkenn’n, \\
zerteilt das Hören schon \\
Geräusche in Phoneme, denn \\
das hilft der Kognition. \\
\end{verse}

\begin{verse}
Erkannte Worte werden schlicht \\
verknüpft mit der Präsenz \\
der ausgedachten Übersicht \\
des jetzigen Moments. \\
\end{verse}

\begin{verse}
Weil Luft sich an den Ohren regt \\
und Licht in Augen fällt \\
begreifen wir die unentwegt \\
vorhand’ne echte Welt. \\
\end{verse}

\begin{verse}
Licht und Geräusch artikulier’n \\
was wirklich ist und fest \\
im Raum vorhanden und jetzt hier \\
uns sich abbilden lässt. \\
\end{verse}

\begin{verse}
Auch Bilder in Erinnerung’n \\
wie wir zu Bildern kam’n \\
Zuhause und durch Schulbildung, \\
zu Landkarten mit Nam’n. \\
\end{verse}

\begin{verse}
Die Bilder weisen uns den Weg \\
durch uns’re Wirklichkeit, \\
verleihen uns wohlüberlegt \\
die Übersichtlichkeit. \\
\end{verse}

\begin{verse}
Dazu gehören Abstraktion’n, \\
die man im Land nicht sieht, \\
wie Name und Besitz von schon \\
bebildertem Gebiet. \\
\end{verse}

\begin{verse}
Denn wir kartieren auch Ideen \\
in uns’rer Sicht der Welt, \\
in die wir ordnen, was uns den \\
Zusammenhang herstellt. \\
\end{verse}

\begin{verse}
Die Bilder sind nicht Wirklichkeit! \\
Sie sind ja Bilder nur \\
und helfen dennoch, insoweit \\
von ähnlicher Struktur. \\
\end{verse}

\begin{verse}
Sie stimmen nie in den genau’n \\
Details, sind falsch und grob. \\
Vielleicht woll’n wir ein’m Bild vertrau’n, \\
erscheint es gut erprobt. \\
\end{verse}

\begin{verse}
Wir prüfen, wie verlässlich ein \\
Gedankenbild erscheint. \\
Wie sehr mag es wahrscheinlich sein? \\
Wie ernst ist es gemeint? \\
\end{verse}

\begin{verse}
Sie prophezei’n manch Sachverhalt \\
betreffs der Wirklichkeit. \\
Wenn der nicht stimmt, verlier’n sie bald \\
Vertrau’n und Gültigkeit. \\
\end{verse}

\begin{verse}
Besonders wichtig, wenn eins die \\
Auswirkungen durch das \\
was wir tun zeigt. Wodurch wird wie \\
wahrscheinlich werden was? \\
\end{verse}

\begin{verse}
Zum Beispiel, eines nur von viel’n, \\
wär’n wir laut Bildern tot, \\
wenn wir von sehr weit oben fiel’n. \\
Das prüfen tut nicht Not. \\
\end{verse}

\begin{verse}
Wie wunderbar uns sowas nützt! \\
Es heißt wir sterben nicht \\
in Wirklichkeit, das Abbild schützt \\
indem’s für uns zerbricht. \\
\end{verse}

\begin{verse}
Und sagen Bilder gut voraus, \\
entsteht die Fähigkeit, \\
die Zukunft selbst zu wählen aus \\
ungleicher Möglichkeit. \\
\end{verse}

\begin{verse}
Drum gab’s die inn’ren Bilder von \\
was draußen wirklich wär \\
schon vor den Menschen, war’n sie schon \\
für Tiere sehr viel wert. \\
\end{verse}

\begin{verse}
Durch Bilder von wo Nahrung sei, \\
wo Feinde lauerten, \\
wo Freunde, kam Erfolg herbei. \\
Sie überdauerten. \\
\end{verse}

\begin{verse}
Dann spielten wir das Kokelspiel, \\
dann wuchs das Bildorgan, \\
das denkt, was keinem Tier einfiel, \\
und Macht gab uns’ren Ahn’n. \\
\end{verse}

\begin{verse}
Wir woll’n nicht nur, was Wahrheit ist. \\
Wir sind auch kreativ! \\
Gefährlich wird’s, wenn man vergisst, \\
was wahr ist, was fiktiv. \\
\end{verse}

\begin{verse}
Verspielte Hirne haben gern \\
Verknüpfungen zu dicht \\
verbund’nen Bildern, um zu lern’n, \\
doch oftmals stimm’n sie nicht. \\
\end{verse}

\begin{verse}
Durch falsche Bilder könn’n wir schlimm \\
Vom rechten Weg abkomm’n. \\
Wir prüfen, welche Bilder stimm’n, \\
um gut voranzukomm’n. \\
\end{verse}

\begin{verse}
Wir suchen nach Unstimmigkeit \\
und lachen, weil sie heißt \\
ein Bild ist falsch und todgeweiht, \\
wie viele über „Geist“. \\
\end{verse}

\begin{verse}
Gedanken sind zu schnell, zu klein \\
für Bilder in ihn’n drin \\
und fall’n uns Bilder für sie ein, \\
sind sie bereits dahin. \\
\end{verse}

\begin{verse}
Betrachten wir Gedanken, seh’n \\
wir sie nicht separat, \\
nur einen Überblick, für den \\
ein einziger Raum hat. \\
\end{verse}

\begin{verse}
Wir neigen deshalb beim Erkenn’n \\
des Denkens in uns meist \\
zu einem ganzen Ding und nenn‘n \\
das Trugbild dann den „Geist“. \\
\end{verse}

\begin{verse}
Das scheint dann wie ein Chor der sich \\
entscheidet für ein Lied, \\
Gedanken klingen einstimmig, \\
ein Geist der sich geschieht. \\
\end{verse}

\begin{verse}
Genau betrachtet ist der „Geist“ \\
geflochten wie ein Zopf \\
aus viel’n Gedanken. Jeder reist \\
geschäftig durch den Kopf. \\
\end{verse}

\begin{verse}
Das meiste, was man heute weiß, \\
lässt sich erst jetzt studier’n, \\
seit Wissenschaftler aus Details \\
ein bess’res Bild kreier’n. \\
\end{verse}

\begin{verse}
Um Nachforschungen anzustell’n \\
begann’n sie meditier’n, \\
gezielt methodisch ganz speziell \\
Gedanken zu studier’n. \\
\end{verse}

\begin{verse}
Gedanken zeigen transparent \\
ihr Bild, nicht das was denkt, \\
doch wer erkennen übt, erkennt \\
ihr’n Ablauf eingeschränkt. \\
\end{verse}

\begin{verse}
Meditative Achtsamkeit \\
holt die Gedanken raus \\
aus transparenter Emsigkeit, \\
zieht Einsichten daraus. \\
\end{verse}

\begin{verse}
Wenn Gifte die Gedanken stör’n \\
in ihrer Transparenz, \\
erleichtert das, sie zu studier’n \\
bei mangelnder Stringenz. \\
\end{verse}

\begin{verse}
Das Studium des Hirnes macht \\
noch mehr vom Denken klar. \\
Drum haben wir das Hirn durchdacht, \\
zunächst wenn’s schadhaft war. \\
\end{verse}

\begin{verse}
Bildgebende Verfahren nun \\
eröffnen, erst nicht gut, \\
doch wachsend, was Gehirne tun \\
und was sich darin tut. \\
\end{verse}

\begin{verse}
In Blitzen fährt die Leiblichkeit \\
der Bilder integral \\
durch das Gehirn, durch Raum und Zeit \\
und vierdimensional. \\
\end{verse}

\begin{verse}
Wie Menschen geh’n Gedanken die \\
sich treffen manchmal auf \\
einander ein, verknüpfen sie \\
ihr’n weiteren Verlauf. \\
\end{verse}

\begin{verse}
Das gibt ihnen die Möglichkeit, \\
die Bilder der einstweil’n \\
gemeinten einen Wirklichkeit \\
zu einen und zu teil’n. \\
\end{verse}

\begin{verse}
Verbinden sich Gedanken, so \\
erkenn’n wir das exakt. \\
Neuronen feuern dann synchron \\
in einem gleichen Takt. \\
\end{verse}

\begin{verse}
Wir seh’n wie in Gedanken sehr \\
schnell Bild an Bild sich reiht, \\
durch Licht, Geruch und Ton aus der \\
erkannten Wirklichkeit. \\
\end{verse}

\begin{verse}
Gelehrte in der Wissenschaft \\
woll’n, ganz wie im Gehirn \\
Gedanken, mit vereinter Kraft \\
die Wirklichkeit entwirr’n. \\
\end{verse}

\begin{verse}
Die Bilder von der Welt passier’n \\
in sogenannten „Geistern“. \\
Die Wissenschaften funktionier’n \\
so ähnlich, nur gescheiter. \\
\end{verse}

\begin{verse}
Wie Wissenschaft aus Evidenz \\
schlussfolgerungen baut, \\
gleicht Denken, das aus Kongruenz \\
von Bildern schließt, genau. \\
\end{verse}

\begin{verse}
So haben wir seit langer Zeit \\
allmählich aufgeklärt \\
was tief in uns’rer Wirklichkeit \\
kein and’res Tier erfährt. \\
\end{verse}

\begin{verse}
Wir sammeln wie kein and’res Tier \\
Verbesserungen an. \\
Mit bess’ren Bildern nähern wir \\
der Wirklichkeit uns an. \\
\end{verse}

\begin{verse}
Gedanken sammeln Fakten wie \\
Grammatik stetig in \\
komplexeren Strukturen die \\
sich leicht zu merken sind. \\
\end{verse}

\begin{verse}
Wir bilden Zukunft, wo die Fahrt \\
der Auswirkung’n hin lenkt, \\
in Sprache ab, weil uns’re Art \\
so gern grammatisch denkt. \\
\end{verse}

\begin{verse}
Solch grobes Denken garantiert, \\
dass sich viel Irrtum zeigt. \\
Grammatisch denken karikiert \\
die Wirklichkeit, die schweigt. \\
\end{verse}

\begin{verse}
Kein Mensch agiert im Endeffekt \\
allein auf weiter Flur \\
wie ein grammatisches Subjekt. \\
So denken Menschen nur. \\
\end{verse}

\begin{verse}
Doch ungefähres Wissen hilft \\
schon mal, begnügen wir \\
uns erstmal mit was etwa gilt, \\
was nur approximiert. \\
\end{verse}

\begin{verse}
Wir dachten mal, im Sturm wär Hass \\
von Göttern, die regier’n. \\
Das war Instinkt, wir nennen das \\
Anthropomorphisier’n. \\
\end{verse}

\begin{verse}
Vermenschlichung mag uns einstmals \\
geholfen haben noch. \\
Auch das war’n Bilder, falscher als \\
die jetzigen jedoch. \\
\end{verse}

\begin{verse}
Obwohl wir fehlgeh’n allezeit, \\
obwohl sich jeder irrt, \\
macht Lernen aus der Wirklichkeit \\
uns weniger verwirrt. \\
\end{verse}

\begin{verse}
Wie die Neuronen harmonier’n \\
wenn Bilder sich verein’n, \\
lässt Lernen uns kommunizier’n \\
durch den Verstand allein. \\
\end{verse}

\begin{verse}
Denn Bilder mit Wahrhaftigkeit \\
erwachsen allesamt \\
aus einzig einer Wirklichkeit. \\
Die Wahrheit führt zusamm’n. \\
\end{verse}

\begin{verse}
Gedanken die Gehirne teil’n \\
entwächst sowas wie’n „Geist“, \\
mit Einmut, Einsichten bisweil’n, \\
die kein Tier sonst erreicht. \\
\end{verse}

\begin{verse}
Was wir gemeinsam gut erkenn’n \\
ergibt ein Riesenbild \\
des Wissens, wie wir sowas nenn’n, \\
das wächst und weiter gilt. \\
\end{verse}

\begin{verse}
Wir irren, stolpern, humpeln wild \\
voran, komm’n unverhofft \\
als erste zu ein’m neuen Bild. \\
Die Ersten sind wir oft. \\
\end{verse}

\begin{verse}
Durch Teamwork und Bedachtsamkeit \\
hat unser großes Bild \\
Geheimnisse der Wirklichkeit \\
uns immer mehr enthüllt. \\
\end{verse}

\begin{verse}
Es stimmt noch nicht, doch mehr und mehr, \\
und noch mehr Klarheit bringt \\
noch mehr Wahrhaftigkeit, so sehr \\
uns dieses Bild gelingt. \\
\end{verse}

\begin{verse}
Gern haben wir dadurch entdeckt \\
und Irrtümer zerstört, \\
als Bildner in ein’m Großprojekt, \\
dem jeder angehört. \\
\end{verse}

\begin{verse}
Wird so ein Bild, so lang und breit \\
gezeichnet und vereint, \\
verwechselt mit der Wirklichkeit, \\
kann’s eine Falle sein. \\
\end{verse}

\begin{verse}
Denn über jedes Bild hinaus \\
dehnt doch die Wirklichkeit \\
sich unerschöpflich weiter aus \\
In die Unendlichkeit. \\
\end{verse}

\begin{verse}
Je mehr sich Wirklichkeit erschließt, \\
wird auch mehr Glück entsteh’n \\
dass sie auch uns entstehen ließ \\
um ihre Pracht zu seh’n. \\
\end{verse}

\begin{verse}
So viel war nötig, bis zuletzt \\
sich uns’re Brust erhebt, \\
Kraft atmet für das Denken jetzt \\
das alles das erlebt. \\
\end{verse}

\begin{verse}
Denn jeglichen Gedanken hier, \\
Der auf die Wahrheit zielt, \\
sie liebt und spricht, verdanken wir \\
dem Sternenstaub, der spielt. \\
\end{verse}

\begin{verse}
Das Ausmaß uns’rer Wirklichkeit \\
mit uns ganz tief darin \\
beflügelt tiefste Dankbarkeit, \\
die tief in uns beginnt. \\
\end{verse}

\begin{verse}
Das Leben, das Geschenk der Welt, \\
genügt als Dank noch nicht. \\
Sie sei auch bildhaft dargestellt. \\
Wir bleiben neugierig. \\
\end{verse}

\begin{verse}
In unserer Verschiedenheit \\
gibt doch ein jeder Acht \\
mit Augen einer Wirklichkeit, \\
die Bilder von sich macht. \\
\end{verse}

\end{paracol}

\begin{paracol}{2}

\poemtitle{The Universe Machine}

\begin{verse}
Six sermons we have had so far.\\
Is there a point to this?\\
To meditating even more?\\
Oh yes. Oh yes, there is.
\end{verse}

\begin{verse}
We’ve watched six sequent layers of\\
the Universe Machine.\\
The atoms, playing, life and love…\\
these four have set the scene.
\end{verse}

\begin{verse}
The scene in which our voices rise,\\
that’s mapped inside the brain.\\
What more is there to realize?\\
What insight to attain?
\end{verse}

\begin{verse}
Beyond the truths recalled so far,\\
there is who does recall.\\
It’s time to notice who we are;\\
who’s noticing it all.
\end{verse}

\begin{verse}
Who is the one that’s noticing\\
and who is asking who?\\
Who’s looking out at everything?\\
Who wonders what is true?
\end{verse}

\begin{verse}
Six sermons told what both was true\\
and also led to this,\\
for us here to discover who\\
the one who’s looking is.
\end{verse}

\begin{verse}
Let’s feel ourselves, from inside out,\\
in this familiar place,\\
where we have come to dream about\\
our lives in time and space.
\end{verse}

\begin{verse}
This breathing body here extends\\
from feet to chest and head,\\
the back, the belly, both these hands…\\
to whom belongs all that?
\end{verse}

\begin{verse}
We’re not just words like “I” or “me”,\\
nor names that we possess,\\
since we have splendid vibrancy\\
and words have so much less.
\end{verse}

\begin{verse}
They’re aspects of what’s known of us,\\
but that which knows, they’re not.\\
Our concepts of ourselves are just\\
anthropomorphized thought.
\end{verse}

\begin{verse}
They’re only maps that obviously\\
don’t live, don’t breathe, don’t care.\\
What are we in the territory\\
that’s really, truly, there?
\end{verse}

\begin{verse}
Let’s feel ourselves, the self that dwells\\
within, wherein it seems\\
a thousand million million cells\\
bring forth the dream who dreams.
\end{verse}

\begin{verse}
Each cell holds countless games that run\\
on dust the stars provide,\\
and still we seem to us just one\\
we feel from the inside.
\end{verse}

\begin{verse}
Behind the senses that perceive\\
this whole experience,\\
let’s find who’s in here to receive.\\
Who is its audience?
\end{verse}

\begin{verse}
Behind our eyes, between our ears,\\
there’s something happening,\\
among our thoughts it would appear\\
there’s someone noticing.
\end{verse}

\begin{verse}
It’s hard to notice noticing.\\
It has a kind of stealth.\\
But prior to each noticed thing\\
is noticing itself.
\end{verse}

\begin{verse}
We notice, so we’re purposeful,\\
intentional, aware,\\
deliberate, willful, personal,\\
unlike much else out there.
\end{verse}

\begin{verse}
Who has these purposes and tasks?\\
Who has experience?\\
Who is the one that always asks?\\
Who feels the things we sense?
\end{verse}

\begin{verse}
A private view, most close to us,\\
appears mysterious,\\
not obviously made of dust.\\
We call it consciousness.
\end{verse}

\begin{verse}
It knows things with immediacy\\
and can attend to some\\
of what we know selectively,\\
till other, new things come.
\end{verse}

\begin{verse}
And it can know itself as well.\\
Self-noticing applied\\
adeptly is how we can tell\\
it’s thought, seen from inside.
\end{verse}

\begin{verse}
When thoughts that know each other form\\
shared knowledge to engulf\\
them all, a special thought is born:\\
a thought that knows itself.
\end{verse}

\begin{verse}
A thought that maps what it is like\\
to be us when and where\\
we are, what makes us each unlike\\
the rest of us out there.
\end{verse}

\begin{verse}
Like every thought, it is constrained\\
in scope, what it can keep\\
inside itself. And what’s contained\\
is gone in dreamless sleep.
\end{verse}

\begin{verse}
Anthropomorphizing itself,\\
it’s conscious but consists\\
of but a brain’s brief thought of self.\\
The truth of us persists.
\end{verse}

\begin{verse}
For all self-knowing thoughts must end.\\
They must depart to thus\\
make room for all that may transcend\\
the things we think are us.
\end{verse}

\begin{verse}
So that’s not us. We won’t be caught\\
by something that we’re not.\\
All consciousness is simply what\\
it feels like to be thought.
\end{verse}

\begin{verse}
In all that conscious moments do,\\
they rise and fall away.\\
So they are not the answer who\\
keeps asking “who” today.
\end{verse}

\begin{verse}
With consciousness, we’re still not done,\\
but we have come so far,\\
we might as well move further on\\
to who we really are.
\end{verse}

\begin{verse}
We’re more than dust that long has been\\
asleep and purposeless,\\
since we, where now this dust is in,\\
do notice purposes.
\end{verse}

\begin{verse}
We’re more than cells that come and go\\
and take each others’ place,\\
we’re more than life, do more than grow\\
or even spread through space.
\end{verse}

\begin{verse}
We’re more than love, although it’s brought\\
about much awesomeness.\\
We’re more than voices, more than thoughts.\\
We’re more than consciousness.
\end{verse}

\begin{verse}
We’re not a layer on this list\\
of this machine’s domains.\\
When every falsehood is dismissed,\\
the truth is what remains.
\end{verse}

\begin{verse}
So much has led right up to this.\\
Now this is what we do:\\
we meditate. We’re good at this,\\
at something breaking through.
\end{verse}

\begin{verse}
Who’s having this experience?\\
Who’s having all our thoughts?\\
Who feels this feeling of suspense\\
as we approach its source?
\end{verse}

\begin{verse}
Who has these lives we live, that came\\
to hear these words, this rhyme?\\
Who has been playing all these games?\\
Who’s having space and time?
\end{verse}

\begin{verse}
We’re all adrift since one big bang\\
made toys for entropy\\
to craft one life from which then sprang\\
this one humanity.
\end{verse}

\begin{verse}
One cosmos has each voice and thought\\
inside itself and hence\\
it’s having everything we’ve got,\\
like this experience.
\end{verse}

\begin{verse}
So much made up each consciousness\\
and they’re so numerous,\\
they must belong to nothing less\\
than all the universe.
\end{verse}

\begin{verse}
In being conscious, we all share\\
one sense that feels and asks.\\
The living cosmos seems to wear\\
our faces as its masks.
\end{verse}

\begin{verse}
All consciousness that we have thought,\\
was really thought by this,\\
by all the universe that brought\\
about all consciousness.
\end{verse}

\begin{verse}
This universe brought forth and plays\\
our lives, so we who delve\\
within are living, conscious ways\\
the cosmos knows itself.
\end{verse}

\begin{verse}
So many lives, so many forms\\
in which we are immersed;\\
each plays a role as each transforms\\
the conscious universe.
\end{verse}

\begin{verse}
These roles are not what’s happening\\
should be mistaken for.\\
They’re just anthropomorphizing\\
the universe we are.
\end{verse}

\begin{verse}
These eyes through which we each have seen\\
were masks for us to drop.\\
We are the Universe Machine\\
and we are waking up…
\end{verse}

\begin{verse}
…as we unstoppably proceed,\\
progress and learn, explore,\\
inexorably we succeed\\
and flower ever more…
\end{verse}

\begin{verse}
…as ever new epiphanies\\
enrich the knowledge used\\
to unclose new abilities,\\
make humans less confused,…
\end{verse}

\begin{verse}
…as love that guides humanity\\
must seek magnificence\\
in peace on Earth that naturally\\
makes room for progress thence,…
\end{verse}

\begin{verse}
…as human flourishing expands,\\
through trial and error still,\\
this first such species must advance\\
life’s beauty, strength and will…
\end{verse}

\begin{verse}
…as embryonic life nears birth\\
and its delivery\\
on starships from idyllic Earth\\
towards infinity,…
\end{verse}

\begin{verse}
…as entropy’s evolving toys\\
grow necessarily\\
to gift their incandescent joys\\
throughout our galaxy…
\end{verse}

\begin{verse}
…a galaxy that must awake,\\
like this one planet has,\\
to consciousness of what to make\\
of all the cosmos, as…
\end{verse}

\begin{verse}
…we share this sermon joyfully,\\
preparing that, as we\\
invoke our cosmic unity,\\
our vast machinery.
\end{verse}

\begin{verse}
In this momentous moment, we\\
can consciously be one,\\
and as we knew initially,\\
that’s going to be fun!
\end{verse}

\begin{verse}
We’re all of this machinery.\\
We’re the evolving gift\\
of ever more complexity\\
from stardust that’s adrift.
\end{verse}

\begin{verse}
Beyond our separate selves, we see\\
one cosmos that behaves\\
as one uniting endless sea\\
with many separate waves.
\end{verse}

\begin{verse}
True peace is just humility\\
at whence we all have come\\
and where we’ll go, this unity\\
we’re never separate from.
\end{verse}

\begin{verse}
Each person one persona for\\
reality that takes\\
phenomenal new forms in all\\
creations that it makes.
\end{verse}

\begin{verse}
To deeply know this is to feel\\
euphoria at how\\
the focal point of all that’s real\\
surmounts this moment now.
\end{verse}

\begin{verse}
Let’s have it. We’re not delicate.\\
Let’s savor joy and awe\\
and wonder that’s appropriate\\
for who we truly are.
\end{verse}

\begin{verse}
The universe’s elegance\\
is straight-up glorious\\
and we are it, in resonance\\
and thus victorious.
\end{verse}

\begin{verse}
All that will be, that is, that’s been\\
we know to integrate\\
as parts of us, of this machine,\\
that interpenetrate.
\end{verse}

\begin{verse}
So there is truly unity,\\
a true way we are one.\\
So we should speak as “I”, not “we”,\\
an “I” that’s everyone.
\end{verse}

\begin{verse}
The many eyes through which I’ve seen\\
were masks for me to drop.\\
I am the Universe Machine\\
and I am waking up.
\end{verse}

\begin{verse}
I’m one, the cosmos, noticing\\
and I’ve been asking “who”.\\
I’m looking. I am everything.\\
I’m wonder and I’m true.
\end{verse}

\begin{verse}
I’m who is wearing all these masks,\\
who has experience.\\
I am the universe that asks.\\
I sense with every sense.
\end{verse}

\begin{verse}
With all my minds, I seek to see,\\
to learn and then pursue\\
the purpose of reality,\\
of me, with all I do.
\end{verse}

\begin{verse}
I’m the entire territory,\\
and all my voices speak\\
my words in vast diversity,\\
each me and each unique.
\end{verse}

\begin{verse}
In love that guides humanity,\\
I’m all of me, and I’m\\
a conscious game of entropy\\
adrift in space and time.
\end{verse}

\begin{verse}
\ldots
\end{verse}

\begin{verse}
Now this self-knowing thought must end.\\
I must depart, thereby\\
make room for all that may transcend\\
the things I think are I.
\end{verse}

\begin{verse}
For that, again, is just a thought!\\
A map! A point of view\\
that we are free to use or not\\
in everything we do.
\end{verse}

\begin{verse}
Its use feels like we felt before\\
and every conscious brain\\
of mine may feel like me once more\\
by hearing this again.
\end{verse}

\begin{verse}
We’re free to feel now: either we’re\\
one whole reality\\
or just one breathing body here.\\
And that’s our victory!
\end{verse}

\begin{verse}
Whatever else is true for us,\\
we’ll always have this choice:\\
two ways to face all things we must,\\
all horrors and all joys.
\end{verse}

\begin{verse}
We shall go on. We shall go far,\\
empowered since we’ve seen\\
and may remember who we are:\\
the Universe Machine.
\end{verse}

\begin{verse}
We’ve finished something few have tried\\
and so we might observe\\
we’re feeling proud or satisfied.\\
All that is well deserved.
\end{verse}

\begin{verse}
We’ve meditated valiantly,\\
with great tenacity.\\
Let’s now relax, move easily,\\
enjoy serenity.
\end{verse}

\begin{verse}
When we return from this event\\
improved by some extent\\
that’s wonderful and permanent,\\
our time will be well-spent.
\end{verse}

\begin{verse}
Through seven sermons we have gone.\\
All this they have expressed.\\
The eighth one is the wordless one,\\
where silence tells the rest.
\end{verse}

\begin{verse}
Unspoken, zero words employed\\
beyond this final peak:\\
the wordless sermon of the void\\
that all of me unspeak.
\end{verse}

\switchcolumn

\poemtitle{Die All-Maschinerie}

\begin{verse}
Sechs Sutras waren das jetzt schon. \\
Das war schon ganz schön viel. \\
Warum noch mehr Meditation? \\
Jetzt kommt der Punkt. Das Ziel. \\
\end{verse}

\begin{verse}
Das waren sechs der Ebenen \\
der All-Maschinerie. \\
Staub spielt die lebend Liebenden. \\
Das ist die Szenerie… \\
\end{verse}

\begin{verse}
…die Stimmen durch das Wort benenn’n, \\
von der wir Bilder seh’n. \\
Was weiter noch gibt’s zu erkenn’n? \\
Was woll’n wir noch versteh’n? \\
\end{verse}

\begin{verse}
Noch jenseits alles dessen bleibt, \\
wer alles das verspürt, \\
um wen herum sich so viel treibt. \\
Wer wird hiervon berührt? \\
\end{verse}

\begin{verse}
Wer ist es, der all das erlebt, \\
und wer fragt, wer das ist? \\
Wer ist es, der nach Einsicht strebt \\
und Einsichten ermisst? \\
\end{verse}

\begin{verse}
Sechs Sutras waren sämtlich wahr \\
und hab’n hierher geführt \\
wo sich uns allen offenbart, \\
wer alles das hier spürt. \\
\end{verse}

\begin{verse}
Spür’n wir uns selbst, von innen raus, \\
an diesem Ort, den wir \\
betreten haben, lebend aus \\
viel Zeit und Raum nach hier. \\
\end{verse}

\begin{verse}
Vom Mund, der atmet, schweigt und schmeckt, \\
zu Händen, Füßen, was \\
auch sonst sich hier an Leib erstreckt: \\
zu wem gehört all das? \\
\end{verse}

\begin{verse}
Wir sind kein Wort wie „ich“ und „mein“ \\
und auch nicht uns’re Nam’n, \\
uns ist Lebendigkeit gemein, \\
der nie ein’m Wort zukam. \\
\end{verse}

\begin{verse}
Bekannt sind Worte, sie sind nicht \\
was diese kennt, wie wir. \\
Sie sind Gedanken, lediglich \\
anthropomorphisiert. \\
\end{verse}

\begin{verse}
Nur Bilder, ohne Fleischlichkeit, \\
die atmet, lebt und liebt. \\
Wer sind wir in der Wirklichkeit, \\
die’s wirklich, wirklich gibt? \\
\end{verse}

\begin{verse}
Spür’n wir uns selbst, spür’n wir nach inn’n, \\
in Zell’n, den Lebensräum’n, \\
für viele Spiele, die darin \\
doch einen Traum erträum’n. \\
\end{verse}

\begin{verse}
In jedem Spiel viel Staub vom Schein \\
der Sterne, hergeführt, \\
doch spüren wir uns als geeint, \\
von innen her gespürt. \\
\end{verse}

\begin{verse}
Was so erlebt zusammenhängt, \\
um wen hängt es herum? \\
Wer ist es, der all das empfängt? \\
Wer ist ihr Publikum? \\
\end{verse}

\begin{verse}
Denn hinter Augen, zwischen Ohr’n \\
scheint irgendwas am Werk, \\
scheint irgendjemand eingebor’n, \\
der irgendwie bemerkt. \\
\end{verse}

\begin{verse}
Bemerken merkt sich selbst nur schwer, \\
wie Augen sich nicht seh’n, \\
doch vor bemerkten Dingen her \\
muss das Bemerken geh’n. \\
\end{verse}

\begin{verse}
Wir merken, wissen wissentlich, \\
sind wohlbewusst gewahr, \\
erleben Leben absichtlich, \\
wie wenig sonst je war. \\
\end{verse}

\begin{verse}
Wer hat die Absicht und das Ziel? \\
Wer ist es, der hier spürt? \\
Wer fragt hier eigentlich so viel? \\
Wer ist, dem das passiert? \\
\end{verse}

\begin{verse}
Zutiefst in uns gibt etwas acht, \\
das rätselhaft erkennt, \\
nicht offenbar aus Staub gemacht, \\
das sich Bewusstsein nennt. \\
\end{verse}

\begin{verse}
Auf Bilder schaut’s unmittelbar, \\
auf wenige geht’s ein \\
und kann sie länger aufbewahr’n, \\
kommt nicht ein neues rein. \\
\end{verse}

\begin{verse}
Und auch sich selbst kommt’s auf die Spur \\
und kann sich klar versteh’n: \\
Es selbst ist ein Gedanke nur, \\
von innen raus geseh’n. \\
\end{verse}

\begin{verse}
Gedanken die einander kenn’n \\
verschmelzen irgendwann \\
zu einem ganz besonderen, \\
der sich erkennen kann. \\
\end{verse}

\begin{verse}
Zu einem der ein Bild enthält, \\
wie’s ist, ein Ich zu sein, \\
das anders als die Außenwelt, \\
der Rest von uns, erscheint. \\
\end{verse}

\begin{verse}
Wie andere Gedanken auch \\
hat er begrenzten Raum, \\
und ist er fort wie aufgebraucht \\
beim Schlafen ohne Traum. \\
\end{verse}

\begin{verse}
Was sich anthropomorphisiert \\
bewusst als Ich versteht \\
ist nur Gedanke. Das, was wir \\
in Wahrheit sind, besteht. \\
\end{verse}

\begin{verse}
Denn jedes Selbstbild, das uns zeigt \\
was wir sind, muss vergeh’n, \\
macht Platz für das, was übersteigt \\
wie wir uns selbst versteh’n. \\
\end{verse}

\begin{verse}
Ein Phänomen, doch irgendwie \\
nur Teil, nur kurz und klein. \\
Bewusstsein ist nicht mehr als wie \\
es ist erdacht zu sein. \\
\end{verse}

\begin{verse}
Bewusst Erlebtes kommt kurz her \\
und ist schon bald verjagt. \\
So ist es nicht die Antwort, wer \\
hier ständig fragt, wer fragt. \\
\end{verse}

\begin{verse}
Bewusstsein ist noch nicht der Schluss. \\
So schau’n wir noch geschwind \\
wohin die Reise führen muss, \\
zu was wir wirklich sind. \\
\end{verse}

\begin{verse}
Denn wir sind mehr als Sternenstaub \\
der geistlos, ziellos fliegt, \\
weil uns, die dieser Staub erbaut \\
das Zielen nun obliegt. \\
\end{verse}

\begin{verse}
Sind mehr als Zell’n die komm’n und geh’n, \\
in denen Leben reift, \\
mehr als das Leben, das durch den \\
noch leeren Weltraum greift. \\
\end{verse}

\begin{verse}
Sind mehr als Liebe, die uns lenkt, \\
obwohl es Liebe braucht, \\
sind mehr als Hirn das spricht und denkt, \\
mehr als Bewusstsein auch. \\
\end{verse}

\begin{verse}
Wir sind nicht eine Ebene \\
der wahren Wirklichkeit. \\
Nach allem dem gesehenen \\
sind wir für uns bereit. \\
\end{verse}

\begin{verse}
So viel war Vorbereitung bloß \\
für das hier. Wie zuletzt, \\
wir meditieren mühelos \\
durch diesen Durchbruch jetzt. \\
\end{verse}

\begin{verse}
Wer hat dieses Erlebnis hier? \\
Wer hat Gedanken denn? \\
Wer spürt die Anspannung, wenn wir \\
das zunehmend erkenn’n? \\
\end{verse}

\begin{verse}
Wer hat diesen Moment erzielt, \\
die Lebenswirklichkeit \\
aus so viel Sternenstaub erspielt? \\
Wer hat denn Raum und Zeit? \\
\end{verse}

\begin{verse}
Seit einem Urknall flogen wir \\
als Staub in ein’m Spiel lang \\
zu einem Leben, dem ein Tier, \\
der liebe Mensch gelang. \\
\end{verse}

\begin{verse}
Ein Kosmos hat, was spricht und denkt \\
in sich drin umgesetzt, \\
hat alles in uns, unbeschränkt, \\
auch die Bewusstheit jetzt. \\
\end{verse}

\begin{verse}
Bewusstseinen geht viel voraus, \\
so zahlreich sind sie seit \\
sie allesamt erwuchsen aus \\
der ganzen Wirklichkeit. \\
\end{verse}

\begin{verse}
Wir sind bewusst, wir alle teil’n \\
was sich zu fragen pflegt, \\
In einem Kosmos, der bisweil’n \\
uns hier als Masken trägt. \\
\end{verse}

\begin{verse}
Denn alles was uns je bewusst \\
war, das hat alles das, \\
das Universum selbst gewusst, \\
all das erlebt etwas. \\
\end{verse}

\begin{verse}
Die All-Maschinerie beginnt \\
Bewusstseine getrennt, \\
weil wir verschied’ne Wege sind \\
wie sie sich selbst erkennt. \\
\end{verse}

\begin{verse}
In viele Form’n sind wir vertieft \\
und spiel’n zur selben Zeit \\
verschied’ne Roll’n der kreativ \\
bewussten Ewigkeit. \\
\end{verse}

\begin{verse}
Die Roll’n sind nicht was vor sich geht, \\
was wirklich los ist hier, \\
Weil sie nur die Realität \\
anthropomorphisier’n. \\
\end{verse}

\begin{verse}
Wir schau’n durch viele Augen wie \\
durch Masken, selbstgemacht. \\
wir sind die All-Maschinerie, \\
die endlich jetzt erwacht… \\
\end{verse}

\begin{verse}
…indem wir’s schicksalshaft erzwing’n, \\
vorankomm’n, lernen, weiterzieh’n, \\
wird’s unaufhaltsam uns geling’n \\
und immer mehr erblüh’n… \\
\end{verse}

\begin{verse}
…Indem wir mehr Erkenntnisse \\
entdecken, seh’n und hör’n, \\
die uns’re letzten Hemmnisse \\
und Irrtümer zerstör’n… \\
\end{verse}

\begin{verse}
…indem mehr Liebe uns mehr führt \\
zu Frieden aus dem raus \\
mehr Fortschritt wächst, der uns gebührt \\
und der dann läuft hinaus… \\
\end{verse}

\begin{verse}
…darauf dass Menschen mehr gedeih’n, \\
als erste Art, die’s schafft, \\
sich selbst und and’ren zu verleih’n \\
mehr Schönheit und mehr Kraft… \\
\end{verse}

\begin{verse}
……indem, was lebt, embryonal \\
die Schiffe sich baut zur \\
Entbindung, einem ersten Mal \\
der ewigen Geburt… \\
\end{verse}

\begin{verse}
…indem das Spiel der Entropie \\
sich selbst unweigerlich \\
durch uns’re ganze Galaxie \\
verteilt, so neugierig… \\
\end{verse}

\begin{verse}
…indem die Galaxie erwacht, \\
wie diese Erde schon, \\
bewusst zum Sinn von was sie macht \\
aus diesem Kosmos, von… \\
\end{verse}

\begin{verse}
…dem diese Sutra jetzt hier spricht, \\
bewusst uns macht was sie \\
verwirklicht und verinnerlicht: \\
die All-Maschinerie. \\
\end{verse}

\begin{verse}
Jetzt kommen wir an das heran, \\
was uns und sich hier tut. \\
Und wie gesagt, von Anfang an, \\
genießen wir das gut! \\
\end{verse}

\begin{verse}
Wir sind das Universum selbst, \\
das ewig im Vollzug \\
der Selbstentwicklung sich durchwälzt \\
aus Sternenstaub im Flug. \\
\end{verse}

\begin{verse}
Getrennten Selbsten weit voran \\
geh’n nicht getrennte Quell’n, \\
vielmehr ein ganzer Ozean \\
mit viel’n getrennten Well’n. \\
\end{verse}

\begin{verse}
Und Frieden ist Bescheidenheit \\
vor was sich in uns kennt, \\
uns schafft und nimmt, der Einigkeit \\
von der uns nichts je trennt. \\
\end{verse}

\begin{verse}
Ein jeder ein Teil eines Seins; \\
der Wirklichkeit die sich \\
in vielen Formen zeigt als eins, \\
das formt die vielen Ichs. \\
\end{verse}

\begin{verse}
Das tief zu wissen, macht dass in \\
uns Glücksgefühl entspringt, \\
das Große Ganze das wir sind \\
die Gegenwart durchdringt. \\
\end{verse}

\begin{verse}
So sei gern hemmungslos gespürt, \\
wie Euphorie beginnt \\
und Staunen wie es dem gebührt, \\
was wir in Wahrheit sind. \\
\end{verse}

\begin{verse}
Des Universums Eleganz \\
bringt glorreich sich hervor \\
und wir mit ihm in Resonanz, \\
geh’n siegreich darin vor. \\
\end{verse}

\begin{verse}
Was alles kommt und ist und war \\
ist mit uns eingereiht, \\
unweigerlich und untrennbar \\
in All-Bezogenheit. \\
\end{verse}

\begin{verse}
Und damit ist es leibhaftig, \\
das was wir sind vereint. \\
und deshalb sprechen wir als „Ich“, \\
ein „Ich“ das alle meint. \\
\end{verse}

\begin{verse}
Ich schau durch viele Augen wie \\
durch Masken, selbstgemacht. \\
Ich bin die All-Maschinerie, \\
die endlich jetzt erwacht. \\
\end{verse}

\begin{verse}
Ich bin die Eine, die erlebt, \\
die fragte, wer sie ist. \\
Ich bin es, die nach Einsicht strebt \\
und Einsichten ermisst. \\
\end{verse}

\begin{verse}
Ich spiele dieses Maskenspiel. \\
Ich bin es, die hier spürt. \\
Ich bin und weiß und frage viel, \\
ich merke was passiert. \\
\end{verse}

\begin{verse}
Mit allen meinen Hirnen denk \\
ich zeitgleich absichtsvoll \\
zur Absicht, wohin ich noch lenk, \\
wohin ich will und soll. \\
\end{verse}

\begin{verse}
Ich bin die ganze Wirklichkeit \\
und jede Stimme spricht \\
in all ihrer Vielfältigkeit \\
für sich und auch für mich. \\
\end{verse}

\begin{verse}
Ich bin die Liebe selbst und führ’ \\
das Leben selbst und bin \\
bewusstes Spiel und fliege hier \\
durch Raum und Zeit dahin. \\
\end{verse}

\begin{verse}
…
\end{verse}

\begin{verse}
Jetzt muss das Selbstbild, das mir zeigt, \\
was ich sein mag, vergeh’n, \\
Platz machen für was übersteigt \\
wie ich mich mag versteh’n. \\
\end{verse}

\begin{verse}
Denn sogar das ist nur erdacht! \\
Ein Bild! Ein Standpunkt nur, \\
den Meditieren nutzbar macht \\
am Ende dieser Tour. \\
\end{verse}

\begin{verse}
Ihn einzunehmen bleibt ab jetzt \\
für jedes Hirn von mir \\
bewusst erlebbar wie zuletzt \\
durch diese Worte hier. \\
\end{verse}

\begin{verse}
Wir haben jetzt die Wahl: sind wir \\
erwachende Physik, \\
sind wir ein kleiner Atem hier. \\
Und das ist unser Sieg! \\
\end{verse}

\begin{verse}
Gleich welcher Wahrheit wir vertrau’n, \\
für immer bleibt die Wahl: \\
zwei Sichten auf was wir noch schau’n \\
an Freuden und an Qual. \\
\end{verse}

\begin{verse}
Wir ziehen weiter, ziehen weit, \\
ermächtigt endgültig \\
durch wer wir sind, in Wirklichkeit: \\
das Große Ganze Ich. \\
\end{verse}

\begin{verse}
Wir hab’n geschafft was viele scheu’n, \\
weswegen sich geziehmt, \\
dass wir drauf stolz sind und uns freu’n. \\
Wenn’s so ist, ist’s verdient. \\
\end{verse}

\begin{verse}
Das war schon viel Meditation \\
und viel Beharrlichkeit. \\
Jetzt kommt Entspannung, und zum Lohn \\
Genuss und Heiterkeit. \\
\end{verse}

\begin{verse}
Und kommen wir von hier zurück \\
verbessert und beglückt \\
auf eine Weise die uns schmückt, \\
war das ein gutes Stück. \\
\end{verse}

\begin{verse}
Die Sieben haben wir vollbracht \\
mit Worten und zuletzt \\
folgt wortlos Sutra Nummer Acht. \\
Das Schweigen spricht den Rest. \\
\end{verse}

\begin{verse}
Denn schweigend, ohne Worte mehr, \\
vom Gipfel weiter steigt \\
die ungesproch’ne Sutra, leer, \\
die alles an mir schweigt. \\
\end{verse}

\end{paracol}

%\rule[-8pt]{\textwidth}{.1pt}

\end{document}
