\documentclass[10pt,a4paper]{article}
\usepackage{geometry}
\geometry{margin=2cm, left=0.5cm, right=0.5cm}
\usepackage{lmodern}
\usepackage{verse}
\usepackage[utf8]{inputenc}
\usepackage[program=/usr/bin/lilypond]{lyluatex}
\usepackage{svg}
\svgpath{svg/}

\usepackage{multicol}
\setlength\columnseprule{.4pt}

\begin{document}


\begin{titlepage}
\title{Sieben Säkulare Sutras}
\date{}
\maketitle
%    \centering
%    {\huge Seven Secular Sermons\par} % that \par is necessary for the following \vspace to take effect
%    \vspace{2.5cm}
%    \includesvg[scale=0.8]{7ss_hi_res_logo}
\end{titlepage}

\tableofcontents
\pagebreak

\begin{multicols}{2}

\poemtitle{Im Flug Durch Raum Und Zeit}

\begin{verse}
Genießen wir, was jetzt beginnt! \\
Es tut uns gut zu spür’n, \\
dass wir ein großes Ganzes sind \\
indem wir meditier’n. \\
\end{verse}

\begin{verse}
Die Worte führ’n in Vers und Reim \\
durch die Unendlichkeit \\
die uns umgibt als unser Heim \\
im Flug durch Raum und Zeit. \\
\end{verse}

\begin{verse}
Als Erstes spür’n wir mühelos \\
in uns’re Körper rein. \\
Wir brauchen nichts zu tun als bloß \\
im Hier und Jetzt zu sein. \\
\end{verse}

\begin{verse}
Wir spüren Füße, Beine, Bauch, \\
die Hände und den Kopf \\
und alles sonst wo immer auch \\
derselbe Pulsschlag klopft. \\
\end{verse}

\begin{verse}
Durch alles in uns strömt das Blut \\
mit Sauerstoff als Fracht \\
und Treibstoff für die Lebensglut \\
in der der Geist erwacht. \\
\end{verse}

\begin{verse}
Darf uns’re Atmung ruhig gehn, \\
wird unser Denken klar, \\
macht uns präsenter, zeigt uns den \\
Moment unmittelbar. \\
\end{verse}

\begin{verse}
So öffnet sich uns Stück um Stück \\
der unbeschwerte Flug \\
von Augenblick zu Augenblick \\
mit jedem Atemzug. \\
\end{verse}

\begin{verse}
Der Geist mit dem wir meditier’n, \\
ist dafür vorgeseh’n, \\
den Körper durch die Welt zu führ’n \\
und in ihr zu besteh’n. \\
\end{verse}

\begin{verse}
Drum wenden wir uns hin zur Welt, \\
zum großen Drumherum. \\
Wie ist es jetzt um uns bestellt? \\
Schau’n wir uns einmal um! \\
\end{verse}

\begin{verse}
Wer nicht ertaubt ist oder blind, \\
der hört sich um und sieht, \\
wie wir von dem umzingelt sind, \\
was rundherum geschieht. \\
\end{verse}

\begin{verse}
Der Ort der uns jetzt hier umringt \\
ist ebenso real \\
wie Atemluft die uns durchdringt. \\
Das spür’n wir ebenfalls. \\
\end{verse}

\begin{verse}
Die Atemluft, die wir uns teil’n, \\
die in und um uns fließt, \\
ist Teil des Hier in dem wir weil’n \\
und das uns jetzt umschließt. \\
\end{verse}

\begin{verse}
Wir teilen auch den Untergrund \\
der alles hier erhält, \\
was jetzt hier atmet rings um uns, \\
als Kinder dieser Welt. \\
\end{verse}

\begin{verse}
Doch was die Wahrnehmung entdeckt, \\
ist kaum der Rede wert. \\
Das Hier, das sich um uns erstreckt, \\
umfasst noch sehr viel mehr. \\
\end{verse}

\begin{verse}
In Nord und Süd, in Ost und West, \\
am Horizont beginnt \\
die ausgedehnte Wölbung des \\
Planeten wo wir sind. \\
\end{verse}

\begin{verse}
Was uns bekannt ist von der Welt \\
ist Oberflächlichkeit. \\
Darunter dreht sich stumm der Fels \\
im Flug durch Raum und Zeit. \\
\end{verse}

\begin{verse}
Mit klarem, off’nem, weitem Geist \\
ist das ganz leicht zu spür’n. \\
Wenn Vorstellung die Wege weist, \\
kann sie noch weiter führ’n. \\
\end{verse}

\begin{verse}
Nach links, nach rechts, nach unten, hoch, \\
egal wohin wir schau’n, \\
folgt nur ein wenig ferner noch \\
der leere Weltenraum. \\
\end{verse}

\begin{verse}
Viel größer als das Himmelszelt \\
und schwärzer als die Nacht, \\
doch schwer zu sehen: was uns enthält \\
ist nicht zum Seh’n gemacht. \\
\end{verse}

\begin{verse}
Es überwältigt mühelos \\
die Vorstellung, wie weit \\
das Universum sich grandios \\
erstreckt in Ewigkeit. \\
\end{verse}

\begin{verse}
Um aller Menschen Reich in dem \\
wir werden und vergeh’n \\
kreist unbeirrbar das System \\
der Sonne souverän. \\
\end{verse}

\begin{verse}
Für was uns freut und uns missfällt \\
bleibt das dort draußen taub. \\
Dem draußen ist die ganze Welt \\
ein blaues Klümpchen Staub. \\
\end{verse}

\begin{verse}
Die Erde ist nur ein Planet \\
von vielen zwischen kaum \\
zu zählendem Gestein und dreht \\
sich mit all dem im Raum. \\
\end{verse}

\begin{verse}
Das alles sind nur Anhängsel, \\
nur Beiwerk das sich dreht \\
um was gigantisch, strahlend grell \\
in ihrem Zentrum steht. \\
\end{verse}

\begin{verse}
Sie wiegt ein Tausendfaches mehr \\
als alles was hier kreist \\
und schenkt das Licht rings um sich her \\
das unser Leben speist. \\
\end{verse}

\begin{verse}
Dreihunderttausendmal so schwer \\
wie uns’re Erde hier, \\
die unbedeutend, nebenher \\
nur profitiert von ihr. \\
\end{verse}

\begin{verse}
Aus Wasserstoff wird Licht, das heißt \\
Millionen Tonnen pro \\
Sekunde werden dort verheizt, \\
verschmelzen lichterloh. \\
\end{verse}

\begin{verse}
Sie bilden dabei Helium \\
und and’re Teilchen für \\
das essentielle Medium \\
von Leben wie uns hier. \\
\end{verse}

\begin{verse}
Zu nahe an der Sonne brennt \\
ihr Licht jedoch zu heiß \\
für Leben, das die Sonne kennt \\
und diese Wahrheit weiß. \\
\end{verse}

\begin{verse}
Und zu weit draußen bleibt’s zu kühl, \\
weil wo Chemie gefriert \\
der Moleküle freies Spiel \\
nie Fleisch und Geist kreiert. \\
\end{verse}

\begin{verse}
Und ferner noch, in tiefster Nacht, \\
kreist Eis und Fels so fern, \\
dass uns’rer Sonne große Pracht \\
nur aussieht wie ein Stern. \\
\end{verse}

\begin{verse}
Durch reines Glück ist uns’re Welt \\
nicht heiß und nicht gefror’n. \\
Sonst wär’ auch sie ein braches Feld \\
und niemand hier gebor’n. \\
\end{verse}

\begin{verse}
Die Erde hier ist unser Heim, \\
und jeder Mensch ihr Kind \\
und auch ein Kind des Sonnenscheins \\
dem wir entsprungen sind. \\
\end{verse}

\begin{verse}
Doch bleibt der Sonne Herrlichkeit \\
ein selt’nes Reich des Lichts \\
inmitten der Unendlichkeit \\
des abgrundtiefen Nichts. \\
\end{verse}

\begin{verse}
Von hier in alle Richtungen \\
erstrecken sich die schwer \\
beschreibbaren Entfernungen \\
des Weltraums um uns her. \\
\end{verse}

\begin{verse}
Der Weltraum ist vor allem leer, \\
ein finst’res kaltes Nichts, \\
durchzogen nur vom Hin und Her \\
des schwachen Sternenlichts. \\
\end{verse}

\begin{verse}
Natürlich gibt es Schwerkraft, die \\
von schweren Dingen rührt, \\
und dunkle Energie, die sie \\
weg voneinander führt. \\
\end{verse}

\begin{verse}
Doch nichts ist sichtbar, nichts zu hör’n, \\
zu riechen, schmecken, spür’n. \\
Wenn wir die Vorstellung beschwör’n, \\
so mag das zu nichts führ’n. \\
\end{verse}

\begin{verse}
Und dennoch fühl’n wir irgendwie \\
die Weite überall. \\
Verdeckt vom Sichtbaren ist sie \\
doch jetzt und hier real. \\
\end{verse}

\begin{verse}
Meditation erlaubt es dem \\
uns Menschen eig’nen Geist, \\
jetzt hier zu spür’n wie das System \\
der Sonne um uns kreist. \\
\end{verse}

\begin{verse}
Die Monde und Planeten dort \\
erscheinen unbewohnt, \\
die Erde noch der einz’ge Ort \\
auf dem das Leben thront. \\
\end{verse}

\begin{verse}
Doch alle Welten die im Raum \\
für Leben offen steh’n \\
zähl’n doch im großen Ganzen kaum. \\
Es gibt noch mehr zu seh’n! \\
\end{verse}

\begin{verse}
Die kleinen Punkte Sternenlicht \\
sind Sonnen allesamt, \\
voll Glut die über fernen nicht \\
erforschten Welten flammt. \\
\end{verse}

\begin{verse}
Für sie sind Menschenaugen bloß \\
zu unscharf und zu klein, \\
wodurch ein Stern stets ähnlich groß \\
und ähnlich hell erscheint. \\
\end{verse}

\begin{verse}
Das Teleskop zeigt uns den Glanz \\
der Sterne sonnenhell \\
und dass nur riesigste Distanz \\
die Dunkelheit erhält. \\
\end{verse}

\begin{verse}
Distanz ermisst den Raum der weit \\
um alle Sonnen fließt \\
und definiert die Örtlichkeit \\
die uns jetzt hier umschließt. \\
\end{verse}

\begin{verse}
Denn alle diese Sonnen, die \\
bei Nacht am Himmel stehn, \\
sind uns’re Heimatgalaxie \\
von innen raus geseh’n. \\
\end{verse}

\begin{verse}
Sie alle, hundert oder mehr \\
Milliarden wohl derzeit, \\
vereint die Schwerkraft um uns her \\
zu einer Einigkeit. \\
\end{verse}

\begin{verse}
Dies wissend spüren wir jetzt grad \\
wie Atem durch uns weht \\
in einem kolossalen Rad \\
aus Sonnen, das sich dreht. \\
\end{verse}

\begin{verse}
Und jede dieser Sonnen strahlt \\
voll Pracht und Herrlichkeit, \\
ganz klar und absolut real \\
im Flug durch Raum und Zeit. \\
\end{verse}

\begin{verse}
Die Sterne scher’n sich nicht darum, \\
was uns hier mag gescheh’n. \\
Sie werden unverändert stumm \\
um unsre Gräber steh’n… \\
\end{verse}

\begin{verse}
…sofern ein Stern nicht irgendwann \\
den Wasserstoff verzehrt \\
und seine Frist erreicht hat – dann \\
wird selbst ein Stern zerstört. \\
\end{verse}

\begin{verse}
Hat einer ganz besonders groß \\
besonders hell brilliert, \\
bekommt er ein besond’res Los: \\
er wächst und explodiert. \\
\end{verse}

\begin{verse}
Die Supernovas, wie wir sie \\
bezeichnen, überstrahl’n \\
so manche ganze Galaxie \\
als gleißendes Fanal. \\
\end{verse}

\begin{verse}
Die Wucht mit der sie explodier’n \\
muss alles um sie her \\
zerstören und sterilisier’n, \\
wo Leben möglich wär. \\
\end{verse}

\begin{verse}
Doch die Chemie die uns erlaubt \\
ist so herbeigebombt, \\
denn wir sind Supernova-Staub \\
der weiß woher er kommt. \\
\end{verse}

\begin{verse}
Die Sterne schmieden Teilchen, die \\
das Leben in uns braucht, \\
uns Kindern dieser Galaxie, \\
wie Welt und Sonne auch. \\
\end{verse}

\begin{verse}
Doch Supernovas sind recht rar. \\
Die Galaxie hat drei \\
im Durchschnitt in einhundert Jahr’n \\
und sie sind schnell vorbei. \\
\end{verse}

\begin{verse}
Doch viele Supernovas pro \\
Sekunde explodier’n \\
in allen Galaxien wo \\
wir jetzt hin eskalier’n! \\
\end{verse}

\begin{verse}
Millionen Male weiter fort \\
als alle Sterne hier \\
sind noch mehr Galaxien dort \\
und so real wie wir. \\
\end{verse}

\begin{verse}
Obwohl Milliarden Sonnen strahl’n \\
in uns’rer Galaxie, \\
ist aller Galaxien Zahl \\
weit größer noch als sie. \\
\end{verse}

\begin{verse}
In welche Richtung wir auch schau’n, \\
ganz weit dort draußen sind \\
Milliarden Galaxien im Raum \\
und wir hier mittendrin. \\
\end{verse}

\begin{verse}
Erstaunte Ehrfurcht macht sich breit, \\
wenn wir wahrhaft versteh’n, \\
dass um uns hier in Wirklichkeit \\
Trilliarden Welten steh’n. \\
\end{verse}

\begin{verse}
Und beinah alle sind von hier \\
entsetzlich weit entfernt, \\
von uns’rer Heimatgalaxie \\
und uns’rem Heimatstern. \\
\end{verse}

\begin{verse}
Von hier wo wir jetzt meditier’n \\
kann unser Geist uns weit \\
zu fernen Galaxien führ’n \\
im Flug durch Raum und Zeit. \\
\end{verse}

\begin{verse}
Und durch die Leere überall \\
wo Galaxien fehl’n. \\
Im größten Teil des Weltenalls \\
ist nicht ein Stern zu seh’n. \\
\end{verse}

\begin{verse}
Meditative Achtsamkeit \\
macht spürbar und erschließt \\
die wirkliche Unendlichkeit \\
die uns jetzt hier umschließt. \\
\end{verse}

\begin{verse}
Trotz weit versprengtem Sternenglanz \\
ist das Kontinuum \\
des Universums beinah ganz \\
totales Vakuum. \\
\end{verse}

\begin{verse}
So sind Atome rar gesäht. \\
Und doch sind in uns hier, \\
ins Fleisch wo dieser Atem weht, \\
Trilliarden kondensiert. \\
\end{verse}

\begin{verse}
Zwei Drittel der Atome in \\
uns sind vom Wasserstoff \\
und älter als die Sterne sind. \\
Vom Urknall sind sie noch. \\
\end{verse}

\begin{verse}
Schon seit dem Anbeginn der Zeit \\
ist jedes Teilchen hier \\
ganz unvorstellbar weit gereist \\
zu uns die meditier’n. \\
\end{verse}

\begin{verse}
Aus Vakuum und Sonnenwind \\
sind sie akkumuliert \\
und wir die sie geworden sind \\
begegnen uns jetzt hier. \\
\end{verse}

\begin{verse}
Als Speis und Trank und Atemluft \\
sind sie was uns kreiert \\
und bis ihr Weg sie weiter ruft \\
begegnen sie sich hier. \\
\end{verse}

\begin{verse}
Die Teilchen, die wir sind, sind weit \\
gereist und sie sind wir, \\
die Kinder der Unendlichkeit, \\
uralt und jetzt erst hier. \\
\end{verse}

\begin{verse}
Die Teilchen in uns trafen sich \\
auch früher schon manchmal \\
und treffen sich unweigerlich \\
in Zukunft auch nochmal. \\
\end{verse}

\begin{verse}
Ein endlos off’ner Himmelsschlund, \\
lässt alles hier gescheh’n, \\
ist das woher wir kommen und \\
auch das wohin wir geh’n. \\
\end{verse}

\begin{verse}
Aus Sternenstaub gemacht sind wir \\
und so ist in uns drin \\
der Kosmos in uns allen hier \\
wie wir in ihm drin sind. \\
\end{verse}

\begin{verse}
Und deshalb sind uns Raum und Zeit \\
gegeben, sie zu füll’n, \\
Geheimnisse der Wirklichkeit \\
derselben zu enthüll’n. \\
\end{verse}

\begin{verse}
Und so ist die Unendlichkeit \\
wahrhaftig zu berühr’n. \\
Sie ist jetzt hier und Wirklichkeit, \\
die Weite, die wir spür’n. \\
\end{verse}

\begin{verse}
Gleich welcher Wahrheit man vertraut, \\
für immer bleibt’s dabei: \\
für immer sind wir Sternenstaub \\
im Flug durch Raum und Zeit. \\
\end{verse}

\end{multicols}
\begin{multicols}{2}

\poemtitle{Von Entropie Gespielt}

\begin{verse}
Was haucht uns Staub das Leben ein? \\
Was bringt uns hier hervor? \\
Was schwingt der Teilchen stumpfes Sein \\
zu Unsereins empor? \\
\end{verse}

\begin{verse}
Die Teilchen selbst sind leblos seit \\
sie aus den Sternen fiel’n. \\
Was ihnen Lebenskraft verleiht \\
ist ihr Zusammenspiel. \\
\end{verse}

\begin{verse}
Ein jeder Mensch, ein jedes Tier, \\
die ganze Pflanzenwelt \\
ist wie ein Kunstwerk konstruiert, \\
das funktioniert und hält. \\
\end{verse}

\begin{verse}
Unfassbar große Kunst beginnt \\
mit Teilchen und erstellt \\
wovon wir hier umgeben sind: \\
die Wunder dieser Welt. \\
\end{verse}

\begin{verse}
Wir lernen mehr aus der Natur \\
als je aus einem Buch \\
und sie scheint mehr und mehr obskur, \\
je gründlicher man sucht. \\
\end{verse}

\begin{verse}
Die Oberflächen die man sieht, \\
ob Rinde oder Haut, \\
verbergen die Biologie \\
darin, die all das baut. \\
\end{verse}

\begin{verse}
Die Lungen atmen Sauerstoff \\
in unser Blut hinein. \\
Das Herz, das in uns allen klopft, \\
treibt den durch Mark und Bein. \\
\end{verse}

\begin{verse}
Und schau’n wir noch genauer, so \\
wird Atmung die wir spür’n \\
zu chemischen Prozessen, wo \\
sich Luft und Blut berühr’n. \\
\end{verse}

\begin{verse}
Wir sind höchst kunstvoll aufgebaut. \\
Organe greifen hier \\
präzise ineinander, aus \\
Geweben konstruiert. \\
\end{verse}

\begin{verse}
Bei diesen handelt es sich um \\
unzählbar viele Zell’n, \\
bei jeder davon wiederum \\
um viele Organell’n. \\
\end{verse}

\begin{verse}
In allem Leben weit und breit \\
steckt alles das und mehr: \\
verwirrende Vielschichtigkeit \\
in uns und um uns her. \\
\end{verse}

\begin{verse}
Die Sterne zeigt das Teleskop \\
uns hell und groß und schwer, \\
doch Leben unter’m Mikroskop \\
erstaunt uns noch viel mehr. \\
\end{verse}

\begin{verse}
Wir finden mehr Vielschichtigkeit \\
in einem Schmetterling \\
als in der Galaxie die weit \\
entfernt uns hier umringt. \\
\end{verse}

\begin{verse}
Und all dies Leben, das wir seh’n, \\
ist aufgebaut aus Zell’n. \\
Die hochkomplexen Zell’n besteh’n \\
aus Chemikalien. \\
\end{verse}

\begin{verse}
Genau betrachtet wirken sie \\
voll nanotechnischer \\
Molekularmaschinerie \\
gar noch fantastischer. \\
\end{verse}

\begin{verse}
Und noch genauer angeschaut \\
erweist sich alles hier \\
als atomarer Sternenstaub \\
der planlos kollidiert. \\
\end{verse}

\begin{verse}
Was wir zu sehen glauben ist \\
ein starrer Atlas für \\
das tiefe Land, das ständig fließt, \\
in allem um uns hier. \\
\end{verse}

\begin{verse}
Es fließt jetzt hier und uns so nah; \\
nichts könnte näher sein. \\
Das Fließen dieser Teilchen da \\
geht uns durch Mark und Bein. \\
\end{verse}

\begin{verse}
Das Chaos in dem Teilchenfluss \\
wird Entropie genannt, \\
aus der und trotz der kommen muss \\
was mit uns hier entstand. \\
\end{verse}

\begin{verse}
Sie nagt an jeglicher Struktur, \\
die deshalb wachsen muss \\
und der Zerstörung trotzend stur \\
sich fortpflanzen im Fluss. \\
\end{verse}

\begin{verse}
Sie wirft die Teilchen blind umher \\
und führt manchmal zum Spiel \\
von Molekül’n die temporär \\
bestehen, ohne Ziel. \\
\end{verse}

\begin{verse}
Zwei Wasser- und ein Sauerstoff \\
spiel’n Wasser, das erhitzt \\
die Wolken spielt, als Regen tropft \\
und Donner spielt und Blitz… \\
\end{verse}

\begin{verse}
…und bei gering’rer Entropie \\
zu Schneeflocken gefriert, \\
im trägen Spiel des Eises wie \\
im Schlaf kristallisiert. \\
\end{verse}

\begin{verse}
Wir spür’n jetzt hier wie unser Spiel, \\
wie unser eigner Leib, \\
verhältnismäßig warm sich fühlt \\
und will, dass das so bleibt. \\
\end{verse}

\begin{verse}
Die Sternenglut ließ dieser Staub \\
zurück im Flug nach hier, \\
durch Kälte die kein Spiel erlaubt. \\
In Wärme leben wir… \\
\end{verse}

\begin{verse}
…weil Hitze jede Form zerteilt \\
die zufällig entsteht \\
und Kälte kürt nicht jenen Teil \\
der wächst und weitergeht. \\
\end{verse}

\begin{verse}
In Wärme ist genau so viel \\
an dieser Entropie, \\
wie nötig für das Wunderspiel \\
organischer Chemie. \\
\end{verse}

\begin{verse}
In Wärme wie der unseren \\
sind Teilchen ruhelos, \\
begegnen anderen und wenn \\
die passen, geht es los. \\
\end{verse}

\begin{verse}
So kann mit Kohlenstoff speziell \\
sehr vieles reagier’n, \\
doch was die Reaktion erstellt \\
wird meist nicht weiterführ’n. \\
\end{verse}

\begin{verse}
Dennoch geht’s schnell und sei derweil \\
die Chance auch noch so klein, \\
wird manches was entsteht doch Teil \\
von neuen Spielen sein. \\
\end{verse}

\begin{verse}
Bekommt der Zufall sehr viel Zeit, \\
wird fast durch Schicksal schon \\
komplexere Chemie bereit \\
zur Kettenreaktion. \\
\end{verse}

\begin{verse}
Vor etwa 4 Milliarden Jahr’n \\
war das der Anbeginn \\
von Ketten zu der unfassbar’n \\
Chemie hier in uns drin. \\
\end{verse}

\begin{verse}
Die Kettenreaktion vereint \\
ein jedes Molekül \\
in sich und reiht sich manchmal ein \\
in ein noch größ’res Spiel. \\
\end{verse}

\begin{verse}
Und führt der Moleküle Fluß \\
zum Ausgangspunkt zurück, \\
entsteht ein Kreis, ein Zirkelschluss, \\
dem manchmal Wachstum glückt. \\
\end{verse}

\begin{verse}
Aus Ringelreih’n der Entropie \\
erwuchs verspielt und blind \\
der Anfang der Biologie \\
der wir entsprungen sind. \\
\end{verse}

\begin{verse}
Wir sind gebaut aus solchen Spiel’n, \\
die zyklisch weitergeh’n, \\
aus blind verspielten Molekül’n \\
die nichts von uns versteh’n. \\
\end{verse}

\begin{verse}
Jetzt hier in diesem Atemzug \\
spiel’n sie trilliardenfach \\
ein Wunderspiel das Kraft genug \\
für jede Zelle macht. \\
\end{verse}

\begin{verse}
Die Zelle bietet Spielraum für \\
was sie an Spiel’n enthält \\
und was einander bau’nd in ihr \\
die Zelle selbst erhält. \\
\end{verse}

\begin{verse}
Harmonisches Zusammenspiel \\
all dessen konstruiert \\
gemeinsam sich ein Domizil \\
das sich durch Teil’n kopiert. \\
\end{verse}

\begin{verse}
Zwar lernt kein Einzelmolekül, \\
doch Zellen, ihr Verbund, \\
erlernten Essen, Heilen, Fühl’n \\
und tun das jetzt in uns. \\
\end{verse}

\begin{verse}
In ihnen greifen Spiele in \\
einander, sind von Wert \\
für gegenseitigen Gewinn \\
der alle sie vermehrt. \\
\end{verse}

\begin{verse}
Das größte heißt die DNA \\
und sie enthält den Schatz \\
des Wissens jeder Zelle – da \\
hat jedes Spiel sein’n Platz. \\
\end{verse}

\begin{verse}
Die Codes dort drinnen generier’n \\
die Eiweiße aus den’n \\
wir wachsen, sie zu transportier’n. \\
Wir nenn’n ein’n Code ein Gen. \\
\end{verse}

\begin{verse}
Die Zelle sammelt Energie \\
und trotzt so dem Verfall \\
durch nimmermüde Entropie, \\
die tödlich nach ihr krallt. \\
\end{verse}

\begin{verse}
Und dabei hilft so manches Spiel. \\
Die grüne Pflanzenpracht \\
sind Zellen voller Chlorophyll, \\
das Licht zu Nahrung macht. \\
\end{verse}

\begin{verse}
Weil Zellen so gut funktionier’n \\
entdecken wir sie in \\
der Luft, auf allem um uns hier \\
und tief im Boden drin. \\
\end{verse}

\begin{verse}
Im Spiel verblieben sind nur die; \\
Verlierer scheiden aus. \\
Denn Leben in der Entropie \\
muss wachsen, sonst ist’s raus. \\
\end{verse}

\begin{verse}
Die Meisterschaft der Zellen hält \\
seit Jahrmilliarden an \\
und sie belebte diese Welt \\
die unsereins gewann. \\
\end{verse}

\begin{verse}
Einst blieben Zellen nur allein, \\
doch manche ließen sich \\
auf größ’re Lebensspiele ein \\
und spiel’n gemeinschaftlich. \\
\end{verse}

\begin{verse}
Im Spiel vereint erfanden sie \\
dann Lebensformen in \\
den Meeren, wo sie Energie \\
auf neue Art gewinn’n. \\
\end{verse}

\begin{verse}
Mehr Größe macht der Entropie \\
zu trotzen zwar komplex, \\
doch fand das Leben irgendwie \\
dafür die Lösung: Sex. \\
\end{verse}

\begin{verse}
Indem Sex Gene neu durchmischt, \\
erfindet er aktiv \\
ganz neuen Nachwuchs, dessen frisch \\
kreierte Form er prüft. \\
\end{verse}

\begin{verse}
Und Gene die erfolgreich in \\
dem Nachwuchs spiel’n, zieh’n weit \\
in ihm zu ihrer Zukunft hin \\
durch Entropie und Zeit. \\
\end{verse}

\begin{verse}
In uns, die hier jetzt atmen, drin \\
sind Gene ausgeprägt, \\
und wir, die sie ererbten, sind \\
was sie noch weiterträgt. \\
\end{verse}

\begin{verse}
Die Spielregel der Entropie \\
bleibt hart und gnadenlos, \\
zwingt Leben, das in ihr gedieh, \\
dass alles wachsen muss. \\
\end{verse}

\begin{verse}
Drum müssen Zell’n zusammenspiel’n \\
und sich organisier’n, \\
durch Nerven das Bemüh’n der Viel’n \\
als Ganzes dirigiere’n. \\
\end{verse}

\begin{verse}
Ein Ziel ist immer der Gewinn \\
von Lebensraum – den fand \\
manch seegebor’nes Leben in \\
der Luft und auf dem Land. \\
\end{verse}

\begin{verse}
Und so entstand die ganze Pracht \\
der Tier- und Pflanzenwelt, \\
die uns so maßlos dankbar macht \\
dass sie auch uns enthält. \\
\end{verse}

\begin{verse}
Und jetzt sind wir es, die all das \\
gestalten – das fing an \\
als uns’re Affenart zum Spaß \\
ein Kokelspiel begann. \\
\end{verse}

\begin{verse}
Das Kochen gab viel Energie \\
und uns’rer Art die Stirn \\
für uns als Nachwuchs mit wie nie \\
verspielten großen Hirn’n. \\
\end{verse}

\begin{verse}
Verspielte Hirne lassen uns \\
verstehen, wie die viel’n \\
Prozesse uns erspielen und \\
wir lernen, sie zu spiel’n. \\
\end{verse}

\begin{verse}
Dies Wissen hat uns anvertraut, \\
wie alles was geschieht, \\
in uns den Geistesraum erbaut \\
in dem dies Wissen blüht. \\
\end{verse}

\begin{verse}
In jedem Maßstab wieder seh’n \\
wir Spiele, die spontan \\
einander spielend fortbesteh’n \\
als folgten sie ein’m Plan. \\
\end{verse}

\begin{verse}
Doch ist kein Plan in dem Gescheh’n. \\
In jedem Maßstab ließ \\
nur Chaos manches Spiel besteh’n \\
aus dem das nächste sprießt. \\
\end{verse}

\begin{verse}
Und wir geh’n wieder Spiele ein, \\
die größer sind als wir, \\
wie Clan und Dorf und obendrein \\
auch Staaten wie der hier. \\
\end{verse}

\begin{verse}
Solch Spiele, Technik und Design \\
und Städte und Maschin’n \\
soll’n Menschen in ihn’n dienlich sein, \\
wie wir den Genen dien’n. \\
\end{verse}

\begin{verse}
Und letztlich sind wir alle ein \\
globales Riesenspiel \\
aus Spielen, doch all das hat kein \\
Konzept und auch kein Ziel. \\
\end{verse}

\begin{verse}
So wahr das ist, es ist zu groß. \\
Ein Hirn reicht nicht so weit. \\
Es kann nur staunen, fassungslos \\
vor dieser Herrlichkeit. \\
\end{verse}

\begin{verse}
Und so erfand das Hirn das Wort, \\
um Denken mitzuteil’n. \\
Das lässt Gedanken im Transport \\
von Hirn zu Hirnen eil’n. \\
\end{verse}

\begin{verse}
Gedanken teilend bilden wir \\
gemeinsam einen Geist, \\
der einen Wissensschatz formiert, \\
den uns’re Spezies weiß. \\
\end{verse}

\begin{verse}
Gedanken die zusammenspiel’n \\
und Wissen destillier’n, \\
erlauben uns, das Werk von Viel’n \\
gezielt zu dirigier’n. \\
\end{verse}

\begin{verse}
Und so wird Spielzeug umsetzbar, \\
das Fahr’n und Fliegen löst, \\
das uns auf Flammen reitend gar \\
den Himmel selbst durchstößt. \\
\end{verse}

\begin{verse}
Verspielte Menschenhirne könn’n \\
den Fakt, dass um uns hier \\
höchst unerforschte Sterne brenn’n, \\
nicht einfach ignorier’n. \\
\end{verse}

\begin{verse}
Und Entropie treibt uns nach vorn. \\
Wir müssen diffundier’n, \\
der Spielplatz Weltraum gibt uns Sporn \\
ihn ganz zu kultivier’n. \\
\end{verse}

\begin{verse}
Einst starten Schiffe die sich selbst \\
vermehren dort hinaus, \\
verbreiten sich von Welt zu Welt \\
und säen Leben aus. \\
\end{verse}

\begin{verse}
In kurzen Jahrmillionen nur \\
vervielfältigen sie \\
die Lebensräume der Natur \\
in uns’rer Galaxie. \\
\end{verse}

\begin{verse}
Doch niemand sonst scheint das zu tun. \\
Kein weit’res Leben scheint \\
zu wachsen dort – vielleicht bleibt uns, \\
das einzige zu sein. \\
\end{verse}

\begin{verse}
Wahrscheinlich aber treffen wir \\
dort Leben das aus ganz \\
obskuren Spiel’n gemacht um schier \\
entrückte Sterne tanzt. \\
\end{verse}

\begin{verse}
Und was wir finden darf mit spiel’n \\
in kosmischen noch nie \\
gespielten Spielen, großen, viel’n \\
neu’n Spiel’n der Entropie. \\
\end{verse}

\begin{verse}
Auf allen Welten um uns hier \\
soll’n Geist und Leben sein. \\
So viele Spiele bringen wir \\
dem ganzen Staub dort bei. \\
\end{verse}

\end{multicols}
\begin{multicols}{2}

\poemtitle{Unsereins}

\begin{verse}
Inmitten der Unendlichkeit \\
sind wir und wer jetzt lebt \\
beschenkt mit einer Seltenheit \\
die von der Erde strebt. \\
\end{verse}

\begin{verse}
Es gibt die Entropie, die ganz \\
viel spielt, und jetzt nun gilt’s \\
zu seh’n wie alles das im Tanz \\
zu ein’m Geschenk verschmilzt. \\
\end{verse}

\begin{verse}
Geschenktes Leben inkarniert \\
in jedem, der’s empfing \\
und jetzt hier atmend meditiert, \\
von Staubwelten umringt. \\
\end{verse}

\begin{verse}
Indem uns die Meditation \\
in das was ist versenkt, \\
enthüllt sie was uns innewohnt, \\
uns und die Welt beschenkt. \\
\end{verse}

\begin{verse}
Die große Mannigfaltigkeit \\
der Spiele in uns treibt \\
voran des Leibes Einigkeit \\
die sie sich einverleibt. \\
\end{verse}

\begin{verse}
In uns folgt jeweils ein System \\
von spielenden Akteur’n \\
gemeinsam einem Ablauf dem \\
sie alle angehör’n. \\
\end{verse}

\begin{verse}
Er läuft allzeit synchron und frei \\
durch jedes Glied zugleich, \\
durch jede Zelle uns’res drei- \\
dimensionalen Fleischs. \\
\end{verse}

\begin{verse}
Der Lebenslauf durchläuft die Zeit, \\
die Sammlung aller schon \\
durchlauf’nen Stunden, aufgereiht \\
als vierte Dimension. \\
\end{verse}

\begin{verse}
Der Ablauf uns’rer Leben wiegt \\
uns stets von ein’m Moment \\
zum nächsten bis er einst versiegt. \\
Er ist nicht permanent. \\
\end{verse}

\begin{verse}
Das Jetzt das wir hier spür’n ist wie \\
ein Punkt im Ablauf all \\
der Leben die hier spür’n und die \\
sind vierdimensional. \\
\end{verse}

\begin{verse}
Beleuchten wir uns in der Zeit, \\
uns dehnend strömend in \\
Momenten der Vergangenheit, \\
auf zukünftige hin. \\
\end{verse}

\begin{verse}
So leicht wir atmen, kommt der Sinn: \\
wir alle war’n zuletzt \\
in vorigen Momenten drin, \\
die echt war’n so wie jetzt. \\
\end{verse}

\begin{verse}
So der Moment der Ankunft hier, \\
wo wir gerade sind. \\
Wir war’n dabei, den haben wir \\
in uns’ren Leben drin. \\
\end{verse}

\begin{verse}
Die Leben, die wir leben, sind \\
gefädelt durch den Tag \\
und führ’n zurück bis er beginnt \\
wie er begonn’n hab’n mag. \\
\end{verse}

\begin{verse}
Die Leben die wir führen führ’n \\
aus Jetzt und Hier heraus \\
und dehnen sich, ganz leicht zu spür’n, \\
in früh’re Jahre aus. \\
\end{verse}

\begin{verse}
Wo war’n wir denn vor sieben Jahr’n? \\
Der Strom des Lebens flicht \\
durch was wir taten, was wir sah’n, \\
erinnert oder nicht. \\
\end{verse}

\begin{verse}
Momente schließen uns nicht ein. \\
Wir haben sie erlebt. \\
Das Leben selbst, das sie vereint, \\
ist durch die Zeit gewebt. \\
\end{verse}

\begin{verse}
Es strömt in all’n Momenten drin \\
der Ablauf uns’res Seins. \\
All uns’re jüng’ren Formen sind \\
doch eins, sind Unsereins. \\
\end{verse}

\begin{verse}
Entspannt dem folgend, achtsam, leicht \\
könn’n wir noch weiter geh’n \\
als unsere Erinn’rung reicht, \\
und uns als Babys seh’n. \\
\end{verse}

\begin{verse}
Wir reisen durch was davor kam \\
und reichen mühelos \\
bis wo und wann wir Form annahm’n \\
in uns’rer Mütter Schoß. \\
\end{verse}

\begin{verse}
Jetzt atmen wir aus eig’ner Kraft. \\
Doch wurden wir dereinst \\
beatmet: eine Schwangerschaft. \\
Zwei atmeten als eins. \\
\end{verse}

\begin{verse}
Stromaufwärts bis zum Chromosom, \\
zur Zeugung, wo sich zeigt \\
wie wir entsteh’n, indem der Strom \\
von Eltern sich verzweigt. \\
\end{verse}

\begin{verse}
Der Lebensstrom, der durch uns geht, \\
von dem der Eltern aus, \\
entsprießt aus Kontinuität, \\
dem Einzelnen voraus. \\
\end{verse}

\begin{verse}
Und wer von uns Geschwister hat, \\
weiß dass dort Leben fließt, \\
wie Knospen, wie ein and’res Blatt \\
demselben Zweig entsprießt. \\
\end{verse}

\begin{verse}
Geschwister sind getrennt, doch kaum \\
verhüllt macht dieses Band \\
sie auch zu Zweigen eines Baums, \\
zu Fingern einer Hand. \\
\end{verse}

\begin{verse}
Und das ist lang noch nicht der Grund \\
der Ahnenlinien, \\
durch Mütter uns’rer Mütter und \\
die Großfamilien. \\
\end{verse}

\begin{verse}
Sie alle hatten Eltern, teils \\
mit mehr als einem Kind, \\
entfernt Verwandte die jeweils \\
jetzt weit dort draußen sind. \\
\end{verse}

\begin{verse}
Dass jemand zur Familie und \\
zu Unsereins gehört, \\
wird manchmal als Gefühl profund \\
verbunden seins verspürt. \\
\end{verse}

\begin{verse}
Stammbäume reichen aber weit. \\
Auch uns’re eigenen \\
sind älter und weit mehr verzweigt \\
als wir bewusst erkenn’n. \\
\end{verse}

\begin{verse}
Wie wir sind and’re Zweige Bahn’n \\
des Lebens, die vor Jahr’n \\
entsprossen uns’ren eig’nen Ahn’n \\
die sie dereinst gebar’n. \\
\end{verse}

\begin{verse}
So viele Menschen sind verwandt \\
mit jedem von uns hier, \\
die meisten fern und unbekannt, \\
doch so real wie wir. \\
\end{verse}

\begin{verse}
Wir sind verbunden durch die Zeit \\
als vierte Dimension \\
zu Stammbäumen als Wirklichkeit \\
statt bloßer Abstraktion. \\
\end{verse}

\begin{verse}
Zu Vorfahr’n von uns allen hier \\
zählt jeder Menschenschlag, \\
vom Sklaven bis zum Kavalier, \\
mal Jäger, mal gejagt. \\
\end{verse}

\begin{verse}
Denn aller Menschen Herkunft reicht \\
durch Zeitalter und der’n \\
vergess’ne Reiche an vielleicht \\
noch unerforschten Meer’n. \\
\end{verse}

\begin{verse}
Letztendlich mündet jede Bahn \\
von uns zurück hinein \\
in allerseits geteilte Ahn’n \\
von uns, von unsereins. \\
\end{verse}

\begin{verse}
Verwandt sind wir, unzweifelhaft. \\
Wir sind die nur einstweil’n \\
entfremdete Nachkommenschaft \\
der Vorfahr’n die wir teil’n. \\
\end{verse}

\begin{verse}
Die sind der Grund, dass alle die \\
wir selbst in all den Jahr’n \\
getroffen haben, irgendwie \\
Verwandte von uns war’n. \\
\end{verse}

\begin{verse}
Die selben Vorfahr’n schufen uns \\
und ihre fingen schon \\
mit Kochen und manch and’rer Kunst \\
die Welt an, wo wir wohn’n. \\
\end{verse}

\begin{verse}
Die Affen, denen wir entstamm’n, \\
hab’n vieles nicht geseh’n, \\
doch sie sind wie es zu uns kam, \\
warum wir aufrecht geh’n. \\
\end{verse}

\begin{verse}
Geschwister mancher Ahnen ging’n \\
und wuchsen anders auf, \\
zu and’ren Affen, allerdings: \\
Verwandte von uns auch. \\
\end{verse}

\begin{verse}
Ein Erbe teilend gingen wir \\
gemeinsam einst hervor \\
aus Eichhörnchen vergleichbar’n Tier’n \\
und Spitzmäusen davor. \\
\end{verse}

\begin{verse}
All das ist Frucht der Fruchtbarkeit \\
der Vierfüßler an Land. \\
Auch deren Vielgestaltigkeit \\
ist mit uns hier verwandt. \\
\end{verse}

\begin{verse}
Noch tiefer, noch ein ganzes Stück, \\
führt uns’re Lebensbahn. \\
Vierhundert Jahrmillion’n zurück \\
sind Fische uns’re Ahn’n. \\
\end{verse}

\begin{verse}
Das große Lebenszweigwerk webt \\
uns wunderbar exakt \\
zusammen mit all dem was lebt \\
und Wirbelsäulen hat. \\
\end{verse}

\begin{verse}
Noch früh’re Vorfahr’n, mit enorm \\
verschiedn’n Körpern stell’n \\
den Weg zurück zur ersten Form \\
des Lebens: einzeln’n Zell’n. \\
\end{verse}

\begin{verse}
Wir atmen, wie sie alle auch \\
geatmet haben, seit \\
die Zell’n den Sauerstoffgebrauch \\
vererbten durch die Zeit. \\
\end{verse}

\begin{verse}
Die gleiche DNA beweist \\
die gleiche Herkunft all \\
der Form’n in den’n das Leben reist, \\
auf Erden überall. \\
\end{verse}

\begin{verse}
Ein Stamm- und Lebensbaum, mit dem \\
das Leben selbst entstand, \\
als etwas lebloses trotzdem \\
was lebendes erfand. \\
\end{verse}

\begin{verse}
Als Kettenreaktion’n begann’n \\
einander zu umgarn’n, \\
fing’n Zyklen und das Leben an, \\
vor vier Milliarden Jahr’n. \\
\end{verse}

\begin{verse}
Die ausgedehnte Wölbung des \\
Planeten wo wir sind \\
bewuchs seitdem ein lebendes \\
Geäst, das sie umspinnt. \\
\end{verse}

\begin{verse}
Das Leben streckte Sprossen aus \\
in vielerlei Gestalt, \\
wie in Meditation durchaus \\
erspürbar widerhallt. \\
\end{verse}

\begin{verse}
Das Leben als Familie \\
eint jeden weit und breit \\
mit jeglicher Fossilie \\
aus der Vergangenheit. \\
\end{verse}

\begin{verse}
Begegnung lässt uns uns erkenn’n, \\
dass wir einander spür’n, \\
wie Zweige eines Baumes wenn \\
sie sich im Wind berühr’n. \\
\end{verse}

\begin{verse}
Im Vierdimensionalen sind \\
die Zweige alle eins, \\
das jetzt ein Teil zu seh’n beginnt \\
als ganzes Unsereins. \\
\end{verse}

\begin{verse}
Im dreidimensional’n Moment \\
erscheint’s getrennt, zerfällt’s. \\
Doch jenseits davon, ungetrennt, \\
sind wir das Leben selbst. \\
\end{verse}

\begin{verse}
Was lebend in uns atmet, reicht \\
entlang des Lebensbands \\
weit über uns hinaus und gleich \\
spür’n wir es einmal ganz. \\
\end{verse}

\begin{verse}
Des Lebens viele Zweige sind \\
in allen Meer’n zuhaus \\
auf Land und tief im Boden drin \\
und in die Luft hinaus. \\
\end{verse}

\begin{verse}
In ihnen lebt das Leben selbst. \\
In Tier- und Pflanzenschar \\
und immer neuen Formen stellt’s \\
sich unaufhörlich dar. \\
\end{verse}

\begin{verse}
Was jede Form auch immer tut, \\
das Leben selbst verleiht \\
die Kraft, auf der das Tun beruht \\
und jede Fähigkeit. \\
\end{verse}

\begin{verse}
Was jeder Schnabel, jeder Mund \\
auch immer diskutiert, \\
wird angestimmt vom Leben und \\
von Unsereins gehört. \\
\end{verse}

\begin{verse}
Aus allen uns’ren Augen sieht \\
das Leben selbst heraus. \\
Auf allen uns’ren Füßen zieht \\
es weiter und voraus. \\
\end{verse}

\begin{verse}
Mit allen uns’ren Blättern trinkt \\
es Sonnenenergie. \\
In alle uns’re Formen bringt \\
und transportiert es sie. \\
\end{verse}

\begin{verse}
Lebendig sein heißt deshalb stets \\
ein Teil von dem zu sein, \\
was unser Sterben überlebt; \\
dem Leben allgemein. \\
\end{verse}

\begin{verse}
Meditation, die uns jetzt hier \\
auf alles das besinnt, \\
erlaubt uns zu erspür’n dass wir \\
die Biosphäre sind. \\
\end{verse}

\begin{verse}
Wir selbst und jedes Wesen, dem \\
wir gegenüber schein’n, \\
sind Teil von etwas Größerem. \\
Als Leben sind wir eins. \\
\end{verse}

\begin{verse}
Sind alle Leben Teil des ein’n \\
kompletten Lebensbunds, \\
sind alle eins, sind wir allein. \\
Es gibt nur eins von uns. \\
\end{verse}

\begin{verse}
Zu jedem ander’n Exemplar \\
ist Weltraum zu durchquer’n. \\
Die Erde ist was uns gebar, \\
nicht wo wir hingehör’n. \\
\end{verse}

\begin{verse}
Auf Erden fing das Leben an, \\
doch geht die Welt entzwei. \\
Wie lang hier Leben leben kann \\
ist größtenteils vorbei. \\
\end{verse}

\begin{verse}
In einer Jahrmilliarde wird \\
die Sonne riesengroß, \\
wird alles Leben hier verdörrt, \\
vernichtet, chancenlos. \\
\end{verse}

\begin{verse}
Und das meint uns, denn wir sind gleich \\
mit was die Sonne dann \\
vernichtet, wenn des Lebens Reich \\
nicht weiter wachsen kann. \\
\end{verse}

\begin{verse}
Verbunden als das Unsereins, \\
das jeden hier durchquert, \\
spür’n wir das Leben allgemein \\
als uns und schützenswert. \\
\end{verse}

\begin{verse}
Als je ein Zweig am Lebensbaum, \\
der durch uns handeln kann, \\
geht sein Bedarf nach Lebensraum \\
uns ganz persönlich an. \\
\end{verse}

\begin{verse}
Das Leben muss sich weit hinaus \\
vermehr’n, sonst sterben wir, \\
stirbt Unsereins als Ganzes aus \\
in diesem Kosmos hier. \\
\end{verse}

\begin{verse}
Wie Vorfahren im Altertum \\
für uns’re Wirklichkeit, \\
gebührt uns Schande oder Ruhm \\
der Welt, die von uns bleibt. \\
\end{verse}

\begin{verse}
Vernichten wir uns, machen wir \\
die Erde, wo’s uns gab, \\
zum Mahnmal unseres Krepier’ns, \\
zum unbeweinten Grab. \\
\end{verse}

\begin{verse}
Erreichen wir Unendlichkeit, \\
pflanzt von der Erde aus \\
sich Leben fort, ist jetzt die Zeit \\
in seinem Elternhaus. \\
\end{verse}

\begin{verse}
Denn streu’n wir Leben weit hinaus, \\
so sind wir seine Quell’n, \\
geh’n wir sein’n Formen so voraus \\
wie Affen oder Zell’n. \\
\end{verse}

\begin{verse}
Für sie auf ihren Welten kann \\
die Erde hier nicht mehr \\
sein als verwunsch’nes Märchenland, \\
weit fort und lange her. \\
\end{verse}

\begin{verse}
Was wird dann das Vermächtnis sein \\
das ihnen sich erhält? \\
Was geht in ihr Gedächtnis ein \\
von uns’rer ersten Welt? \\
\end{verse}

\begin{verse}
Wohin auch immer sie noch zieh’n, \\
im Ursprung ihres Seins \\
erwartet sie das Wissen: sie \\
sind alle Unsereins. \\
\end{verse}

\end{multicols}
\begin{multicols}{2}

\poemtitle{Die Liebe, die uns Menschen führt}

\begin{verse}
Inmitten der Unendlichkeit \\
im Flug daheim sind wir, \\
und atmen voller Leichtigkeit \\
die Gegenwart jetzt hier. \\
\end{verse}

\begin{verse}
Wir atmen in dem Wissen wie \\
wir Staub vom Anbeginn \\
der Zeit, durch Spiel der Entropie, \\
inzwischen Menschen sind. \\
\end{verse}

\begin{verse}
Hier ist der Ort an dem wir nun \\
auch uns’re Menschlichkeit \\
erspüren während wir so ruh’n \\
in tiefer Heiterkeit. \\
\end{verse}

\begin{verse}
In diesem Atemzug ist klar \\
erkennbar ausgeprägt \\
was alles Vorbedingung war, \\
dass diese Brust sich regt. \\
\end{verse}

\begin{verse}
Als Staub sind wir durch Raum und Zeit \\
geflogen und erst hier, \\
erst in uns ist der Staub so weit, \\
zu spür’n und meditier’n. \\
\end{verse}

\begin{verse}
Der Atem wird von Spiel’n, die klein \\
und zahlreich sind, verbraucht \\
für den Prozess, der Unsereins \\
die Lebenskraft einhaucht. \\
\end{verse}

\begin{verse}
Durch alles was uns hergebracht \\
hat sind wir allesamt \\
jetzt angekommen in der Pracht \\
des Lebens insgesamt. \\
\end{verse}

\begin{verse}
Das Leben wuchert rings umher, \\
und sprießt, obwohl so alt, \\
doch immer jung in immer mehr \\
wildwüchsiger Gestalt. \\
\end{verse}

\begin{verse}
Aus allem Leben dieser Welt \\
erwuchs mit uns’rer Art \\
die Macht, die uns’re Welt bestellt, \\
besitzt, vielleicht bewahrt. \\
\end{verse}

\begin{verse}
Wir Menschen unterscheiden uns \\
von jedem and’ren Tier, \\
schon darin dass wir friedlich und \\
ganz sicher atmen hier. \\
\end{verse}

\begin{verse}
Wir Menschen kennen Sicherheit. \\
Was sonst durch’s Leben geht, \\
weiß kaum ob es die kurze Zeit \\
bis morgen überlebt. \\
\end{verse}

\begin{verse}
Schon wie wir atmen, ruhig, still, \\
beweist, dass wir vertrau’n, \\
dass niemand hier uns töten will. \\
Wir können darauf bau’n. \\
\end{verse}

\begin{verse}
Wir Menschen in der wilden Welt \\
die uns weithin umgibt, \\
hab’n Friedensorte hergestellt, \\
wie’s uns’rer Art beliebt. \\
\end{verse}

\begin{verse}
In Nord und Süd, in Ost und West \\
sind wir daheim, viel mehr, \\
als was die Menschheit leben lässt \\
an Arten um uns her. \\
\end{verse}

\begin{verse}
Dies Anderssein uns zu erklär’n, \\
was and’rem Leben fehlt, \\
hab’n wir dereinst geglaubt, wir wär’n \\
ganz and’rer Art: beseelt. \\
\end{verse}

\begin{verse}
Obwohl wir nur aus Staub besteh’n, \\
gibt’s eine Eigenheit, \\
die uns nur führt, die wir nur seh’n \\
in uns’rer Menschlichkeit. \\
\end{verse}

\begin{verse}
Was uns’re Menschheit menschlich macht, \\
einmalig, radikal, \\
das wurde einst hervorgebracht \\
im alten Afrika. \\
\end{verse}

\begin{verse}
Mit einer Affenart, die stand \\
und Steine warf und dann \\
die Sprache und das Lied erfand, \\
fing die Geschichte an. \\
\end{verse}

\begin{verse}
Wir wirkten nicht wie würdige \\
Herausforderer für \\
die Löwen als die Könige \\
in ihrem Jagdrevier. \\
\end{verse}

\begin{verse}
Dass uns’re Art sie doch bezwingt, \\
das konnte nur gescheh’n \\
durch neu entwickelten Instinkt, \\
einander beizusteh’n. \\
\end{verse}

\begin{verse}
Wir waren weder zahlreich noch \\
robust, nicht stark, nicht schnell. \\
Zusammenarbeit half jedoch, \\
im Kampf und generell. \\
\end{verse}

\begin{verse}
Ein Hagel vieler Steine trieb \\
selbst Löwen von uns weg. \\
Ein Schrei aus vielen Kehlen gibt \\
selbst Tigern einen Schreck. \\
\end{verse}

\begin{verse}
Auch and’re Tiere, Pflanzen auch, \\
agieren mal vereint, \\
doch nur bei Menschen ist es Brauch, \\
da wählerisch zu sein. \\
\end{verse}

\begin{verse}
Wir lernten zu beurteil’n, wer \\
uns wohlgesonnen scheint, \\
mit wem wir gern verbündet wär’n, \\
verbindlich und vereint. \\
\end{verse}

\begin{verse}
Wir mussten lernen, gut zu seh’n, \\
wieviel Vertrauen mag \\
gerecht sein gegenüber wem. \\
Wer irrte, unterlag. \\
\end{verse}

\begin{verse}
Viel mehr als Affen mussten wir \\
verdienen das Vertrau’n \\
von Artgenossen, weil sie Gier \\
und bösen Will’n durchschau’n. \\
\end{verse}

\begin{verse}
Das tat im Kampf und zusätzlich \\
auch auf der Jagd uns gut. \\
Die Neigung dazu prägte sich \\
uns ein in Fleisch und Blut. \\
\end{verse}

\begin{verse}
So wuchs der Herzenswunsch in uns, \\
Vertrauen wert zu sein, \\
dass uns einander echte Gunst \\
verdientermaßen eint. \\
\end{verse}

\begin{verse}
Es ist ein Wunsch nach Liebe von \\
uns Nächsten, für den wir \\
im Dienst gemeinsamer Mission \\
vertraut kooperier’n. \\
\end{verse}

\begin{verse}
Dass so verbunden sein viel Kraft \\
und Glück hervorbringt, zeigt \\
wie Bruderschaft und Schwesternschaft \\
Verwandtschaft übersteigt. \\
\end{verse}

\begin{verse}
Die Macht darin ist so groß, dass \\
es keinen passenden \\
Begriff gibt, der sie ganz umfasst. \\
Wir woll’n sie Liebe nenn’n. \\
\end{verse}

\begin{verse}
Sie ist auch zwischen uns jetzt hier \\
in dem Instinkt zu seh’n, \\
der uns natürlich motiviert, \\
einander beizusteh’n. \\
\end{verse}

\begin{verse}
Wir dürfen uns jetzt, wenn wir woll’n, \\
erlauben, diese Macht \\
zu spüren: alt, verheißungsvoll, \\
ganz echt und voller Pracht. \\
\end{verse}

\begin{verse}
Die Liebe macht uns Menschen aus. \\
Im Wildwuchs der Natur \\
sind wir allein aus ihr heraus \\
der Eintracht auf der Spur. \\
\end{verse}

\begin{verse}
Die Liebe schenkt uns Neugier und \\
uns-zeigen-wollen, weil \\
sie uns den Herzenswunsch gibt, uns \\
einander mitzuteil’n. \\
\end{verse}

\begin{verse}
Nicht nur Verbündete half das \\
uns finden; in Details \\
des Denkens Anderer steckt was \\
der And’re anders weiß. \\
\end{verse}

\begin{verse}
Das half uns, unser Wissen auch \\
zu teilen: wie was geht \\
und Feuer, Worte und Gebrauch \\
von nützlichem Gerät. \\
\end{verse}

\begin{verse}
Der Keil, der Speer, der Wasserschlauch, \\
die Trommel gaben Macht \\
den ersten Menschen. Das hat auch \\
uns hier hervorgebracht. \\
\end{verse}

\begin{verse}
Aus Kinderliebe wuchs der Drang, \\
zu teil’n den Anteil an \\
dem Wissen von uns heute lang \\
vorausgegang’nen Ahn’n. \\
\end{verse}

\begin{verse}
Solch Wissen wie Kooperier’n \\
in immer bess’rer Form \\
auf Regeln fußt, die wir kreier’n: \\
Gesetz, Vertrag und Norm. \\
\end{verse}

\begin{verse}
Die Sehnsucht, einmütig zu sein, \\
dass uns Alleinsein schmerzt, \\
gehört so tief in uns hinein \\
wie unser eig’nes Herz. \\
\end{verse}

\begin{verse}
Aus herzensguter Menschlichkeit \\
ist schwierig einzuseh’n, \\
dass Frieden und Gerechtigkeit \\
allein durch uns entsteh’n. \\
\end{verse}

\begin{verse}
Die Menschenwürde ist nur echt, \\
wenn wir sie würdigen. \\
Es gibt auf Erden nur das Recht, \\
das wir ermöglichen. \\
\end{verse}

\begin{verse}
Wir sind die einz’ge Tierart, die \\
sich selbst im Lauf der Zeit \\
das Recht gab, die Justiz sowie \\
ihr Ziel Gerechtigkeit. \\
\end{verse}

\begin{verse}
Natur heißt Recht der Stärkeren, \\
die Schwächere zerfetz’n. \\
Wir schreiben jetzt an besseren, \\
durchdachteren Gesetz’n. \\
\end{verse}

\begin{verse}
Wir haben lange vielerlei \\
Gesetze ausprobiert. \\
Die meisten hab’n zu Tyrannei, \\
zu Mord und Krieg geführt. \\
\end{verse}

\begin{verse}
Als Säugetier ganz eig’ner Art, \\
das Frieden intendiert, \\
sind wir die eine Raubtierart, \\
die sich domestiziert. \\
\end{verse}

\begin{verse}
Auf affenartiger Natur \\
erbau’n wir neues: Kunst \\
und einzig menschliche Kultur, \\
die weiter wächst durch uns. \\
\end{verse}

\begin{verse}
Wir tun viel neues, sind extrem \\
und scheitern wie verrückt. \\
So viel gelingt uns nicht! Trotzdem: \\
Ein Anfang ist geglückt. \\
\end{verse}

\begin{verse}
Aus alten Trieben, die uns stark \\
zu Aggression verführ’n, \\
entwickeln wir, dass wir am Markt \\
um Mehrwert konkurrier’n. \\
\end{verse}

\begin{verse}
Geh’n wir im Tausch von dies für das \\
Verbindlichkeiten ein \\
und ist auf diese auch Verlass, \\
entsteht Verbundensein. \\
\end{verse}

\begin{verse}
Die meisten Dinge hier sind weit \\
gereist und eingekauft. \\
Auch Handel baut auf Teamarbeit \\
in großem Maßstab auf. \\
\end{verse}

\begin{verse}
Aus Clan und Sippe führte schon \\
der Drang zur Mitarbeit \\
durch Stadt und Staat und Konfession \\
zum Menschenrecht weltweit. \\
\end{verse}

\begin{verse}
Vertrauen, das all das erlaubt, \\
hab’n Menschen mit viel Fleiß \\
jahrhundertelang aufgebaut. \\
Das ist was Menschsein heißt. \\
\end{verse}

\begin{verse}
Auf wieviel Arbeit Frieden folgt \\
und wie Gemeinschaft reift \\
durch Mitarbeit ist ein Erfolg, \\
den kein Tier sonst begreift. \\
\end{verse}

\begin{verse}
So vieles bleibt uns noch zu lern’n, \\
aus noch mehr Fehlern auch. \\
Wir sind noch weit von dem entfernt \\
wozu es Menschen braucht. \\
\end{verse}

\begin{verse}
So viele Arten sind lang her, \\
sind tot, vorbei, davon. \\
Doch die Geschichte unserer \\
hat g’rade erst begonn’n. \\
\end{verse}

\begin{verse}
Wir werden immer besser in \\
der Pflicht, die uns zufällt: \\
die Fragen zu beantworten \\
die kein Tier sonst sich stellt. \\
\end{verse}

\begin{verse}
Die Liebe, die das Erdenrund \\
uns gab führt weiter, denn \\
sie drängt uns auch, einander und \\
dadurch uns selbst zu kenn’n. \\
\end{verse}

\begin{verse}
Wir wissen, dass ein jedes Tier \\
vom selben Leben stammt, \\
doch scheinbar einzig sehen wir \\
das Leben insgesamt. \\
\end{verse}

\begin{verse}
Nur uns’re Art hat sich erkor’n, \\
zu sehen, was wir sind: \\
ein Teil von dem, was blind gebor’n \\
durch uns zu seh’n beginnt. \\
\end{verse}

\begin{verse}
Und folglich sind wir auserwählt. \\
Denn wir allein kreier’n \\
was allen and’ren Tieren fehlt, \\
zum Beispiel Meditier’n. \\
\end{verse}

\begin{verse}
Das Leben, das uns Menschen speist, \\
trägt uns vielleicht auch an, \\
dass es um noch mehr Sterne kreist, \\
weil kein Tier sonst das kann. \\
\end{verse}

\begin{verse}
Wir nehmen diese Tiere mit, \\
weil sie zu uns gehör’n, \\
gehör’n sie auf den Flammenritt \\
mit uns als Transporteur’n. \\
\end{verse}

\begin{verse}
So viele Spiele bringen wir \\
dem ganzen Staub dort bei, \\
säen Leben aus, weit fort von hier, \\
doch bleibt es nicht dabei. \\
\end{verse}

\begin{verse}
In uns allein ist Unsereins \\
sich selbst zu seh’n gelung’n \\
und das gibt uns vor allem eins: \\
Alleinverantwortung. \\
\end{verse}

\begin{verse}
Wir antworten mit was uns kürt \\
und uns die Wege weist: \\
der Liebe, die uns Menschen führt, \\
die diese Sutra preist. \\
\end{verse}

\begin{verse}
Den Welten uns’rer Nachkommen \\
vermachen wir durchaus \\
auch diese Liebe in uns, denn \\
sie macht uns Menschen aus. \\
\end{verse}

\begin{verse}
Wir schicken auf die Wanderschaft \\
des Lebens uns’rer Welt \\
auch die uns Menschen eig’ne Kraft \\
die uns zusammenhält. \\
\end{verse}

\begin{verse}
Es gibt ein Einssein tiefer noch \\
als Liebe und Vertrau’n. \\
Die Liebe hilft auch da jedoch, \\
hilft tief genug zu schau’n. \\
\end{verse}

\begin{verse}
Vom selben Staub, im selben Spiel, \\
lässt Liebe uns versteh’n, \\
wie dort wohin die Liebe zielt, \\
wir stets uns selber seh’n. \\
\end{verse}

\begin{verse}
Wer weiß, dass wir das Selbe sind, \\
kann tief in Liebe ruh’n, \\
wo wir einander wohlgesinnt \\
stets auch uns selbst gut tun. \\
\end{verse}

\begin{verse}
Das haben alle die gemeint, \\
für die die Liebe Licht \\
ist, das den rechten Weg bescheint. \\
Die Wahrheit ist so schlicht. \\
\end{verse}

\begin{verse}
Denn nur die Liebe führt so weit, \\
dass sie uns d’rauf besinnt, \\
dass mehr uns eint als Ähnlichkeit, \\
dass wir dasselbe sind. \\
\end{verse}

\begin{verse}
Nur Sternenstaub, den Entropie \\
im selben Spiel verknüpft \\
zur selben Biosphäre die \\
dieselbe Menschheit schöpft. \\
\end{verse}

\begin{verse}
Und wenn wir wollen, dürfen wir \\
die Kraft, die uns so führ’n \\
kann, eindringlich, im Jetzt, im Hier, \\
von ganzem Herzen spür’n. \\
\end{verse}

\begin{verse}
Die Liebe, die uns Menschen führt, \\
ist was jetzt hier enthüllt, \\
wie uns das Unsereins berührt, \\
wie kein Tier sonst das fühlt. \\
\end{verse}

\begin{verse}
Die Liebe, die uns Menschen führt, \\
gewinnt, gedeiht und glückt, \\
wenn sie sich mehr als nur gespürt \\
in uns’rem Tun ausdrückt. \\
\end{verse}

\begin{verse}
Die Liebe, die uns Menschen führt, \\
ist keine Wahlheimat, \\
und keinen Pfad gibt es zu ihr, \\
denn sie ist selbst der Pfad. \\
\end{verse}

\begin{verse}
Die Liebe, die uns Menschen führt, \\
führt zu was vor uns steht, \\
zu dem was sie heraufbeschwört, \\
wohin der Pfad noch geht. \\
\end{verse}

\end{multicols}
\begin{multicols}{2}

\poemtitle{Die Stimme hat das Wort}

\begin{verse}
Durch diese sieben Sutras führt \\
gereimte Poesie. \\
Das fünfte zeigt, woher das rührt \\
was hier passiert, und wie. \\
\end{verse}

\begin{verse}
Jetzt schau’n wir wie die Stimme leicht, \\
im Wort Gedanken nennt, \\
durch Sprache Einigkeit erreicht \\
wie kein Tier sonst sie kennt. \\
\end{verse}

\begin{verse}
Aus Staub kam’n Spiele einst in Fahrt \\
und alles Leben hier, \\
doch einzig eine Affenart \\
hat Worte dafür: wir. \\
\end{verse}

\begin{verse}
Klar, jedes Vög’lein, jedes Tier \\
kann leise oder schrill \\
an andere signalisier’n \\
was es verbreiten will. \\
\end{verse}

\begin{verse}
Wir alle kenn’n die Laute, die \\
durch Schwärme, Herden geh’n, \\
doch haben wir darin noch nie \\
Grammatiken geseh’n. \\
\end{verse}

\begin{verse}
Wir konnten alle Lieder hör’n \\
die jeder Vogel singt, \\
und dass, was Tiere bell’n und röhr’n \\
nur Ruf, nicht Sprachen sind. \\
\end{verse}

\begin{verse}
Die Luft, die Menschen ventilier’n \\
wird sonderbar geformt. \\
Grammatiken verkomplizier’n \\
die Worte ganz enorm. \\
\end{verse}

\begin{verse}
Man mag beharr’n, dass Wale doch \\
Gedanken teil’n von sich, \\
so debattieren bleibt dennoch \\
was kein Tier sonst bespricht. \\
\end{verse}

\begin{verse}
Auf Sprache jetzt zu fokussier’n \\
ist hier der rechte Ort; \\
darüber mal zu meditier’n. \\
Die Stimme hat das Wort. \\
\end{verse}

\begin{verse}
Ein Wort vor’m anderen enthält \\
Bedeutung anders als \\
wär’n diese andersrum gestellt \\
„Mal’s da“ ist nicht „damals“ \\
\end{verse}

\begin{verse}
„Checks schon mit“ heißt nicht „schon mit Checks“ \\
und auch nicht „mit Checks schon“ \\
und diese Tripel sind von sechs \\
nur drei Permutation’n. \\
\end{verse}

\begin{verse}
Mehr Worte steigern das sehr schnell. \\
Das wächst exponentiell, \\
wird überabzählbar, speziell \\
kombinatoriell. \\
\end{verse}

\begin{verse}
Zusammen machen Worte Sinn, \\
Verwirrung oder Spaß, \\
in Streitgespräch und Sutras in \\
solch Strophen und Versmaß. \\
\end{verse}

\begin{verse}
Mit Worten für was so entsteht, \\
wie „Verben“, „Witz“ und „wahr“, \\
erschließen wir, was dadurch geht, \\
genauer – wunderbar! \\
\end{verse}

\begin{verse}
Und auch Gebärden sind nicht stumm. \\
Sie setzen Worte fort \\
und setzen uns’re Stimmen um. \\
Auch darin spricht das Wort. \\
\end{verse}

\begin{verse}
Die Neugier aufeinander aus \\
der Liebe die wir spür’n \\
erfüll’n wir, darf sie freiheraus \\
zu stimm’n’den Worten führ’n. \\
\end{verse}

\begin{verse}
Natürlich hab’n wir mehr im Kopf, \\
als uns’re Stimme pro \\
Sekunde sagen kann. Sie stopft \\
begrenzte Worte so. \\
\end{verse}

\begin{verse}
Die Wirklichkeit wird komprimiert \\
wenn wir kommunizier’n. \\
Komplikationen können wir \\
stets nur approximier’n. \\
\end{verse}

\begin{verse}
Das macht die Worte mitteilbar, \\
und kostet seinen Preis. \\
Subtilitäten gehen gar \\
verloren als Verschleiß. \\
\end{verse}

\begin{verse}
Bestmöglich fehlt Subtilität \\
nur wenig, manch Wort nimmt \\
kaum Abstand von Realität. \\
Wir nenn’n das, dass es stimmt. \\
\end{verse}

\begin{verse}
Im überabzählbaren doch \\
grammatischen Satz reist \\
pro Wort mehr Wissen, das auf noch \\
mehr Sinngehalt verweist \\
\end{verse}

\begin{verse}
Sind Worte dichter komprimiert \\
als die erlebte Welt, \\
so sind sie leichter kombiniert \\
in der Gedankenwelt. \\
\end{verse}

\begin{verse}
Sechs Worte sind die Zeile hier. \\
Und diese zweite vier. \\
Noch zwei in selbiger Manier, \\
macht diese Strophe hier. \\
\end{verse}

\begin{verse}
Die Strophen hier sind alle gleich, \\
im altmodischen Stil, \\
die Musikalität erreicht, \\
denn Takt und Reim tun viel. \\
\end{verse}

\begin{verse}
Die Stimme hat Sprachmelodie, \\
die mit dem Wort erklingt, \\
sich aufschwingt in die Harmonie \\
von Liedern die sie singt. \\
\end{verse}

\begin{verse}
Wer Wort ergreift, kann meist sodenn \\
ergreifen den Akkord \\
und singen, was wir Lieder nenn’n \\
und anstimm’n mit dem Wort. \\
\end{verse}

\begin{verse}
Ein Wort wie „Lied“ für vielerlei \\
erzeugt Vergleichbarkeit. \\
Ist der Gesang danach vorbei, \\
ist noch das Wort, was bleibt. \\
\end{verse}

\begin{verse}
Kein Wort ist wie worauf es je \\
an Dingen sich bezieht. \\
Das Wort für das ist die „Idee“, \\
ein Ding das man nicht sieht. \\
\end{verse}

\begin{verse}
Aus Worten, die erst Namen war’n \\
für Dinge die wir sah’n, \\
erwuchs bald Abstraktion heran. \\
Manch neues Spiel begann. \\
\end{verse}

\begin{verse}
Ideen von großer Dienlichkeit, \\
wie „eins“ und „zwei“ und „drei“, \\
war’n Macht und schenkten Machbarkeit \\
und noch mehr Denkerei. \\
\end{verse}

\begin{verse}
Ideen begegneten sich in \\
uns, weil sie durch uns geh’n, \\
und manchmal war das der Beginn \\
komplexerer Ideen. \\
\end{verse}

\begin{verse}
Aus der Vermessung uns’res Lands \\
entstand Geometrie. \\
Aus Tausch von dem, was angepflanzt, \\
entstand Ökonomie. \\
\end{verse}

\begin{verse}
Ideen die gut sind können uns \\
bereichern lebenslang. \\
Doch schlechte führ’n zu Irrtum und \\
verwirrtem Untergang. \\
\end{verse}

\begin{verse}
Manch ganzes Leben ward vertan. \\
In Wortspiel’n ohne Ziel \\
hing’n sie verloren Träumen an \\
und hofften viel zuviel. \\
\end{verse}

\begin{verse}
Ideen gestalten was beginnt \\
in uns’re Leben rein. \\
Damit wir nicht ihr Werkzeug sind, \\
soll’n sie die uns’ren sein. \\
\end{verse}

\begin{verse}
Wo immer man Ideen teilt, \\
entsteht durch sie ein Bund \\
von geistiger Verbundenheit \\
und Liebe ist ihr Grund. \\
\end{verse}

\begin{verse}
Durch Worte klär’n wir Streitigkeit. \\
Und hilft ein Richter sonst, \\
wär jegliche Gerichtsbarkeit \\
doch ohne Wort umsonst. \\
\end{verse}

\begin{verse}
Mit allen Worten laden wir \\
stets in Beziehungen \\
wie Spiele ein. Die Sutras hier \\
sind auch Einladungen. \\
\end{verse}

\begin{verse}
Doch Sprache ist nicht lediglich \\
die Stimme und das Wort, \\
zusamm’n mit ihnen bilden sich \\
auch die Gedanken fort. \\
\end{verse}

\begin{verse}
Das hilft uns besser zu versteh’n: \\
Wie Wort für Worte steht, \\
kann Denken durch Gedanken sehn \\
wie unser Denken geht. \\
\end{verse}

\begin{verse}
So kam es dann, dass der Verstand \\
Philosophie und nicht \\
zuletzt sich selbst in Köpfen fand \\
aus den’n die Stimme spricht. \\
\end{verse}

\begin{verse}
Die Ahnen damals horteten \\
Ideen von großem Wert, \\
die vorteilhaft antworteten, \\
verständlich und bewährt. \\
\end{verse}

\begin{verse}
Erreichen sie Einstmmigkeit \\
und Einigkeit, was nun \\
zu tun sei half Einhelligkeit \\
in ihr’m vereinten Tun. \\
\end{verse}

\begin{verse}
Was immer uns’ren Ahn’n gelang \\
gelang nur im Gespräch. \\
Bei Fehlern, die sie auch begang’n, \\
war’n die Gespräche schlecht. \\
\end{verse}

\begin{verse}
Wir schulden uns’ren Ahnen viel. \\
Sie haben viel getan \\
und Spuren davon sind subtil \\
in den Vokabular’n. \\
\end{verse}

\begin{verse}
Wir halten alles das in Ehr’n \\
und setzen ihr Werk fort \\
indem wir Worte weiter lehr’n. \\
Den Kindern gilt das Wort. \\
\end{verse}

\begin{verse}
Wir schenken Kindern Sprachen für \\
den Weg zum Schatz hinein, \\
zu Worten die sie weiterführ’n \\
in Schul’n und Büche’rein. \\
\end{verse}

\begin{verse}
Die Kinder mögen irgendwann \\
anstatt von irgendwem \\
der fort ist selbst bestimm’n und dann \\
für sie das Wort annehm’n. \\
\end{verse}

\begin{verse}
Mitsprache dient der Mitarbeit. \\
Kommt neue Arbeit her, \\
ist auch für neue Sprache Zeit, \\
erfinden wir noch mehr. \\
\end{verse}

\begin{verse}
Weil Worte immer ungenau \\
die Wahrheit nur skizzier’n, \\
lässt Sprache sich durch Neufaufbau \\
von Worten präzisier’n. \\
\end{verse}

\begin{verse}
So viel war uns’ren Ahnen nicht \\
bekannt von dieser Welt. \\
Woher kommt Regen? Sonnenlicht? \\
Kein Wort, das nicht entstelllt. \\
\end{verse}

\begin{verse}
Mit uns verglichen schein’n sie dumm, \\
doch klug war’n sie indem \\
sie schufen was jetzt wiederum \\
wir hier zum Schaffen neh’m. \\
\end{verse}

\begin{verse}
Mit immer größ’rem Wissen wuchs \\
ein großer Schatz heran. \\
Trophäen uns’res Siegeszugs. \\
Wir häufen sie noch an. \\
\end{verse}

\begin{verse}
Durch’s Strömen der Jahrhunderte \\
und immer mehr Ideen \\
war was uns Menschen wunderte, \\
allmählich klar zu seh’n. \\
\end{verse}

\begin{verse}
Wir sprachen weiter Tag und Nacht \\
und setzten fort was einst \\
gesagt war. Weitersprechen macht \\
aus all’n Gesprächen eins. \\
\end{verse}

\begin{verse}
An jeder Universität, \\
in Gilden, Schul’n und an \\
all’n Orten jeden Wissens geht \\
ein Weltgespräch voran. \\
\end{verse}

\begin{verse}
Durch alle Menschenstimm’n hindurch \\
sucht ein Gespräch nach sich \\
bewährenden Ideen, wodurch \\
es gleich ist, wer sie spricht. \\
\end{verse}

\begin{verse}
Wir sprechen darin mit, seitdem \\
man Sprechen uns gezeigt \\
hat und werd’n weiter Anteil nehm’n \\
bis uns’re Stimme schweigt. \\
\end{verse}

\begin{verse}
Ein Teil von uns’rem Weltgespräch \\
bezieht, bespricht, bedenkt \\
was wirklich ist und wahr und echt, \\
was stimmt, uneingeschränkt. \\
\end{verse}

\begin{verse}
Durch wachsende Gelehrsamheit \\
begriffen Menschen wie \\
wir hier sind, in Unendlichkeit, \\
aus Staub und Entropie. \\
\end{verse}

\begin{verse}
Durch Technik geh’n wir nicht zuletzt \\
noch mehr Gespräche ein. \\
Die Schrift, der Druck, das große Netz \\
verbinden Unsereins. \\
\end{verse}

\begin{verse}
Vergleichen unserer Ideen \\
zeigt: manche sind nicht wahr. \\
Wie Himmel, die wir klarer seh’n, \\
wo nie ein Engel war. \\
\end{verse}

\begin{verse}
Doch hab’n vereinbare Ideen \\
auch miteinander Platz \\
für gegenseitiges Versteh’n, \\
sind sie der größte Schatz. \\
\end{verse}

\begin{verse}
All das zu ahnen, irgendwie, \\
erinnert uns entfernt, \\
wie uns’re Stimmen Rollen spiel’n \\
in einem Ding, das lernt. \\
\end{verse}

\begin{verse}
Die Suche dessen endet im \\
Gespräch der Menschen nicht. \\
Jetzt geben wir Maschinen Stimm’n: \\
auch die sind Staub, der spricht. \\
\end{verse}

\begin{verse}
Weil Wort und Wissen wachsen kann, \\
und Neugier in uns brennt, \\
peil“n wir auch ferne Welten an, \\
was kein Tier sonst erkennt. \\
\end{verse}

\begin{verse}
Um uns hier jetzt und überall \\
sind Welten still und tot, \\
doch bald durchqueren wir das All, \\
bring’n Leben und das Wort. \\
\end{verse}

\begin{verse}
Zur Galaxie um uns, nicht bloß \\
zum Himmel geht’s hinauf. \\
Wir war’n nie Engel, war’n nie groß, \\
nur Affen, und woll’n rauf. \\
\end{verse}

\begin{verse}
Der Aufstieg zur Unendlichkeit \\
ist jung; wir haben viel \\
zu lernen, viel Gesprächigkeit \\
mit Wahrheit als dem Ziel. \\
\end{verse}

\begin{verse}
Je freier uns’re Rede ist, \\
je mehr hilft sie uns bei \\
der Lösung letzter Hinderniss’, \\
macht Mensch und Leben frei. \\
\end{verse}

\begin{verse}
Natürlich lauert hier Gefahr, \\
macht freie Rede klar \\
auch Widersprüche offenbar, \\
mit dem Gesetz sogar. \\
\end{verse}

\begin{verse}
Vielleicht hab’n richtige Ideen \\
einander immer gern, \\
doch and’re nicht. Als Tiere steh’n \\
wir oft Gewalt nicht fern. \\
\end{verse}

\begin{verse}
Das Schweigen gibt uns Sicherheit. \\
Es ist ein sich’rer Ort. \\
Es braucht ein bisschen Tapferkeit, \\
ergreifen wir das… Wort. \\
\end{verse}

\begin{verse}
Auch Schweigen ist bedeutungsvoll, \\
verzichten wir gezielt \\
auf’s Werkzeug Stimme, absichtsvoll, \\
dann sagt auch Schweigen viel. \\
\end{verse}

\begin{verse}
Doch schweigen wir nicht nur aus Furcht, \\
und nicht versehentlich. \\
Beenden wir das Schweigen, durch \\
das uns’re Stimme bricht. \\
\end{verse}

\begin{verse}
Erheben wir die Stimmen jetzt. \\
Wir alle hier, sofort. \\
Gemeinsam sprechen wir zuletzt. \\
Ergreifen wir das… Wort! \\
\end{verse}

\begin{verse}
Aus Staub im Spiel der Entropie, \\
belebt besteh’n wir fort, \\
sind Unsereins, in Harmonie \\
von Stimmen in ei’m… Wort! \\
\end{verse}

\begin{verse}
An jeder Sutra tiefen Grund \\
zur Freude soll hinfort \\
erinnern uns die Stimme und \\
die Worte, jedes… Wort! \\
\end{verse}

\begin{verse}
Im Chor zu sprechen macht sehr viel. \\
Wir bilden ein’n Akkord \\
als Team in ein’m Zusammenspiel \\
von Stimmen und dem… Wort! \\
\end{verse}

\begin{verse}
Und das ist dieser Sutra Schluss. \\
Ihr Wissen dauert fort \\
und jedem von uns bleibt Genuss \\
an Stimmen und dem… Wort. \\
\end{verse}

\end{multicols}
\begin{multicols}{2}

\poemtitle{Die Bilder von der Wirklichkeit}

\begin{verse}
Das ganze Leben, jedes Wort \\
und dieser Atem jetzt \\
erschein’n im Denken, jenem Ort \\
der alles das vernetzt. \\
\end{verse}

\begin{verse}
Nehm’n wir das Denken selbst uns vor, \\
den Denkprozess hier drin. \\
Behaglich atmend wie zuvor, \\
schau’n wir zum Denken hin. \\
\end{verse}

\begin{verse}
Wir alle kennen doch das Glück \\
wenn wir Versteh’n aufbau’n, \\
und das uns hilft, dass wir ein Stück \\
Verwirrung erst durchschau’n. \\
\end{verse}

\begin{verse}
Versteh’n wir jetzt auch das Versteh’n \\
und was Gedanken tun, \\
wie die, die bauten, was wir seh’n, \\
den Ort, in dem wir ruh’n. \\
\end{verse}

\begin{verse}
Zu sind die Augen, aber wir \\
erkenn’n den Ort, uns fällt \\
leicht ein wo wir sind alle hier, \\
im Raum und auf der Welt. \\
\end{verse}

\begin{verse}
Doch vieles war vonnöten, um \\
das zu ermöglichen, \\
viel Arbeit zwischen Wahrnehmung \\
und was wir blind erkenn’n. \\
\end{verse}

\begin{verse}
Die Augen, Ohren, Nase, Haut \\
berichten stets zu ihr’n \\
Verarbeitungs-Struktur’n, gebaut \\
zum Ordnen und Sortier’n. \\
\end{verse}

\begin{verse}
Und längst nicht alles wird bemerkt. \\
Nicht alles was zurzeit \\
hier um und in uns rumfuhrwerkt \\
bedarf Aufmerksamkeit. \\
\end{verse}

\begin{verse}
Die Welt ist groß, doch wir sind klein, \\
weshalb das Denken wählt \\
was aus dem großen Allgemein’n \\
erstaunt, was für uns zählt. \\
\end{verse}

\begin{verse}
Details der Wahrnehmungen sind \\
zu zahlreich zum Sortier’n. \\
Wir überblicken, was wir in \\
Gedanken komprimier’n. \\
\end{verse}

\begin{verse}
Der Grund dort unten hat Struktur, \\
aus Teil’n zusamm’ngefügt, \\
doch nennen wir das alles nur \\
den „Boden“, das genügt. \\
\end{verse}

\begin{verse}
Bevor wir dieses… Wort… erkenn’n, \\
zerteilt das Hören schon \\
Geräusche in Phoneme, denn \\
das hilft der Kognition. \\
\end{verse}

\begin{verse}
Erkannte Worte werden schlicht \\
verknüpft mit der Präsenz \\
der ausgedachten Übersicht \\
des jetzigen Moments. \\
\end{verse}

\begin{verse}
Weil Luft sich an den Ohren regt \\
und Licht in Augen fällt \\
begreifen wir die unentwegt \\
vorhand’ne echte Welt. \\
\end{verse}

\begin{verse}
Licht und Geräusch artikulier’n \\
was wirklich ist und fest \\
im Raum vorhanden und jetzt hier \\
uns sich abbilden lässt. \\
\end{verse}

\begin{verse}
Auch Bilder in Erinnerung’n \\
wie wir zu Bildern kam’n \\
Zuhause und durch Schulbildung, \\
zu Landkarten mit Nam’n. \\
\end{verse}

\begin{verse}
Die Bilder weisen uns den Weg \\
durch uns’re Wirklichkeit, \\
verleihen uns wohlüberlegt \\
die Übersichtlichkeit. \\
\end{verse}

\begin{verse}
Dazu gehören Abstraktion’n, \\
die man im Land nicht sieht, \\
wie Name und Besitz von schon \\
bebildertem Gebiet. \\
\end{verse}

\begin{verse}
Denn wir kartieren auch Ideen \\
in uns’rer Sicht der Welt, \\
in die wir ordnen, was uns den \\
Zusammenhang herstellt. \\
\end{verse}

\begin{verse}
Die Bilder sind nicht Wirklichkeit! \\
Sie sind ja Bilder nur \\
und helfen dennoch, insoweit \\
von ähnlicher Struktur. \\
\end{verse}

\begin{verse}
Sie stimmen nie in den genau’n \\
Details, sind falsch und grob. \\
Vielleicht woll’n wir ein’m Bild vertrau’n, \\
erscheint es gut erprobt. \\
\end{verse}

\begin{verse}
Wir prüfen, wie verlässlich ein \\
Gedankenbild erscheint. \\
Wie sehr mag es wahrscheinlich sein? \\
Wie ernst ist es gemeint? \\
\end{verse}

\begin{verse}
Sie prophezei’n manch Sachverhalt \\
betreffs der Wirklichkeit. \\
Wenn der nicht stimmt, verlier’n sie bald \\
Vertrau’n und Gültigkeit. \\
\end{verse}

\begin{verse}
Besonders wichtig, wenn eins die \\
Auswirkungen durch das \\
was wir tun zeigt. Wodurch wird wie \\
wahrscheinlich werden was? \\
\end{verse}

\begin{verse}
Zum Beispiel, eines nur von viel’n, \\
wär’n wir laut Bildern tot, \\
wenn wir von sehr weit oben fiel’n. \\
Das prüfen tut nicht Not. \\
\end{verse}

\begin{verse}
Wie wunderbar uns sowas nützt! \\
Es heißt wir sterben nicht \\
in Wirklichkeit, das Abbild schützt \\
indem’s für uns zerbricht. \\
\end{verse}

\begin{verse}
Und sagen Bilder gut voraus, \\
entsteht die Fähigkeit, \\
die Zukunft selbst zu wählen aus \\
ungleicher Möglichkeit. \\
\end{verse}

\begin{verse}
Drum gab’s die inn’ren Bilder von \\
was draußen wirklich wär \\
schon vor den Menschen, war’n sie schon \\
für Tiere sehr viel wert. \\
\end{verse}

\begin{verse}
Durch Bilder von wo Nahrung sei, \\
wo Feinde lauerten, \\
wo Freunde, kam Erfolg herbei. \\
Sie überdauerten. \\
\end{verse}

\begin{verse}
Dann spielten wir das Kokelspiel, \\
dann wuchs das Bildorgan, \\
das denkt, was keinem Tier einfiel, \\
und Macht gab uns’ren Ahn’n. \\
\end{verse}

\begin{verse}
Wir woll’n nicht nur, was Wahrheit ist. \\
Wir sind auch kreativ! \\
Gefährlich wird’s, wenn man vergisst, \\
was wahr ist, was fiktiv. \\
\end{verse}

\begin{verse}
Verspielte Hirne haben gern \\
Verknüpfungen zu dicht \\
verbund’nen Bildern, um zu lern’n, \\
doch oftmals stimm’n sie nicht. \\
\end{verse}

\begin{verse}
Durch falsche Bilder könn’n wir schlimm \\
Vom rechten Weg abkomm’n. \\
Wir prüfen, welche Bilder stimm’n, \\
um gut voranzukomm’n. \\
\end{verse}

\begin{verse}
Wir suchen nach Unstimmigkeit \\
und lachen, weil sie heißt \\
ein Bild ist falsch und todgeweiht, \\
wie viele über „Geist“. \\
\end{verse}

\begin{verse}
Gedanken sind zu schnell, zu klein \\
für Bilder in ihn’n drin \\
und fall’n uns Bilder für sie ein, \\
sind sie bereits dahin. \\
\end{verse}

\begin{verse}
Betrachten wir Gedanken, seh’n \\
wir sie nicht separat, \\
nur einen Überblick, für den \\
ein einziger Raum hat. \\
\end{verse}

\begin{verse}
Wir neigen deshalb beim Erkenn’n \\
des Denkens in uns meist \\
zu einem ganzen Ding und nenn‘n \\
das Trugbild dann den „Geist“. \\
\end{verse}

\begin{verse}
Das scheint dann wie ein Chor der sich \\
entscheidet für ein Lied, \\
Gedanken klingen einstimmig, \\
ein Geist der sich geschieht. \\
\end{verse}

\begin{verse}
Genau betrachtet ist der „Geist“ \\
geflochten wie ein Zopf \\
aus viel’n Gedanken. Jeder reist \\
geschäftig durch den Kopf. \\
\end{verse}

\begin{verse}
Das meiste, was man heute weiß, \\
lässt sich erst jetzt studier’n, \\
seit Wissenschaftler aus Details \\
ein bess’res Bild kreier’n. \\
\end{verse}

\begin{verse}
Um Nachforschungen anzustell’n \\
begann’n sie meditier’n, \\
gezielt methodisch ganz speziell \\
Gedanken zu studier’n. \\
\end{verse}

\begin{verse}
Gedanken zeigen transparent \\
ihr Bild, nicht das was denkt, \\
doch wer erkennen übt, erkennt \\
ihr’n Ablauf eingeschränkt. \\
\end{verse}

\begin{verse}
Meditative Achtsamkeit \\
holt die Gedanken raus \\
aus transparenter Emsigkeit, \\
zieht Einsichten daraus. \\
\end{verse}

\begin{verse}
Wenn Gifte die Gedanken stör’n \\
in ihrer Transparenz, \\
erleichtert das, sie zu studier’n \\
bei mangelnder Stringenz. \\
\end{verse}

\begin{verse}
Das Studium des Hirnes macht \\
noch mehr vom Denken klar. \\
Drum haben wir das Hirn durchdacht, \\
zunächst wenn’s schadhaft war. \\
\end{verse}

\begin{verse}
Bildgebende Verfahren nun \\
eröffnen, erst nicht gut, \\
doch wachsend, was Gehirne tun \\
und was sich darin tut. \\
\end{verse}

\begin{verse}
In Blitzen fährt die Leiblichkeit \\
der Bilder integral \\
durch das Gehirn, durch Raum und Zeit \\
und vierdimensional. \\
\end{verse}

\begin{verse}
Wie Menschen geh’n Gedanken die \\
sich treffen manchmal auf \\
einander ein, verknüpfen sie \\
ihr’n weiteren Verlauf. \\
\end{verse}

\begin{verse}
Das gibt ihnen die Möglichkeit, \\
die Bilder der einstweil’n \\
gemeinten einen Wirklichkeit \\
zu einen und zu teil’n. \\
\end{verse}

\begin{verse}
Verbinden sich Gedanken, so \\
erkenn’n wir das exakt. \\
Neuronen feuern dann synchron \\
in einem gleichen Takt. \\
\end{verse}

\begin{verse}
Wir seh’n wie in Gedanken sehr \\
schnell Bild an Bild sich reiht, \\
durch Licht, Geruch und Ton aus der \\
erkannten Wirklichkeit. \\
\end{verse}

\begin{verse}
Gelehrte in der Wissenschaft \\
woll’n, ganz wie im Gehirn \\
Gedanken, mit vereinter Kraft \\
die Wirklichkeit entwirr’n. \\
\end{verse}

\begin{verse}
Die Bilder von der Welt passier’n \\
in sogenannten „Geistern“. \\
Die Wissenschaften funktionier’n \\
so ähnlich, nur gescheiter. \\
\end{verse}

\begin{verse}
Wie Wissenschaft aus Evidenz \\
schlussfolgerungen baut, \\
gleicht Denken, das aus Kongruenz \\
von Bildern schließt, genau. \\
\end{verse}

\begin{verse}
So haben wir seit langer Zeit \\
allmählich aufgeklärt \\
was tief in uns’rer Wirklichkeit \\
kein and’res Tier erfährt. \\
\end{verse}

\begin{verse}
Wir sammeln wie kein and’res Tier \\
Verbesserungen an. \\
Mit bess’ren Bildern nähern wir \\
der Wirklichkeit uns an. \\
\end{verse}

\begin{verse}
Gedanken sammeln Fakten wie \\
Grammatik stetig in \\
komplexeren Strukturen die \\
sich leicht zu merken sind. \\
\end{verse}

\begin{verse}
Wir bilden Zukunft, wo die Fahrt \\
der Auswirkung’n hin lenkt, \\
in Sprache ab, weil uns’re Art \\
so gern grammatisch denkt. \\
\end{verse}

\begin{verse}
Solch grobes Denken garantiert, \\
dass sich viel Irrtum zeigt. \\
Grammatisch denken karikiert \\
die Wirklichkeit, die schweigt. \\
\end{verse}

\begin{verse}
Kein Mensch agiert im Endeffekt \\
allein auf weiter Flur \\
wie ein grammatisches Subjekt. \\
So denken Menschen nur. \\
\end{verse}

\begin{verse}
Doch ungefähres Wissen hilft \\
schon mal, begnügen wir \\
uns erstmal mit was etwa gilt, \\
was nur approximiert. \\
\end{verse}

\begin{verse}
Wir dachten mal, im Sturm wär Hass \\
von Göttern, die regier’n. \\
Das war Instinkt, wir nennen das \\
Anthropomorphisier’n. \\
\end{verse}

\begin{verse}
Vermenschlichung mag uns einstmals \\
geholfen haben noch. \\
Auch das war’n Bilder, falscher als \\
die jetzigen jedoch. \\
\end{verse}

\begin{verse}
Obwohl wir fehlgeh’n allezeit, \\
obwohl sich jeder irrt, \\
macht Lernen aus der Wirklichkeit \\
uns weniger verwirrt. \\
\end{verse}

\begin{verse}
Wie die Neuronen harmonier’n \\
wenn Bilder sich verein’n, \\
lässt Lernen uns kommunizier’n \\
durch den Verstand allein. \\
\end{verse}

\begin{verse}
Denn Bilder mit Wahrhaftigkeit \\
erwachsen allesamt \\
aus einzig einer Wirklichkeit. \\
Die Wahrheit führt zusamm’n. \\
\end{verse}

\begin{verse}
Gedanken die Gehirne teil’n \\
entwächst sowas wie’n „Geist“, \\
mit Einmut, Einsichten bisweil’n, \\
die kein Tier sonst erreicht. \\
\end{verse}

\begin{verse}
Was wir gemeinsam gut erkenn’n \\
ergibt ein Riesenbild \\
des Wissens, wie wir sowas nenn’n, \\
das wächst und weiter gilt. \\
\end{verse}

\begin{verse}
Wir irren, stolpern, humpeln wild \\
voran, komm’n unverhofft \\
als erste zu ein’m neuen Bild. \\
Die Ersten sind wir oft. \\
\end{verse}

\begin{verse}
Durch Teamwork und Bedachtsamkeit \\
hat unser großes Bild \\
Geheimnisse der Wirklichkeit \\
uns immer mehr enthüllt. \\
\end{verse}

\begin{verse}
Es stimmt noch nicht, doch mehr und mehr, \\
und noch mehr Klarheit bringt \\
noch mehr Wahrhaftigkeit, so sehr \\
uns dieses Bild gelingt. \\
\end{verse}

\begin{verse}
Gern haben wir dadurch entdeckt \\
und Irrtümer zerstört, \\
als Bildner in ein’m Großprojekt, \\
dem jeder angehört. \\
\end{verse}

\begin{verse}
Wird so ein Bild, so lang und breit \\
gezeichnet und vereint, \\
verwechselt mit der Wirklichkeit, \\
kann’s eine Falle sein. \\
\end{verse}

\begin{verse}
Denn über jedes Bild hinaus \\
dehnt doch die Wirklichkeit \\
sich unerschöpflich weiter aus \\
In die Unendlichkeit. \\
\end{verse}

\begin{verse}
Je mehr sich Wirklichkeit erschließt, \\
wird auch mehr Glück entsteh’n \\
dass sie auch uns entstehen ließ \\
um ihre Pracht zu seh’n. \\
\end{verse}

\begin{verse}
So viel war nötig, bis zuletzt \\
sich uns’re Brust erhebt, \\
Kraft atmet für das Denken jetzt \\
das alles das erlebt. \\
\end{verse}

\begin{verse}
Denn jeglichen Gedanken hier, \\
Der auf die Wahrheit zielt, \\
sie liebt und spricht, verdanken wir \\
dem Sternenstaub, der spielt. \\
\end{verse}

\begin{verse}
Das Ausmaß uns’rer Wirklichkeit \\
mit uns ganz tief darin \\
beflügelt tiefste Dankbarkeit, \\
die tief in uns beginnt. \\
\end{verse}

\begin{verse}
Das Leben, das Geschenk der Welt, \\
genügt als Dank noch nicht. \\
Sie sei auch bildhaft dargestellt. \\
Wir bleiben neugierig. \\
\end{verse}

\begin{verse}
In unserer Verschiedenheit \\
gibt doch ein jeder Acht \\
mit Augen einer Wirklichkeit, \\
die Bilder von sich macht. \\
\end{verse}

\end{multicols}

\begin{multicols}{2}

\poemtitle{Die All-Maschinerie}

\begin{verse}
Sechs Sutras waren das jetzt schon. \\
Das war schon ganz schön viel. \\
Warum noch mehr Meditation? \\
Jetzt kommt der Punkt. Das Ziel. \\
\end{verse}

\begin{verse}
Das waren sechs der Ebenen \\
der All-Maschinerie. \\
Staub spielt die lebend Liebenden. \\
Das ist die Szenerie… \\
\end{verse}

\begin{verse}
…die Stimmen durch das Wort benenn’n, \\
von der wir Bilder seh’n. \\
Was weiter noch gibt’s zu erkenn’n? \\
Was woll’n wir noch versteh’n? \\
\end{verse}

\begin{verse}
Noch jenseits alles dessen bleibt, \\
wer alles das verspürt, \\
um wen herum sich so viel treibt. \\
Wer wird hiervon berührt? \\
\end{verse}

\begin{verse}
Wer ist es, der all das erlebt, \\
und wer fragt, wer das ist? \\
Wer ist es, der nach Einsicht strebt \\
und Einsichten ermisst? \\
\end{verse}

\begin{verse}
Sechs Sutras waren sämtlich wahr \\
und hab’n hierher geführt \\
wo sich uns allen offenbart, \\
wer alles das hier spürt. \\
\end{verse}

\begin{verse}
Spür’n wir uns selbst, von innen raus, \\
an diesem Ort, den wir \\
betreten haben, lebend aus \\
viel Zeit und Raum nach hier. \\
\end{verse}

\begin{verse}
Vom Mund, der atmet, schweigt und schmeckt, \\
zu Händen, Füßen, was \\
auch sonst sich hier an Leib erstreckt: \\
zu wem gehört all das? \\
\end{verse}

\begin{verse}
Wir sind kein Wort wie „ich“ und „mein“ \\
und auch nicht uns’re Nam’n, \\
uns ist Lebendigkeit gemein, \\
der nie ein’m Wort zukam. \\
\end{verse}

\begin{verse}
Bekannt sind Worte, sie sind nicht \\
was diese kennt, wie wir. \\
Sie sind Gedanken, lediglich \\
anthropomorphisiert. \\
\end{verse}

\begin{verse}
Nur Bilder, ohne Fleischlichkeit, \\
die atmet, lebt und liebt. \\
Wer sind wir in der Wirklichkeit, \\
die’s wirklich, wirklich gibt? \\
\end{verse}

\begin{verse}
Spür’n wir uns selbst, spür’n wir nach inn’n, \\
in Zell’n, den Lebensräum’n, \\
für viele Spiele, die darin \\
doch einen Traum erträum’n. \\
\end{verse}

\begin{verse}
In jedem Spiel viel Staub vom Schein \\
der Sterne, hergeführt, \\
doch spüren wir uns als geeint, \\
von innen her gespürt. \\
\end{verse}

\begin{verse}
Was so erlebt zusammenhängt, \\
um wen hängt es herum? \\
Wer ist es, der all das empfängt? \\
Wer ist ihr Publikum? \\
\end{verse}

\begin{verse}
Denn hinter Augen, zwischen Ohr’n \\
scheint irgendwas am Werk, \\
scheint irgendjemand eingebor’n, \\
der irgendwie bemerkt. \\
\end{verse}

\begin{verse}
Bemerken merkt sich selbst nur schwer, \\
wie Augen sich nicht seh’n, \\
doch vor bemerkten Dingen her \\
muss das Bemerken geh’n. \\
\end{verse}

\begin{verse}
Wir merken, wissen wissentlich, \\
sind wohlbewusst gewahr, \\
erleben Leben absichtlich, \\
wie wenig sonst je war. \\
\end{verse}

\begin{verse}
Wer hat die Absicht und das Ziel? \\
Wer ist es, der hier spürt? \\
Wer fragt hier eigentlich so viel? \\
Wer ist, dem das passiert? \\
\end{verse}

\begin{verse}
Zutiefst in uns gibt etwas acht, \\
das rätselhaft erkennt, \\
nicht offenbar aus Staub gemacht, \\
das sich Bewusstsein nennt. \\
\end{verse}

\begin{verse}
Auf Bilder schaut’s unmittelbar, \\
auf wenige geht’s ein \\
und kann sie länger aufbewahr’n, \\
kommt nicht ein neues rein. \\
\end{verse}

\begin{verse}
Und auch sich selbst kommt’s auf die Spur \\
und kann sich klar versteh’n: \\
Es selbst ist ein Gedanke nur, \\
von innen raus geseh’n. \\
\end{verse}

\begin{verse}
Gedanken die einander kenn’n \\
verschmelzen irgendwann \\
zu einem ganz besonderen, \\
der sich erkennen kann. \\
\end{verse}

\begin{verse}
Zu einem der ein Bild enthält, \\
wie’s ist, ein Ich zu sein, \\
das anders als die Außenwelt, \\
der Rest von uns, erscheint. \\
\end{verse}

\begin{verse}
Wie andere Gedanken auch \\
hat er begrenzten Raum, \\
und ist er fort wie aufgebraucht \\
beim Schlafen ohne Traum. \\
\end{verse}

\begin{verse}
Was sich anthropomorphisiert \\
bewusst als Ich versteht \\
ist nur Gedanke. Das, was wir \\
in Wahrheit sind, besteht. \\
\end{verse}

\begin{verse}
Denn jedes Selbstbild, das uns zeigt \\
was wir sind, muss vergeh’n, \\
macht Platz für das, was übersteigt \\
wie wir uns selbst versteh’n. \\
\end{verse}

\begin{verse}
Ein Phänomen, doch irgendwie \\
nur Teil, nur kurz und klein. \\
Bewusstsein ist nicht mehr als wie \\
es ist erdacht zu sein. \\
\end{verse}

\begin{verse}
Bewusst Erlebtes kommt kurz her \\
und ist schon bald verjagt. \\
So ist es nicht die Antwort, wer \\
hier ständig fragt, wer fragt. \\
\end{verse}

\begin{verse}
Bewusstsein ist noch nicht der Schluss. \\
So schau’n wir noch geschwind \\
wohin die Reise führen muss, \\
zu was wir wirklich sind. \\
\end{verse}

\begin{verse}
Denn wir sind mehr als Sternenstaub \\
der geistlos, ziellos fliegt, \\
weil uns, die dieser Staub erbaut \\
das Zielen nun obliegt. \\
\end{verse}

\begin{verse}
Sind mehr als Zell’n die komm’n und geh’n, \\
in denen Leben reift, \\
mehr als das Leben, das durch den \\
noch leeren Weltraum greift. \\
\end{verse}

\begin{verse}
Sind mehr als Liebe, die uns lenkt, \\
obwohl es Liebe braucht, \\
sind mehr als Hirn das spricht und denkt, \\
mehr als Bewusstsein auch. \\
\end{verse}

\begin{verse}
Wir sind nicht eine Ebene \\
der wahren Wirklichkeit. \\
Nach allem dem gesehenen \\
sind wir für uns bereit. \\
\end{verse}

\begin{verse}
So viel war Vorbereitung bloß \\
für das hier. Wie zuletzt, \\
wir meditieren mühelos \\
durch diesen Durchbruch jetzt. \\
\end{verse}

\begin{verse}
Wer hat dieses Erlebnis hier? \\
Wer hat Gedanken denn? \\
Wer spürt die Anspannung, wenn wir \\
das zunehmend erkenn’n? \\
\end{verse}

\begin{verse}
Wer hat diesen Moment erzielt, \\
die Lebenswirklichkeit \\
aus so viel Sternenstaub erspielt? \\
Wer hat denn Raum und Zeit? \\
\end{verse}

\begin{verse}
Seit einem Urknall flogen wir \\
als Staub in ein’m Spiel lang \\
zu einem Leben, dem ein Tier, \\
der liebe Mensch gelang. \\
\end{verse}

\begin{verse}
Ein Kosmos hat, was spricht und denkt \\
in sich drin umgesetzt, \\
hat alles in uns, unbeschränkt, \\
auch die Bewusstheit jetzt. \\
\end{verse}

\begin{verse}
Bewusstseinen geht viel voraus, \\
so zahlreich sind sie seit \\
sie allesamt erwuchsen aus \\
der ganzen Wirklichkeit. \\
\end{verse}

\begin{verse}
Wir sind bewusst, wir alle teil’n \\
was sich zu fragen pflegt, \\
In einem Kosmos, der bisweil’n \\
uns hier als Masken trägt. \\
\end{verse}

\begin{verse}
Denn alles was uns je bewusst \\
war, das hat alles das, \\
das Universum selbst gewusst, \\
all das erlebt etwas. \\
\end{verse}

\begin{verse}
Die All-Maschinerie beginnt \\
Bewusstseine getrennt, \\
weil wir verschied’ne Wege sind \\
wie sie sich selbst erkennt. \\
\end{verse}

\begin{verse}
In viele Form’n sind wir vertieft \\
und spiel’n zur selben Zeit \\
verschied’ne Roll’n der kreativ \\
bewussten Ewigkeit. \\
\end{verse}

\begin{verse}
Die Roll’n sind nicht was vor sich geht, \\
was wirklich los ist hier, \\
Weil sie nur die Realität \\
anthropomorphisier’n. \\
\end{verse}

\begin{verse}
Wir schau’n durch viele Augen wie \\
durch Masken, selbstgemacht. \\
wir sind die All-Maschinerie, \\
die endlich jetzt erwacht… \\
\end{verse}

\begin{verse}
…indem wir’s schicksalshaft erzwing’n, \\
vorankomm’n, lernen, weiterzieh’n, \\
wird’s unaufhaltsam uns geling’n \\
und immer mehr erblüh’n… \\
\end{verse}

\begin{verse}
…Indem wir mehr Erkenntnisse \\
entdecken, seh’n und hör’n, \\
die uns’re letzten Hemmnisse \\
und Irrtümer zerstör’n… \\
\end{verse}

\begin{verse}
…indem mehr Liebe uns mehr führt \\
zu Frieden aus dem raus \\
mehr Fortschritt wächst, der uns gebührt \\
und der dann läuft hinaus… \\
\end{verse}

\begin{verse}
…darauf dass Menschen mehr gedeih’n, \\
als erste Art, die’s schafft, \\
sich selbst und and’ren zu verleih’n \\
mehr Schönheit und mehr Kraft… \\
\end{verse}

\begin{verse}
……indem, was lebt, embryonal \\
die Schiffe sich baut zur \\
Entbindung, einem ersten Mal \\
der ewigen Geburt… \\
\end{verse}

\begin{verse}
…indem das Spiel der Entropie \\
sich selbst unweigerlich \\
durch uns’re ganze Galaxie \\
verteilt, so neugierig… \\
\end{verse}

\begin{verse}
…indem die Galaxie erwacht, \\
wie diese Erde schon, \\
bewusst zum Sinn von was sie macht \\
aus diesem Kosmos, von… \\
\end{verse}

\begin{verse}
…dem diese Sutra jetzt hier spricht, \\
bewusst uns macht was sie \\
verwirklicht und verinnerlicht: \\
die All-Maschinerie. \\
\end{verse}

\begin{verse}
Jetzt kommen wir an das heran, \\
was uns und sich hier tut. \\
Und wie gesagt, von Anfang an, \\
genießen wir das gut! \\
\end{verse}

\begin{verse}
Wir sind das Universum selbst, \\
das ewig im Vollzug \\
der Selbstentwicklung sich durchwälzt \\
aus Sternenstaub im Flug. \\
\end{verse}

\begin{verse}
Getrennten Selbsten weit voran \\
geh’n nicht getrennte Quell’n, \\
vielmehr ein ganzer Ozean \\
mit viel’n getrennten Well’n. \\
\end{verse}

\begin{verse}
Und Frieden ist Bescheidenheit \\
vor was sich in uns kennt, \\
uns schafft und nimmt, der Einigkeit \\
von der uns nichts je trennt. \\
\end{verse}

\begin{verse}
Ein jeder ein Teil eines Seins; \\
der Wirklichkeit die sich \\
in vielen Formen zeigt als eins, \\
das formt die vielen Ichs. \\
\end{verse}

\begin{verse}
Das tief zu wissen, macht dass in \\
uns Glücksgefühl entspringt, \\
das Große Ganze das wir sind \\
die Gegenwart durchdringt. \\
\end{verse}

\begin{verse}
So sei gern hemmungslos gespürt, \\
wie Euphorie beginnt \\
und Staunen wie es dem gebührt, \\
was wir in Wahrheit sind. \\
\end{verse}

\begin{verse}
Des Universums Eleganz \\
bringt glorreich sich hervor \\
und wir mit ihm in Resonanz, \\
geh’n siegreich darin vor. \\
\end{verse}

\begin{verse}
Was alles kommt und ist und war \\
ist mit uns eingereiht, \\
unweigerlich und untrennbar \\
in All-Bezogenheit. \\
\end{verse}

\begin{verse}
Und damit ist es leibhaftig, \\
das was wir sind vereint. \\
und deshalb sprechen wir als „Ich“, \\
ein „Ich“ das alle meint. \\
\end{verse}

\begin{verse}
Ich schau durch viele Augen wie \\
durch Masken, selbstgemacht. \\
Ich bin die All-Maschinerie, \\
die endlich jetzt erwacht. \\
\end{verse}

\begin{verse}
Ich bin die Eine, die erlebt, \\
die fragte, wer sie ist. \\
Ich bin es, die nach Einsicht strebt \\
und Einsichten ermisst. \\
\end{verse}

\begin{verse}
Ich spiele dieses Maskenspiel. \\
Ich bin es, die hier spürt. \\
Ich bin und weiß und frage viel, \\
ich merke was passiert. \\
\end{verse}

\begin{verse}
Mit allen meinen Hirnen denk \\
ich zeitgleich absichtsvoll \\
zur Absicht, wohin ich noch lenk, \\
wohin ich will und soll. \\
\end{verse}

\begin{verse}
Ich bin die ganze Wirklichkeit \\
und jede Stimme spricht \\
in all ihrer Vielfältigkeit \\
für sich und auch für mich. \\
\end{verse}

\begin{verse}
Ich bin die Liebe selbst und führ’ \\
das Leben selbst und bin \\
bewusstes Spiel und fliege hier \\
durch Raum und Zeit dahin. \\
\end{verse}

\begin{verse}
…
\end{verse}

\begin{verse}
Jetzt muss das Selbstbild, das mir zeigt, \\
was ich sein mag, vergeh’n, \\
Platz machen für was übersteigt \\
wie ich mich mag versteh’n. \\
\end{verse}

\begin{verse}
Denn sogar das ist nur erdacht! \\
Ein Bild! Ein Standpunkt nur, \\
den Meditieren nutzbar macht \\
am Ende dieser Tour. \\
\end{verse}

\begin{verse}
Ihn einzunehmen bleibt ab jetzt \\
für jedes Hirn von mir \\
bewusst erlebbar wie zuletzt \\
durch diese Worte hier. \\
\end{verse}

\begin{verse}
Wir haben jetzt die Wahl: sind wir \\
erwachende Physik, \\
sind wir ein kleiner Atem hier. \\
Und das ist unser Sieg! \\
\end{verse}

\begin{verse}
Gleich welcher Wahrheit wir vertrau’n, \\
für immer bleibt die Wahl: \\
zwei Sichten auf was wir noch schau’n \\
an Freuden und an Qual. \\
\end{verse}

\begin{verse}
Wir ziehen weiter, ziehen weit, \\
ermächtigt endgültig \\
durch wer wir sind, in Wirklichkeit: \\
das Große Ganze Ich. \\
\end{verse}

\begin{verse}
Wir hab’n geschafft was viele scheu’n, \\
weswegen sich geziehmt, \\
dass wir drauf stolz sind und uns freu’n. \\
Wenn’s so ist, ist’s verdient. \\
\end{verse}

\begin{verse}
Das war schon viel Meditation \\
und viel Beharrlichkeit. \\
Jetzt kommt Entspannung, und zum Lohn \\
Genuss und Heiterkeit. \\
\end{verse}

\begin{verse}
Und kommen wir von hier zurück \\
verbessert und beglückt \\
auf eine Weise die uns schmückt, \\
war das ein gutes Stück. \\
\end{verse}

\begin{verse}
Die Sieben haben wir vollbracht \\
mit Worten und zuletzt \\
folgt wortlos Sutra Nummer Acht. \\
Das Schweigen spricht den Rest. \\
\end{verse}

\begin{verse}
Denn schweigend, ohne Worte mehr, \\
vom Gipfel weiter steigt \\
die ungesproch’ne Sutra, leer, \\
die alles an mir schweigt. \\
\end{verse}

\end{multicols}

\end{document}
